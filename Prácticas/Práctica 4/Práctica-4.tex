\documentclass[12pt]{article}
\textheight=240mm
\evensidemargin=-0cm
\oddsidemargin=-0cm
\textwidth=160mm
\topmargin=-1.5cm
\usepackage{bm}
\usepackage{bbm}
\newcommand{\be}{\begin{equation}}
\newcommand{\ee}{\end{equation}}
\newcommand{\ben}{\begin{eqnarray}}
\newcommand{\een}{\end{eqnarray}}
\newcommand{\half}{\mbox{${\textstyle\frac{1}{2}}$}}
\newcommand{\quart}{\mbox{${\textstyle\frac{1}{4}}$}}
\newcommand{\sen}{\mbox{${\rm sen}$}}
\newcommand{\senh}{\mbox{${\rm senh}$}}
\newcommand{\arcsen}{\mbox{${\rm arcsen}$}}
\newcommand{\dxy}{\mbox{$\frac{dx}{dy}$}}
\newcommand{\dyx}{\mbox{$\frac{dy}{dx}$}}
\newcommand{\rdo}{\mbox{${\Rightarrow}$}}
%\newcommand{\arccos}{\mbox{${\rm arccos}$}}
%\newcommand{\arctan}{\mbox{${\rm arctan}$}}
%\newcommand{\cos}{\mbox{${\rm cos}$}}
%\newcommand{\tan}{\mbox{${\rm tan}$}}
\newcommand{\pid}{\mbox{${\textstyle\frac{\pi}{2}}$}}
\newcommand{\pic}{\mbox{${\textstyle\frac{\pi}{4}}$}}
\newcommand{\pitt}{\mbox{${\textstyle\frac{\pi}{3}}$}}
\newcommand{\pis}{\mbox{${\textstyle\frac{\pi}{6}}$}}
%\newcommand{\quart}{\mbox{$\case{1}{4}$}}
\renewcommand{\thepage}{-- \arabic{page} --}
\begin{document}
\begin{center}
\subsection*{Seminario de Mec\'anica Cu\'antica / \\Teor\'{\i}a de la Informaci\'on Cu\'antica}
\subsection*{Pr\'actica IV (Curso 2020)}
\end{center}
{\bf I. Evoluci\'on de sistemas cu\'anticos abiertos}\\
1) Dado un estado inicial producto $\rho_{AB}=\rho_A\otimes\rho_B$ de un sistema bipartito, % |\Phi_A\rangle\otimes|\Phi_B\rangle$, 
y  un operador evoluci\'on
$U_{AB} = \exp[-iH_{AB}t]$, derivar la expresi\'on
\[\rho'_A={\rm Tr}_B\,[U_{AB}\rho_{AB}U_{AB}^\dagger]=\sum_{\alpha} E_\alpha\, \rho_A E_\alpha^\dagger\]
%con $\rho_{AB} = |\Psi_{AB}\rangle\langle\Psi_{AB}|$, $\rho_A=|\psi_A\rangle\langle\psi_A|$ y $\sum_{\alpha}E^\dagger_\alpha 
dando la forma expl\'{\i}cita de de $E_\alpha$. Probar tambi\'en que $\sum_{\alpha} E_\alpha^\dagger E_\alpha=I_A$.\\ \hfill\break
2) a) Mostrar que para el caso de la evoluci\'on por un operador control not \\
$U_{AB} = |0\rangle\langle 0|\otimes I +|1\rangle\langle 1|\otimes X$  de un estado producto de dos qubits $\rho_{AB}=\rho_{A}\otimes |\psi_{B}\rangle\langle\psi_B|$, % = |\psi_A\rangle\otimes|\psi_B\rangle$, se 
se obtiene 
\[\rho'_A=(1-p)\rho_A+pZ\rho_A Z\]
Dar el valor expl\'{\i}cito de $p$ e identificar los operadores $E_0$ y $E_1$. \\
Interpretar el resultado como disminuci\'on de elementos no diagonales en una determinada base  e indicar para cuales
estados iniciales $|\psi_B\rangle$ se obtiene i) $p = 1/2$ (decoherencia completa) ii) $p = 0$ (ausencia de decoherencia). Discutir la variaci\'on de la entrop\'{\i}a de $\rho_A$. \\
b) Hallar la evoluci\'on de $\rho_A$ para el mismo operador $U_{AB}$ y un estado
inicial producto general %i) $\rho_A\otimes|\psi_B\rangle\langle\psi_B|$  y ii) 
$\rho_A\otimes \rho_B$. 
Indicar que sucede si $\rho_B=I_B/d_B$. \\\hfill\break
3) Considerar los operadores 
$E_1=|0\rangle\langle 0|+\sqrt{1-p}|1\rangle\langle 1|$, 
$E_2=\sqrt{p}|0\rangle\langle 1|$, con $p\in[0,1]$. \\
a) Mostrar que satisfacen en general $\sum_{i=1,2} E_i^\dagger E_i=I$, 
pero $\sum_{i=1,2} E_i E_i^\dagger\neq I$ si $p\neq 0$. \\
b) Dar una expresi\'on de $\rho'_A=\sum_{i=1,2} E_i\rho_A E_i^\dagger$ para un operador densidad $\rho_A$ general de un qubit. Interpretar y discutir la variaci\'on de la entrop\'{\i}a de $\rho_A$. \\\hfill\break
4) Considerando un \'atomo de 2 niveles con estados $|0
\rangle$, $|1\rangle$ con   energ\'{\i}as $0$ y $\varepsilon$, y los estados de campo $|0\rangle$ y $|1\rangle=a^\dagger_w|0\rangle$, con $\hbar\omega=\varepsilon$, mostrar que si 
\[U_{AC}=\left(\begin{array}{cccc}1&0&0&0\\0&\cos\theta&\sin\theta&0\\0&-\sin\theta&\cos\theta&0\\
0&0&0&1\end{array}\right)
\] 
en la base producto estándar, 
entonces 
\[\rho'_A={\rm Tr}_C\,[U_{AC}\,\rho_A\otimes|0\rangle\langle 0|\,U_{AC}^\dagger]=
E_0\rho_A E_0^\dagger+E_1\rho_A E_1^\dagger\]
con 
\[E_0=\left(\begin{array}{cc}1&0\\0&\cos\theta\end{array}\right)\,,\;\;\;\;
E_1=\left(\begin{array}{cc}0&\sin\theta\\0&0\end{array}\right)\]
Hallar tambi\'en 
$\rho'_A$ para $\rho_A=p_0|0\rangle\langle 0|+p_1|1\rangle\langle 1|$. \\

Determinar tambi\'en un $H_{AC}$ y un tiempo $t$ tal que $\exp[-iH_{AC}t]=U_{AC}$. 
\hfill\break \\ \\
{\bf II. Algoritmo de B\'usqueda de Grover} \\
1) Considerar el estado buscado $|B\rangle =\frac{1}{\sqrt{M}}\sum\limits_{f(j)=1}|j\rangle$, ($M$ denota el n\'umero de estados $|j\rangle$  tales que $f(j)=1$),  el estado ortogonal $|A\rangle=\frac{1}{\sqrt{N-M}}\sum\limits_{f(j)=0}|j\rangle$ y el estado inicial 
($N=2^n$)
\[|\Phi\rangle=H^{\otimes n}|0\rangle={\textstyle\frac{1}{\sqrt{N}}\sum_j|j\rangle=\sqrt{\frac{M}{N}}|B\rangle+\sqrt{\frac{N-M}{N}}|A\rangle}\]
Probar que si $O|j\rangle = (-1)^{f(j)}|j\rangle$, la iteraci\'on de Grover 
$G = (2|\Phi\rangle\langle\Phi|-I)\,O$ equivale
a una rotaci\'on en sentido antihorario de \'angulo $2\theta$, con $\sin\theta =\sqrt{\frac{M}{N}}$, en el subespacio
generado por los estados $|A\rangle$  y $|B\rangle$. 
\hfill\break\break
2) Si $H = \hbar\omega(|B\rangle\langle B| + |\Phi\rangle\langle\Phi|)$, mostrar que existe un tiempo $t$ independiente de $|B\rangle$ tal
que $\exp[-iHt/\hbar]|\Phi\rangle = |B\rangle$.  El problema de b\'usqueda puede pues reducirse al problema 
de la simulaci\'on del Hamiltoniano $H$.
\end{document}