%%%%%%%%%%%%%%%%%%%%%%%%%%%%%%%%%%%%%%%%%%%%%%%%%
%%%% Manuscript 
%%%% Authors: N. Canosa R. Rossignoli,
%%%% Title: Fluctuations and parity effects in the paramagnetic breakdown of 
%%%%        small superconductors 
%%%% Figures: 3 (separate postscript files) 
%%%%%%%%%%%%%%%%%%%%%%%%%%%%%%%%%%%%%%%%%%%%%%%%%
%
% Beginning of Manuscript 
% 
%%%%%%%%%%%%%%%%%%%%%%%%%%%%%%%%%%%%%%%%%%%%%%%%%
\documentclass[12pt]{article}
\textheight=240mm
\evensidemargin=-0cm
\oddsidemargin=-0cm
\textwidth=160mm
\topmargin=-1.5cm
\usepackage{bm}
\usepackage{bbm}
\newcommand{\be}{\begin{equation}}
\newcommand{\ee}{\end{equation}}
\newcommand{\ben}{\begin{eqnarray}}
\newcommand{\een}{\end{eqnarray}}
\newcommand{\half}{\mbox{${\textstyle\frac{1}{2}}$}}
\newcommand{\quart}{\mbox{${\textstyle\frac{1}{4}}$}}
\newcommand{\sen}{\mbox{${\rm sen}$}}
\newcommand{\senh}{\mbox{${\rm senh}$}}
\newcommand{\arcsen}{\mbox{${\rm arcsen}$}}
\newcommand{\dxy}{\mbox{$\frac{dx}{dy}$}}
\newcommand{\dyx}{\mbox{$\frac{dy}{dx}$}}
\newcommand{\rdo}{\mbox{${\Rightarrow}$}}
%\newcommand{\arccos}{\mbox{${\rm arccos}$}}
%\newcommand{\arctan}{\mbox{${\rm arctan}$}}
%\newcommand{\cos}{\mbox{${\rm cos}$}}
%\newcommand{\tan}{\mbox{${\rm tan}$}}
\newcommand{\pid}{\mbox{${\textstyle\frac{\pi}{2}}$}}
\newcommand{\pic}{\mbox{${\textstyle\frac{\pi}{4}}$}}
\newcommand{\pitt}{\mbox{${\textstyle\frac{\pi}{3}}$}}
\newcommand{\pis}{\mbox{${\textstyle\frac{\pi}{6}}$}}
%\newcommand{\quart}{\mbox{$\case{1}{4}$}}
\renewcommand{\thepage}{-- \arabic{page} --}
\begin{document}
\begin{center}
\subsection*{Seminario de Mec\'anica Cu\'antica / \\Teor\'{\i}a de la informaci\'on cu\'antica}
\subsection*{Pr\'actica I (Curso 2020)}
\end{center}
I {\bf Operador densidad.}\\
\\
I.1 Demostrar que un operador densidad $\rho$ describe un estado puro sii $\rho^2=\rho$. \\
\\
I.2 Mostrar que si $\rho_i$, $i=1,\ldots,m$, son operadores densidad, entonces \[\rho=\sum_{i=1}^m p_i \rho_i,\;\;p_i\geq 0,\;\;\sum_{i=1}^m p_i=1\,,\] es tambi\'en un operador densidad. 
Esto muestra que el conjunto de operadores de
densidad para un dado sistema es un conjunto convexo.\\ \\
%Si alg\'un $p_i$ es negativo pero se sigue cumpliendo que $\sum_i p_i=1$, puede $\rho$ seguir siendo un operador densidad?\\
I.3 Mostrar que $\rho=p|0\rangle\langle 0|+(1-p)|1\rangle\langle 1|$, con  $p\in(1/2,1)$, puede ser escrito  como 
\[\rho=q|\alpha\rangle\langle\alpha|+(1-q)|\beta\rangle\langle\beta|\]
con $q\in[1-p,p]$ arbitrario y $|\alpha\rangle$, $|\beta\rangle$ estados normalizados. Determine $|\alpha\rangle$ y $|\beta\rangle$ e interprete ambas representaciones. \\
\\
I.4  Mostrar que la matriz densidad m\'as general 
para un qubit puede escribirse como 
\[\rho=\half(I+\bm{r}\cdot\bm{\sigma})\]
donde $\bm{r}=(r_x,r_y,r_z)$ es un vector arbitrario con $|\bm{r}|\leq 1$, 
y $\vec{\sigma}=(X,Y,Z)\equiv(\sigma_x,\sigma_y,\sigma_z)$ las matrices de Pauli. Determine los autovalores de $\rho$ e indique en qu\'e casos $\rho$ representa un estado puro. Exprese tambi\'en $\bm{r}$ en t\'erminos de $\langle\bm{\sigma}\rangle={\rm Tr}\,\rho\,\bm{\sigma}$.  \hfill\break\\
I.5 Generalizar I.4 a un sistema de dos qubits. \\ \\
I.6 Determinar todos los valores posibles de $x$ para los cuales 
\[\rho=x|\Phi\rangle\langle\Phi|+(1-x)I_d/d\]
con $|\Phi\rangle$ un estado normalizado e $I_d$ la identidad de $d\times d$ ($d={\rm Tr}\, I_d$ es la dimensi\'on del espacio de estados), es un operador densidad. Interpretar este estado. \\\hfill\break
II {\bf Estados de sistemas compuestos. Entrelazamiento.}\\\\
\noindent
II.1 Para un sistema de dos qubits, escribir expl\'{\i}citamente la matriz que representa a 
$\rho_{AB}=|\Phi_{AB}\rangle\langle\Phi_{AB}|$ en la base computacional  
$\{|00\rangle,|01\rangle,|10\rangle,|11\rangle\}$ para:\\ 	
\[ a)\; %$|\Phi_{AB}\rangle=|00\rangle$  b)  
|\Phi_{AB}\rangle=\frac{|00\rangle\pm|11\rangle}{\sqrt{2}}\,,\;\;\;\;\;\;
b) \;|\Psi_{AB}\rangle=\frac{|00\rangle+|10\rangle-|01\rangle-|11\rangle}{2}\]%\;\;  \hfill\break
Verificar  en todos los casos que los autovalores de $\rho$ son $(1,0,0,0)$. \hfill\break
\newpage
\noindent
II.2 Hallar la matriz densidad reducida $\rho_A={\rm Tr}_B\rho_{AB}$ en todos los casos anteriores, y a partir de ella evaluar la entropía de entrelazamiento del estado.\\ \\%$E(A,B)=S(\rho_A)=S(\rho_B)$. \hfill\break
II.3  Hallar la descomposici\'on de Schmidt de los estados anteriores. \\\\
II.4 Para $|\alpha|^2+|\beta|^2=1$, hallar la descomposici\'on de Schmidt del estado  \[{\textstyle|\Psi_{AB}\rangle=\alpha\frac{|00\rangle+|11\rangle}{\sqrt{2}}+\beta
\frac{|01\rangle+|10\rangle}{\sqrt{2}}}\]
y a partir de ella indicar a) cuando el estado ser\'a separable b) cuando ser\'a entrelazado c) en qu\'e caso el entrelazamiento ser\'a m\'aximo.\\\\
II.5 a) Indicar en qu\'e se diferencian el estado de Bell $|\Psi_{AB}\rangle=\frac{|01\rangle+|10\rangle}{\sqrt{2}}$ y el estado descripto por el operador densidad \[\rho_{AB}=\frac{1}{2}(|01\rangle\langle 01|+|10\rangle\langle 10|)\]
b) Indicar si es posible distinguirlos mediante \\
i) el valor medio de un observable local 
$ O_A\otimes I_B$. \\
ii) el valor medio de un observable $O=O_A\otimes O_B$.  

%Hallar observablea del tipo  $O=O_A\otimes O_B$ que logre %distinguirlos y otro que no logre distinguirlos. 
\end{document}
correspondiente y=(|00\rangle+|11\rangle)/\sqrt{2}$ y $I=I_A\otimes I_B$, 
es un operador densidad. Cual es su interpretaci\'on f\'{\i}sica?\hfill\break 
II. Estados compuestos

%
% 
%
%a) Muy f\'aciles: \hfill\break
%1) Probar que todo operador positivo $A$ ($\langle %\phi|A|\phi\rangle\geq 0$ (y por ende real)  
%$\forall|\phi\rangle$) es necesariamente herm\'{\i}tico. 
%(Sug.: $A=A_r+iA_i$, con $A_r=(A+A^\dagger)/2$, %$A_i=(A-A^\dagger)/(2i)$). 
%\hfill\break
0) Demostrar que $\rho$ describe un estado puro sii $\rho^2=\rho$.\hfill\break	
3) Hallar los operadores densidad reducidos 
$\rho_A$ y $\rho_B$ para 
\[|\Phi\rangle=\sqrt{p}|ab\rangle+e^{i\phi}\sqrt{q}|\bar{a}\bar{b}\rangle\]
con $p\geq 0$, $q\geq 0$, $p+q=1$, y $\langle a|\bar{a}\rangle=0$, 
$\langle b|\bar{b}\rangle=0$, siendo $|a\rangle$, $|b\rangle$ 
estados normalizados arbitrarios de un qubit. En que caso 
es $\rho_A=I_A/2$, $\rho_B=I_B/2$?\hfill\break
4) Mostrar que $\rho=p|0\rangle\langle 0|+q|1\rangle\langle 1|$, 
con $p\geq 0$, $q\geq 0$, $p+q=1$,   puede tambi\'en escribirse como 
\[\rho=\half(|\phi_+\rangle\langle\phi_+|+|\phi_-\rangle\langle\phi_-|)\]
con $|\phi_{\pm}\rangle=\sqrt{p}|0\rangle\pm \sqrt{q}|1\rangle$. 
%Son $|\phi_{\pm}\rangle$ ortogonales? 
%Encontrar tambi\'en la matriz unitaria que vincula a 
%$|\tilde{0}\rangle=\sqrt{p}|0\rangle,|\tilde{1}\rangle=\sqrt{q}|1\rangle$  con 
%$|\tilde{\pm}\rangle=\frac{1}{\sqrt{2}}|\pm\rangle$. 
Como se interpretan ambas representaciones?
\hfill\break
5) Probar que si $\rho_i$, $i=1,\ldots,n$ son matrices densidad para un sistema dado $\Rightarrow$ la combinaci\'on convexa 
\[\rho=\sum_{i=1}^n p_i\rho_i,\;\;\sum_{i=1}^n p_i=1,\;\;p_i\geq 0\]
es tambi\'en una matriz densidad. 
Si $p_1<0$ pero $\sum_{i=1}^n p_i=1$, puede seguir siendo $\rho$ una matriz densidad? \hfill\break
6) Probar que si en un sistema bipartito $\rho=\rho_A\otimes\rho_B$ 
 $\Rightarrow$ ${\rm Tr}_B\rho=\rho_A$, 
${\rm Tr}_A\rho=\rho_B$.  \hfill\break
%b) F\'aciles:\hfill\break
%b) Probar que $\rho$ describe un estado puro sii $|\vec{r}|=1$. \hfill\break
8) Consideremos el estado de Bell \[|\Phi\rangle=(|00\rangle+|11\rangle)/\sqrt{2}\] (o su 
operador densidad correspondiente $\rho=|\Phi\rangle\langle\Phi|$), 
y la mezcla estad\'{\i}stica \[\rho'=\half(|00\rangle\langle 00|+|11\rangle\langle 11|).\] 
En que se diferencian $\rho$ de $\rho'$? 
Hallar un obserable $O=O_A\otimes O_B$ ($O^\dagger=O$) que logre distinguir $\rho$ de $\rho'$ 
y otro que no logre distinguirlos.  
\hfill\break
%11) Consideremos el operador densidad general para un sistema %bipartito en un estado puro,  
%\[\rho=\sum_{i,j,i',j'}C_{ij}C^*_{i'j'}|ij\rangle\langle i'j'|\]
%donde $|ij\rangle=|i_A\rangle|j_B\rangle$ es una base producto %ortonormal arbitraria 
%del sistema $A+B$. \hfill\break
%a) Mostrar que  
%\[\rho=\sum_{j,j'}C_{jj'}|\tilde{j}j\rangle\langle \tilde{j}'j'|\]
%dando una expresi\'on para $|\tilde{j}_A\rangle$ y $C_{jj'}$.
%\hfill\break
%b) Hallar una expresi\'on para la matriz reducida $\rho_A$ en %t\'erminos de los estados 
%y coeficientes anteriores. \hfill\break
%c) Mostrar que el operador  
%\[\rho_d=\sum_{j}C_{jj}|\tilde{j}j\rangle\langle \tilde{j}j|\]
%conduce a la {\it misma} matriz reducida $\rho_A$. \hfill\break
%d) Concluir que si se efect\'ua una medici\'on proyectiva %arbitraria de $B$, 
%pero no se conoce el resultado, la misma no influye 
%el valor medio de una medici\'on de $A$, cualquiera sea el estado %conjunto $\rho$. \hfill\break 
9) En el proceso de teleportaci\'on, hallar la matriz reducida de B (Bob) en las siguientes 
etapas:\hfill\break
a) Antes de que Alice realice la medida \hfill\break
b) Luego de que Alice realice la medida y se la transmita a B \hfill\break
c) Luego de que Alice realice la medida pero sin concer el resultado de esta.\hfill\break
%13) a) Probar, hallando la descomposici\'on de Schmidt %correspondiente, que 
%los estados b) y c) del probl.\ 2 no pueden ser escritos como %producto 
%$|\Phi\rangle=|\phi_A\rangle|\phi_B\rangle$ en ninguna base, siendo %por lo tanto 
%entrelazados.  \hfill\break
%b) Determinar, hallando la descomposici\'on de Schmidt %correspondiente,  si los siguientes estados 
%son entrelazados: \hfill\break
%b1) $|\Phi\rangle=(|00\rangle+|01\rangle)/\sqrt{2}$. \hfill\break
%b2) %$|\Phi\rangle=(|00\rangle+|01\rangle+|10\rangle+|11\rangle)/2$.\hfill%\break
%b3) %$|\Phi\rangle=(|00\rangle+|01\rangle-|10\rangle+|11\rangle)/2$.\hfil%l\break
%c) Determinar el entrelazamiento $E(|\Phi\rangle)$. \\
10) Mediante la descomposici\'on de Schmidt de un estado puro $|\Phi\rangle$ de un 
sistema bipartito, probar que  
a) Los autovalores no nulos de $\rho_A$ y $\rho_B$ son id\'enticos.\hfill\break
b) $|\Phi\rangle=|\phi_A\rangle|\psi_B\rangle$ sii $\rho_A^2=\rho_A$, es decir, sii $\rho_A$ corresponde a un estado puro. 
\hfill\break
c) Se cumplen las condiciones a) y b) para estados no puros? En caso negativo dar un 
contraejemplo.\hfill\break
11) 
\end{document}
1) Escribir las matrices que representan, en la 
base computacional $(|00\rangle, |01\rangle, |10\rangle, |11\rangle)$), los operadores 
\[a)\;H\otimes I,\;\;\;b)\;I\otimes H,\;\;\;\;c)\;H\otimes H,\;\;\;d)\;
U_X=|0\rangle\langle 0|\otimes I+|1\rangle\langle 1|\otimes X,\;\;
d)\;W\equiv U_{X}(H\otimes I)\]
Aqu\'{\i} $H$ denota el gate de Hadamard, $H=(X+Z)/\sqrt{2}$, y $U_{X}=U_{CN}$ 
el gate Control Not (por que?), definido por $U_{X}|a,b\rangle
=|a,a\oplus b\rangle$ en la base computacional. 
Verificar que son matrices unitarias. Representar los operadores anteriores 
gr\'aficamente, mediante un esquema de circuito. \hfill\break\break
2) Comprobar, a partir de la forma de la matriz, que  $W=U_{X}(H\otimes I)$ transforma 
la base computacional en la base de Bell. 
% y que 
%$W^\dagger=(H\otimes I_B)U_{CN}$ transforma 
%la base de Bell en la base computacional. 
Concluir que una medici\'on 
en la base de Bell (es decir, basada en proyectores de estos estados) 
es equivalente a aplicar $W^\dagger=(H\otimes I)U_X$ seguido de una medici\'on  en la base computacional.
Representar mediante un circuito la anterior equivalencia.  
\hfill\break\break 
%Como queda el circuito de teleportaci\'on b\'asico si se mide el qubit de Alice y el qubit 
%a ser teleportado directamente en la base de Bell? \hfill\break
3) {\it Superdense Coding} elemental:  Mostrar que Alicia puede enviar {\it dos} 
bits de informaci\'on cl\'asica a Bob envi\'andole {\it un s\'olo qubit}, 
siempre y cuando Bob este en posesi\'on de un qubit entrelazado con el de Alicia. 
(Sug.\ : Probar con la transformaci\'on  $Z^jX^i$ aplicada al bit de Alicia). 
Dibujar el circuito correspondiente. \hfill\break\break
%suponiendo que Bob s\'olo dispone de aparatos de medida en la base computacional y de 
%gates $U_{CN}$ y $H$. \hfill\break
4) a) Escribir expl\'{\i}citamente el operador de rotaci\'on de un qubit alrededor 
de un eje $\vec{n}$, $R_{\vec{n}}(\theta)=\exp[-I\theta\vec{n}\cdot\vec{\sigma}]$, 
y mostrar que una transformaci\'on unitaria arbitraria 
de un qubit puede escribirse  como $U=e^{i\alpha}R_{\vec{n}}(\theta)$. 
\hfill\break
b) Verificar que $X=iR_x(\pi)$, $Y=iR_y(\pi)$, $Z=iR_z(\pi)$, $H=iR_{\vec{n}}(\pi)$, 
con $\vec{n}=(1,0,1)/\sqrt{2}$. y que por lo tanto 
$XY=iZ$, $ZX=iY$,$YZ=iX$, $XZX=-Z$, $XYX=-Y$, $HXH=Z$, $HZH=X$.  
\hfill\break
c) Determinar los tiempos $t$ y el Hamiltoniano (de dos qubits) tales que el correspondiente 
operador evoluci\'on (tomando $\hbar=1$)  $U(t)=\exp[-iHt]$ coincida 
 con a) $R_{\vec{n}}(\theta)\otimes I$, b) $R_{\vec{n}}(\theta)\otimes R_{\vec{n'}}(\theta')$. 
\hfill\break\break
5) a) A partir de los resultados del p.\ 4, mostrar que $U_{X}=(I\otimes H)U_{Z}(I\otimes H)$, 
(recordar y hacer el gr\'afico del circuito correspondiente), 
donde $U_Z=|0\rangle\langle 0|\otimes I+|1\rangle\langle 1|\otimes Z$ es el gate Control $Z$ 
($U_Z|ab\rangle=(-1)^{ab}|ab\rangle$ en la base computacional). 
\hfill\break
b) Mostrar (o darse cuenta) que $U_Z=\bar{U}_Z$. donde $\bar{U}_Z=I\otimes |0\rangle\langle 0|+
Z\otimes |1\rangle\langle 1|$ denota el gate 
control $Z$ pero con el qubit $B$ como controlador. A partir de este resultado 
verificar que $(H\otimes H)U_{X} (H\otimes H)=\bar{U}_X$, con 
$\bar{U}_X|ab\rangle=|a\oplus b,b\rangle$ (en la base computacional). 
Escribir la matriz que representa a $\bar{U}_X$. \hfill\break\break
6) Mostrar que $U_S=U_{X}\bar{U}_XU_{X}$ es el operador de ``swap'', definido 
por $U_S|ab\rangle=|ba\rangle$ en la base computacional. Escribir la matriz que 
representa a $U_S$ en esta base.\hfill\break\break
7) Escribir expl\'{\i}citamente el gate $U_{R_X(\theta)}$ (control $R_x(\theta)$), definido por 
$U_{R_x(\theta)}|ab\rangle=I_A\otimes R^a_x(\theta)|ab\rangle$ en la base computacional, 
en t\'erminos de operadores $U_X$ (control Not) y de un s\'olo qubit. 
\hfill\break\break 
8) Descomponer la matriz unitaria 
$U={\protect\left(\begin{array}{cccc}1&0&0&0\\0&1/\sqrt{2}&1/2&1/2\\0&-1/\sqrt{2}&1/2&1/2
\\0&0&-1/\sqrt{2}&1/\sqrt{2}\end{array}\right)}$ 
como producto de matrices unitarias compuestas de a lo sumo un bloque de $2\times 2$  
y bloques de $1\times 1$. Realizar un circuito basado en operadores $U_X$ y de un s\'olo qubit 
que ejecute la operaci\'on anterior. \hfill\break\break
9) {\it Medidas}:
Consideremos un sistema de dos componentes $A+B$, el cual se encuentra 
 en un estado puro $|\Phi\rangle$. Si se mide 
 en $A$ el valor $j$ correspondiente a un observable $O_A=\sum_j|j_A\rangle\langle j_A|$,
\hfill\break
 a) Escribir el estado conjunto resultante $|\Phi_{j}\rangle$ luego 
de la medici\'on. \hfill\break 
b) Comprobar que la matriz densidad reducida resultante $\rho_B^j$ 
del sistema $B$ corresponde siempre a un estado puro. \hfill\break
c) Si el sistema se encuentra inicialmente en un estado descripto por una 
 matriz densidad general $\rho$,  cual es la matriz densidad 
del sistema luego de la medici\'on del valor $j$ en $A$? \hfill\break
d) Cual ser\'a la matriz reducida $\rho^j_B$ luego de la medici\'on? 
Sigue correspondiendo a un estado puro?    
\newpage
\end{document}
\hfill\break\break
Transformada de Fourier Finita (Quantum Fourier Transform): \hfill\break
1) Sea $\{|j\rangle,\;j=0,\ldots,n-1\}$ una base ortonormal de un espacio de dimensi\'on $n$. 
Consideremos los estados 
\[|\tilde{j}\rangle=U|j\rangle=\sum_{k=0}^{n-1}U_{kj}|k\rangle,\;\;\;
U_{kj}\equiv\langle k|U|j\rangle=\frac{1}{\sqrt{n}}e^{2\pi ikj/n}|k\rangle\]
a) Probar que $U$ es un operador unitario, y que por lo tanto 
\[\langle \tilde{j}|\tilde{j'}\rangle=\delta_{jj'}\] 
%(lo que equivale a probar que $\frac{1}{N}\sum_{k=0}^{n-1}e^{2\pi i k(j-j')/n}=\delta_{jj'}$ $\forall$
% $j,j'$ enteros). \hfill\break
%b) Mostrar que la t.\ inversa es 
%\[|k\rangle=\frac{1}{\sqrt{n}}\sum_{j=0}^{n-1}e^{-2\pi ijk}|\tilde{j}\rangle,\;\;j=1,\ldots,n-1\]
% |(o sea, $c_k=\langle k|\phi\rangle$) $\rightarrow$ 
%\[|\phi\rangle=\sum_{j}\tilde{f}_j|\tilde{j}\rangle,\,\tilde{f}_j=\sum_{k=0}^{n-1}e^{-2\pi ijk/n}f_k\]
b) Mostrar que $|\tilde{j}\rangle$ son autoestados del operador de 
traslaci\'on $T$, definido por $T|k\rangle=|k+1\rangle$  
(con la convenci\'on $|n\rangle\equiv|0\rangle$), con 
\[T|\tilde{j}\rangle=e^{-2\pi i j/n}}|\tilde{j}\rangle\]
c) Sea $X\equiv\sum_j j|j\rangle\langle j|$. Mostrar que  
\[P\equiv UXU^\dagger=\sum_{j=0}^{N-1}j|\tilde{j}\rangle\langle\tilde{j}|\]
y que por lo tanto $T=e^{-2\pi i P/N}$. \hfill\break
d) Mostrar que si $|\phi\rangle=\sum_jf_j|j\rangle$ $\Rightarrow$ 
$U|\phi\rangle=\sum_j \tilde{f}_j|j\rangle$, con $\tilde{f}_j=\sum_{k=0}^{n-1}U_{kj}f_j$. 
\hfill\break
Como ejemplo, probar que si  $|\phi\rangle=\sum_{j=0}^7 \cos(j\pi/8)|j\rangle$, 
$\Rightarrow $U|\phi\rangle=(|1\rangle+|7\rangle)/\sqrt{2}$.  
(\hfill\break
$\Rightarrow$ $U|\phi\rangle=(|1\rangle+|7\rangle)/\sqrt{2}$. 
e) Mostrar que $|\phi\rangle=\sum_jf_j|j\rangle$ puede tambi\'en escribirse como 
$|\phi\rangle=\sum_jg_j|\tilde{j}\rangle$, con $\tilde{g}_j=\sum_{k=0}^{n-1}U^\dagger_{jk}f_{k}$. 
\hfill\break
f) Si $f_k$ es una funci\'on peri\'odica de $k$ de per\'{\i}odo $m$ ($f_{k+m}=f_k$, con la convenci\'on 
$f_{n-1+j}\equiv f_{j}$), donde $ml=n$ y $m\leq n$, $l\leq n$, naturales, probar que $\tilde{f}_j$ ser\'a no nulo 
s\'olo si $j$ es un m\'ultiplo de $l$ (con $0\leq j\leq n-1$).  \hfill\break\break
\end{documemt}	
%e{j}\rangle$ son los autoestados de $T$. Definiendo el operador 
%$\tilde{O}=\sum_j|j\rangle j\langle$ tenemos entonces $T=e^{-2\pi i \tilde{O}/n}$. \hfill\break
%e)  Sea $\tilde{T}$ el operador definido por $\tilde{T}|\tilde{j}\rangle=|\tilde{j+1}\rangle$. 
%Probar que 
%\[\tilde{T}|k\rangle=e^{2\pi i k/n}|k\rangle\]
%de modo que $\tilde{T}=e^{2\pi i O/n}$, con $O=\sum_k|k\rangle k\langle k|$. 
%\hfill\break

\hfill\break 
probar

a) Muy f\'aciles: \hfill\break
1) Probar que todo operador positivo $A$ ($\langle \phi|A|\phi\rangle\geq 0$ (y por ende real)  
$\forall|\phi\rangle$) es necesariamente herm\'{\i}tico. 
(Sug.: $A=A_r+iA_i$, con $A_r=(A+A^\dagger)/2$, $A_i=(A-A^\dagger)/(2i)$). 
\hfill\break
2) Escribir expl\'{\i}citamente la matriz que representa a 
$\rho=|\Phi\rangle\langle\Phi|$ en la base computacional  
$(|00\rangle,|01\rangle,|10\rangle,|11\rangle)\;{\rm para}$ 	
a) $|\Phi\rangle=|00\rangle$ \hfill\break
 b) $|\Phi\rangle=(|00\rangle\pm|11\rangle)/\sqrt{2}$\hfill\break
c) $|\Phi\rangle=(|01\rangle\pm|10\rangle)/\sqrt{2}$ \hfill\break
Verificar que en todos los casos, $\rho^2=\rho$ y que los autovalores 
de $\rho$ son $(1,0,0,0)$. \hfill\break
3) Hallar la matriz densidad reducida $\rho_A={\rm Tr}_B\rho$ en todos los 
casos anteriores.\hfill\break
4) Mostrar que 
\[\rho=p|0\rangle\langle 0|+q|1\rangle\langle 1|\]
con $p\geq 0$, $q\geq 0$, $p+q=1$,  puede tambi\'en escribirse como 
\[\rho=\half(|\phi_+\rangle\langle\phi_+|+|\phi_-\rangle\langle\phi_-|)\]
con $|\phi_{\pm}\rangle=\sqrt{p}|0\rangle\pm \sqrt{q}|1\rangle$. 
Son $|\phi_{\pm}\rangle$ ortogonales? 
Encontrar tambi\'en la matriz unitaria que vincula a 
$|\tilde{0}\rangle=\sqrt{p}|0\rangle,|\tilde{1}\rangle=\sqrt{q}|1\rangle$  con 
$|\tilde{\pm}\rangle=\frac{1}{\sqrt{2}}|\pm\rangle$. 
Como se interpretan f\'{\i}sicamente ambas representaciones?
\hfill\break
5) Probar que si $\rho_i$, $i=1,\ldots,n$ son matrices densidad para un 
sistema dado $\Rightarrow$ la combinaci\'on convexa 
\[\rho=\sum_{i=1}^n p_i\rho_i,\;\;\sum_ip_i=1,\;\;p_i\geq 0\]
es tambi\'en una matriz densidad.\hfill\break
Si alg\'un $p_i$ es negativo pero se sigue cumpliendo que $\sum_ip_i=1$, 
puede seguir siendo $\rho$ una matriz densidad? \hfill\break
6) Demostrar que $\rho$ describe un estado puro sii $\rho^2=\rho$.\hfill\break	
7) Probar que si en un sistema bipartito $\rho=\rho_A\otimes\rho_B$ 
(matriz densidad no correlacionada) $\Rightarrow$ ${\rm Tr}_B\rho=\rho_A$, 
${\rm Tr}_A\rho=\rho_B$.  \hfill\break
b) F\'aciles:\hfill\break
7) Hallar los operadores densidad reducidos 
$\rho_A$ y $\rho_B$ para 
\[|\Phi\rangle=\sqrt{p}|ab\rangle\pm\sqrt{q}|\bar{a}\bar{b}\rangle\]
con $p\geq 0$, $q\geq 0$, $p+q=1$, y $\langle a|\bar{a}\rangle=0$, 
$\langle b|\bar{b}\rangle=0$, siendo $|a\rangle$, $|b\rangle$ 
estados normalizados arbitrarios de un qubit. En que caso 
es $\rho_A=I_A/2$, $\rho_B=I_B/2$?\hfill\break
8) a) Mostrar que la matriz densidad m\'as general 
para un qubit puede escribirse como 
\[\rho=\half(I+\vec{r}\cdot\vec{\sigma})\]
donde $\vec{r}=(r_x,r_y,r_z)$ es un vector arbitrario con $|\vec{r}|\leq 1$, 
y $\vec{\sigma}=(X,Y,Z)\equiv(\sigma_x,\sigma_y,\sigma_z)$ las matrices de Pauli.  \hfill\break
b) Probar que $\rho$ describe un estado puro sii $|\vec{r}|=1$. \hfill\break
9) Consideremos el estado de Bell \[|\Phi\rangle=(|00\rangle+|11\rangle)/\sqrt{2}\] (o su 
operador densidad correspondiente $\rho=|\Phi\rangle\langle\Phi|$), 
y la mezcla estad\'{\i}stica \[\rho'=\half(|00\rangle\langle 00|+|11\rangle\langle 11|).\] 
En que se diferencian $\rho$ de $\rho'$? 
Hallar un obserable $O=O_A\otimes O_B$ ($O^\dagger=O$) que logre distinguir $\rho$ de $\rho'$ 
y otro que no logre distinguirlos.  
\hfill\break
10) Consideremos el operador densidad general para un sistema bipartito en un estado puro,  
\[\rho=\sum_{i,j,i',j'}C_{ij}C^*_{i'j'}|ij\rangle\langle i'j'|\]
donde $|ij\rangle=|i_A\rangle|j_B\rangle$ es una base producto ortonormal arbitraria 
del sistema $A+B$. \hfill\break
a) Mostrar que  
\[\rho=\sum_{j,j'}C_{jj'}|\tilde{j}j\rangle\langle \tilde{j}'j'|\]
dando una expresi\'on para $|\tilde{j}_A\rangle$ y $C_{jj'}$.
\hfill\break
b) Hallar una expresi\'on para la matriz reducida $\rho_A$ en t\'erminos de los estados 
y coeficientes anteriores. \hfill\break
c) Mostrar que el operador  
\[\rho_d=\sum_{j}C_{jj}|\tilde{j}j\rangle\langle \tilde{j}j|\]
conduce a la {\it misma} matriz reducida $\rho_A$. \hfill\break
d) Concluir que si se efect\'ua una medici\'on proyectiva arbitraria de $B$, 
pero no se conoce el resultado, la misma no influye 
el valor medio de una medici\'on de $A$, cualquiera sea el estado conjunto $\rho$. \hfill\break 
9) En el proceso de teleportaci\'on, hallar la matriz reducida de B (Bob) en las siguientes 
etapas:\hfill\break
a) Antes de que Alice realice la medida \hfill\break
b) Luego de que Alice realice la medida y se la transmita a B \hfill\break
c) Luego de que Alice realice la medida pero sin concer el resultado de esta.\hfill\break
10) a) Probar, hallando la descomposici\'on de Schmidt correspondiente, que 
los estados b) y c) del probl.\ 2 no pueden ser escritos como producto 
$|\Phi\rangle=|\phi_A\rangle|\phi_B\rangle$ en ninguna base, siendo por lo tanto 
entrelazados.  \hfill\break
b) Determinar, hallando la descomposici\'on de Schmidt correspondiente,  si los siguientes estados 
son entrelazados: \hfill\break
a) $|\Phi\rangle=(|00\rangle+|01\rangle)/\sqrt{2}$. \hfill\break
b) $|\Phi\rangle=(|00\rangle+|01\rangle+|10\rangle+|11\rangle)/2$.\hfill\break
11) Mediante la descomposici\'on de Schmidt de un estado puro $|\Phi\rangle$ de un 
sistema bipartito, probar que  
a) Los autovalores no nulos de $\rho_A$ y $\rho_B$ son id\'enticos.\hfill\break
b) El estado $|\Phi\rangle$ puede escribirse como estado producto 
sii $\rho_A^2=\rho_A$, es decir, si $\rho_A$ corresponde a un estado puro. 
\hfill\break
c) Se cumplen las condiciones a) y b) para estados no puros? En caso negativo dar un 
contraejemplo.\hfill\break
12) Determinar todos los valores posibles de $x$ para los cuales 
\[\rho=x|\Phi\rangle\langle\Phi|+(1-x)I/4\]
con $|\Phi\rangle=(|00\rangle+|11\rangle)/\sqrt{2}$ y $I=I_A\otimes I_B$, 
es un operador densidad. \hfill\break
c) Para pensar:\hfill\break
13) Mostrar que no es siempre posible efectuar una descomposici\'on tipo Schmidt de  
estados puros de sistemas tripartitos. \hfill\break
14) a) Mostrar que si el estado que comparten Alice y Bob es cualquiera de los estados de Bell 
(y no necesariamente $(|00\rangle+|11\rangle)/\sqrt{2}$) 
la teleportaci\'on puede igual realizarse perfectamente.\hfill\break
b) Mostrar que si el estado que comparten Alice y Bob es un estado puro separable, la teleportaci\'on no 
puede realizarse.\hfill\break
c) Mostrar que si el estado que comparten es de la forma $\sqrt{p}|00\rangle+\sqrt{q}|11\rangle$, 
con $p+q=1$, $p\geq 0$, $q\geq 0$, la teleportaci\'on en la forma descripta 
puede realizarse exactamente s\'olo si $p=q=1/2$. \hfill\break
15) a) Demostrar que si $|\Phi\rangle=(|01\rangle-|10\rangle)/\sqrt{2}$ y 
$\vec{a}$, $\vec{b}$, etc., son vectores unitarios 
$\Rightarrow$ 
\[\langle \Phi|(\vec{a}\cdot\vec{\sigma}_A)\otimes(\vec{b}\cdot\vec{\sigma}_B)|\Phi\rangle=-\vec{a}\cdot\vec{b}\] 
\hfill\break 
b) Probar, definiendo $S_{\vec{a}}\equiv\vec{a}\cdot\vec{\sigma}_A$ y 
$O=S^A_{\vec{a}}\otimes S^B_{\vec{b}}+S^A_{\vec{a}}\otimes S^B_{\vec{b'}}+
S^A_{\vec{a'}}\otimes S^B_{\vec{b}}-S^A_{\vec{a'}}\otimes S^B_{\vec{b'}}$ 
que  
\[|\langle\Phi|O|\Phi\rangle|\leq 2\sqrt{2}\] 
con el valor m\'aximo alcanzado si $\vec{a}\cdot\vec{a}'=0=\vec{b}\cdot\vec{b'}$ 
y el par $(\vec{b},\vec{b}')$ est\'a desfasado $-\pi/4$ o $3\pi/4$ de $(\vec{a},\vec{a'})$  
(recordar dibujo).\hfill\break
c) Probar que si $\vec{a}$ y $\vec{a'}$, y  $\vec{b}$ y $\vec{b'}$ son ortogonales, y  
$S^A_{\vec{a}}S^B_{\vec{b}}|\nu\nu'\rangle_{\vec{a}\vec{b}}=\nu\nu'|\nu\nu'\rangle_{\vec{a}\vec{b}}$ 
(o sea, autoestados del esp\'{\i}n en la direcci\'on de $\vec{a}$ y $\vec{b}$, con $\nu,\nu'=\pm 1$) 
$\Rightarrow$ $|_{\vec{a}\vec{b}}\langle \nu\nu'|O|\nu\nu'\rangle_{\vec{a}\vec{b}}|=1$. 
Por lo tanto, para cualquier combinaci\'on convexa $\rho$ de este tipo de estados tendremos 
$|{\rm Tr}\rho O|\leq 1$. \hfill\break
d) Probar que en un modelo cl\'asico en el que se supone que el sistema proviene con valores 
ya prefijados del esp\'{\i}n 
en todas las direcciones $\vec{a},\vec{a'},\vec{b},\vec{b'}$ (lo que sabemos no es posible en MQ)
 con alguna distribuci\'on de probabilidad, se tiene $|\langle O\rangle|\leq 2$. 
\end{document}


\nu'''_{\vec{b'}}\rangle$'' con valores simult\'aneows 
\sum_{i_a,i_b,i_{a'},i_{b'}}p
|{\rm Tr}\rho O|\leq 2$. 
\[\rho=\sum_{i,j=0,1}|i_ajb\rangle\langle i_aj_b\rangle
\end{document}



 si $A$ no conoce el resultado de la medici\'on efectuada en $B$.     
5) Consideremos como medida de entrelazamiento la cantidad 
\[E(\rho)=S(\rho_A)=S(\rho_B)\]
con $S(\rho)\equiv=-{\rm Tr}\rho{\rm log}_2\rho$. 
Mostrar que   


	
\title{ANALISIS MATEMATICO I\\ (Inform\'atica)\\ $ $ \\
Clases Te\'oricas \\ (Segundo Semestre 1999)} 
\author{R. Rossignoli\\  Universidad Nacional de La Plata\\ $ $ \\(Versi\'on preliminar)}
\date{$ $}
\maketitle

%\begin{abstract}
%$ $ 
%\end{abstract}
\newpage
\section{Funciones Trigonom\'etricas Inversas}
\subsection{Funci\'on Arco-seno}
$\Re$ 
La funci\'on arco-seno ({\it arcsen}) es la inversa de la funci\'on seno considerada 
en el intervalo $[-\half\pi,\half\pi]$,  donde es estrictamente creciente. 
  Obviamente, no podemos definir la inversa tomando $D=\Re$ pues 
 $\sen(x+2\pi)=\sen(x)=\sen(\pi-x)$. Es necesario 
restringir el dominio a un intervalo donde sen$(x)$ sea estrictamente 
creciente o decreciente. Tenemos 
%Puede elegirse entonces un intervalo de la forma  
% $[-\pid+n\pi,\pid+n\pi]$ con $n$ entero arbitrario. 
%Por convenci\'on, se elige el intervalo $[-\pid,\pid]$, 
%donde es estrictamente creciente,  
\[y=\sen(x),\;\;\; x\in[-\pid,\pid]\;\;\;\Rightarrow x=\arcsen(y),\;\;\;y\in[-1,1].\]
El dominio de la funci\'on  arco-seno es $[-1,1]$ y sus valores est\'an comprendidos en el 
intervalo $[-\pid,\pid]$. Como $\sin(-x)=-\sin(x)$, 
$\arcsen(-y)=-\arcsen(y)$. Ejemplos: 
\[\arcsen(0)=0,\;\;\arcsen(\pm\half)=\pm\pis,\;\;\arcsen(\pm{\textstyle\frac{\sqrt{3}}{2}})=
\pm\pitt,\] 
\[\arcsen[\sen(\pid)]=\arcsen(1)=\pid,\;\;
 \arcsen[\sen(2\pi)]=\arcsen(0)=0.\]
Notemos que  $\arcsen(\sen(x))=x$   {\it s\'olo si} $x\in[-\pid,\pid]$. 
%Por ejemplo, $\arcsen(\sen(2\pi))=\arcsen(0)=0$. 
\hfill\break
Dado que $\sen(x)=\sen(\pi-x)=\sen(x+2n\pi)$, la soluci\'on general de la ecuaci\'on 
$\sen(x)=y$ es 
\[x=\arcsen(y)+2n\pi,\;\;\;x=\pi-\arcsen(y)+2n\pi,\]
con $n$ entero arbitrario. El \'angulo (en radianes) dado por $\arcsen(y)$ pertenece al primer 
 o cuarto cuadrante, mientras que $\pi-\arcsen(y)$ corresponde a un \'angulo del segundo o 
tercer cuadrante, con el mismo valor del seno. 

%Fuera de este intervalo $\arcsen(\sen(x))\neq x$. 

No es posible expresar la funci\'on arco-seno en t\'erminos de las funciones elementales 
que hemos utilizado hasta ahora. Si es posible, sin embargo, calcular expl\'{\i}citamente su derivada. 
 Dado que $d\,\sen(x)/dx=\cos(x)$ se anula en $x=\pm\pid$, 
la derivada de la funci\'on inversa existir\'a s\'olo en el intervalo abierto 
$(\sen(-\pid),\sen(\pid))=(-1,1)$. Obtenemos 
%Utilizando la f\'ormula de la derivada de la funci\'on inversa, obtenemos  
\ben 
\frac{d\,\arcsen(y)}{dy}&=&\frac{dx}{dy}=\frac{1}{\frac{dy}{dx}}=\frac{1}{\cos(x)}=
\frac{1}{\sqrt{1-\sen^2(x)}}\nonumber\\&=&
\frac{1}{\sqrt{1-y^2}},\;\;\;y\in(-1,1)\label{arcsend}
\een
donde hemos utilizado la relaci\'on $\cos(x)=\sqrt{1-\sen^2(x)}$ v\'alida para 
 $x\in [-\pid,\pid]$, 
obtenida a partir de $\sen^2(x)+\cos^2(x)=1$ y la desigualdad $\cos(x)\geq 0$ si 
$x\in [-\pid,\pid]$.  
%Vemos que a pesar de que es imposible expresar el arco-seno en t\'erminos de las 
%funciones elementales vistas hasta ahora, su derivada es en cambio una funci\'on sencilla. 
%%Como $\sqrt{1-y^2}>0$ para $y\in(-1,1)$,  
El arco-seno es  estrictamente creciente para $y\in[-1,1]$. 
 Notemos   que la derivada tiende a infinito en los extremos: 
\[\lim_{y\rightarrow \pm 1^{\mp}}\frac{1}{\sqrt{1-y^2}}=\infty.\] 
\vspace*{7.5cm}

\epsfxsize=6cm  
%%%\epsffile{bcsepar.ps}
\epsffile{plotsen.ps}
\vspace*{-11.cm}

\epsfxsize=6cm 
\hspace*{7cm}\epsffile{sen1.eps}
\vspace*{-15cm}

\newpage
\subsection{Funci\'on Arco-coseno}
La funci\'on arco-coseno ({\it arccos}) es la inversa de la funci\'on coseno considerada 
en el intervalo $[0,\pi]$, donde es estrictamente decreciente.  
Obviamente, dado que $\cos(x+2\pi)=\cos(x)=\cos(-x)$, 
 podemos definir la inversa s\'olo si restringimos el 
dominio a un intervalo donde $\cos(x)$ sea estrictamente creciente o decreciente, 
es decir,  de la forma $[n\pi,(n+1)\pi]$, con $n$ entero arbitrario. Obtenemos 
\[y=\cos(x),\;\;\; x\in[0,\pi],\;\;\;\Rightarrow x=\arccos(y),\;\;\;y\in[-1,1].\]
El dominio de la funci\'on  arccos es nuevamente $[-1,1]$ y sus valores est\'an comprendidos en el intervalo 
$[0,\pi]$. 
Ejemplos: 
\[\arccos(0)=\pid,\;\;\arccos(\pm\half)=\,^{\pi/3}_{2\pi/3},
\;\;\arccos(\pm{\textstyle\frac{\sqrt{3}}{2}})=\,^{\pi/6}_{5\pi/6},\] 
\[\arccos[\cos(\pid)]=\arccos(0)=\pid,\;\;
 \arccos[\cos(2\pi)]=\arccos(1)=0.\]
Notemos que  $\arccos(\cos(x))=x$ s\'olo si $x\in[0,\pi]$. 
%Por ejemplo, $\arccos(\cos(2\pi))=\arccos(1)=0$. 
\hfill\break 
Como $\cos(x)=\cos(-x)=\cos(x+2n\pi)$, la soluci\'on general de la ecuaci\'on $\cos(x)=y$ es 
\[x=\arccos(y)+2n\pi,\;\;\;\;x=-\arccos(y)+2n\pi\]	
con $n$ entero arbitrario. $\arccos(y)$ representa un \'angulo (en radianes) 
del primer o segundo cuadrante, mientras que $-\arccos(y)$ un \'angulo del tercero o cuarto. 

Dado que $y=\cos(x)=\sen(\pid-x)$, para $x\in[0,\pi]$ vale la relaci\'on 
 $\pid-x=\arcsen(y)$  y por lo tanto \hfill\break 
 $x=\pid-\arcsen(y)$. Entonces, 
\[\arccos(y)=\pid-\arcsen(y).\] 
El arco-coseno se expresa pues sencillamente en t\'erminos del arco-seno. 
Su derivada es, utilizando (\ref{arcsend}), 
\be\frac{d\arccos(y)}{dy}=-\frac{d\arcsen(y)}{dy}=-\frac{1}{\sqrt{1-y^2}},\;\;\;y\in(-1,1)\label{arccosd1},\ee
que puede calcularse tambi\'en por el m\'etodo de la 
funci\'on inversa.El arco-coseno es una funci\'on estrictamente decreciente. 

%La derivada puede  Dado que $d\cos(x)/dx=-\sen(x)$ se anula en $x=0$ y $x=\pi$, 
%la derivada de la funci\'on inversa existir\'a s\'olo en el intervalo abierto 
%$(\cos(\pi),\cos(0))=(-1,1)$. Tenemos 
%\ben \frac{d\arccos(y)}{dy}&=&\frac{dx}{dy}=\frac{1}{\frac{dy}{dx}}=
%-\frac{1}{\sen(x)}=
%-\frac{1}{\sqrt{1-\cos^2(x)}}\nonumber\\ &=&-\frac{1}{\sqrt{1-y^2}},\;\;\;
%y\in(-1,1)\een
%donde hemos utilizado la relaci\'on $\sen(x)=\sqrt{1-\cos^2(x)}$, 
% v\'alida para  $x\in [0,\pi]$. 
%El resultado coincide con (\ref{arccosd1}). 

\vspace*{9cm}
\epsfxsize=6cm  
%%%\epsffile{bcsepar.ps}
\epsffile{plotcos.ps}
\vspace*{-11.6cm}
%% 4294-4353 4236-4600 Claypole direcc: Cavour y 2 de Abril 
\epsfxsize=6cm 
\hspace*{7cm}\epsffile{cos1.eps}
\vspace*{-5cm}
\newpage
\subsection{Funci\'on Arco-tangente}
La funci\'on arco-tangente ({\it arctan}) es la inversa de la funci\'on tangente 
considerada en el intervalo $(-\pid,\pid)$, donde es estrictamente creciente. 
Dado que $\tan(x)=\tan(x+\pi)$ y dado que $\tan(x)$ no est\'a definida en $x=\pid+n\pi$, 
s\'olo podemos definir la inversa 
en un intervalo abierto de la forma $(-\pid+n\pi,\pid+n\pi)$.  Obtenemos 
%Por convenci\'on, se elige el intervalo $(-\pid,\pid)$, 
\[y=\tan(x),\;\;\; x\in(-\pid,\pid),\;\;\;\Rightarrow x=\arctan(y),\;\;\;y\in R.\]
El dominio de la funci\'on  arctan son todos los reales, dado que 
$\tan(x)$ no es acotada ni superior ni inferiormente para $x\in (-\pid,\pid)$, tomando todos 
los valores reales. Como $\tan(-x)=-\tan(x)$, $\arctan(-x)=-\arctan(x)$. 
Ejemplos: 
\[\arctan(0)=0,\;\;\arctan(\pm 1)=\pm\pic,\;\;\arctan[\pm\sqrt{3}]=\pm\pitt\] 
\[\arctan[\tan(\pis)]=\arctan({\textstyle\frac{1}{\sqrt{3}}})=\pis,\;\;
 \arctan[\tan(2\pi)]=\arctan(0)=0.\]
Nuevamente, $\arctan(\tan(y))=\tan(y)$ s\'olo si $y\in(-\pid,\pid)$. 
La soluci\'on general de la ecuaci\'on $\tan(x)=y$ es pues 
\[x=\arctan(y)+n\pi\]
con $n$ entero arbitrario. Notemos tambi\'en que, como 
$\lim\limits_{x\rightarrow\pm\frac{\pi}{2}}\tan(x)=\pm\infty$, 
\[\lim_{y\rightarrow\pm\infty}\arctan(y)=\pm\pid\]
%No es posible expresar la funci\'on arctan en t\'erminos de las funciones elementales 
%que hemos utilizado hasta ahora. 
Calculemos su derivada. Dado que $d\tan(x)/dx=1/\cos^2(x)=1+\tan^2(x)$,  
se tiene, para $x=\arctan(y)$, $y=\tan(x)$,  
\ben\frac{d\arctan(y)}{dy}&=&\frac{dx}{dy}=\frac{1}{\frac{dy}{dx}}=
\frac{1}{1+\tan^2(x)}\nonumber\\&=&\frac{1}{1+y^2}\label{arctand}\een
% Notemos que $\lim_{y\rightarrow \pm \infty}\frac{1}{1+y^2}=0$.  
%La derivada del arco-tangente es expresable en una funci\'on sencilla. 

El arco-tangente puede  expresarse tambi\'en en t\'erminos del arco-seno y vice-versa. 
Como\hfill\break $\tan^2(x)=\frac{{\rm sen}^2(x)}{\cos^2(x)}=\frac{{\rm sen}^2(x)}{1-{\rm sen}^2(x)}$, 
tenemos, si $y=\sen(x)$, $x\in(-\pid,\pid)$, 
\[x=\arcsen(y)=\arctan[\frac{y}{\sqrt{1-y^2}}],\;\;\;\;y\in(-1,1)\]
%${\rm sen}^2(x)=\frac{\tan^2(x)}{1+\tan^2(x)}$ y 
% por lo tanto, 
%para $y=\tan(x)$,  
Adem\'as, de la relaci\'on anterior obtenemos 
\[\sen(x)=\frac{\tan(x)}{\sqrt{1+\tan^2(x)}}\;\;\;x\in(-\pid,\pid)\] %\frac{y}{\sqrt{1+y^2}},
 (el signo de $\sen(x)$ es igual al de $\tan(x)$ para 
$x\in(-\pid,\pid)$). Por lo tanto, si $y=\tan(x)$. % $x=\arctan(y)$,  puede obtenerse tambi\'en como 
\be x=\arctan(y)=\arcsen(\frac{y}{\sqrt{1+y^2}}),\;\;\;\;y\in\Re\label{arctansen}\ee
Notemos que para $y\in\Re$, $-1<\frac{y}{\sqrt{1+y^2}}<1$, con 
 $\lim\limits_{y\rightarrow\pm\infty}\frac{y}{\sqrt{1+y^2}}=\pm 1$. 
Podemos reobtener el resultado (\ref{arctand}) a partir de (\ref{arcsend}) y (\ref{arctansen})  
(se deja la verificaci\'on para el lector). 

\vspace*{8cm}
\epsfxsize=6cm  
\epsffile{plottan.ps}
\vspace*{-12.cm}

\epsfxsize=6cm 
\hspace*{7cm}\epsffile{tan1.eps}
\vspace*{-5cm}
\newpage
\subsection{Resumen}
Considerando las definiciones  anteriores como funci\'on de una 
variable gen\'erica $x$, tenemos 
\[\frac{d\,\arcsen(x)}{dx}=\frac{1}{\sqrt{1-x^2}},\;\;x\in(-1,1)\]
\[\frac{d\,\arccos(x)}{dx}=\frac{-1}{\sqrt{1-x^2}},\;\;x\in(-1,1)\]
\[\frac{d\,\arctan(x)}{dx}=\frac{1}{1+x^2},\;\;x\in\Re\]
\[\arccos(x)=\pid-\arcsen(x),\;\;\;\;\;\;\;\;\arctan(x)=\arcsen(\frac{x}{\sqrt{1+x^2}})\]
%\[\frac{d\arctan(y)}{dy}=\frac{\arcsen(\frac{y}{\sqrt{1+y^2}})}{dy}=
%\frac{1}{\sqrt{1-\frac{y^2}{1+y^2}}}\frac{d}{dy}\frac{y}{\sqrt{1+y^2}}=
%\frac{1}{1+y^2},\;\;\;y\in\Re\]
%donde dejamos los detalles de las cuentas para el lector. 
Remarquemos, para evitar errores frecuentes,  que estas funciones   nada tienen que ver con 
las rec\'{\i}procas, es decir, $\arcsen(x)\neq 1/\sen(x)$,
$\arccos(x)\neq 1/\cos(x)$, $\arctan(x)\neq 1/\tan(x)$. 
\hfill\break\break
Ejemplos: \hfill\break\break 
1)\hfill\break
\[\frac{d\,\arcsen(x^2)}{dx}=\frac{2x}{\sqrt{1-x^4}},\;\;x\in(-1,1)\]
2)\hfill\break
\[y=\cos(x^2),\;\;x\in[0,\sqrt{\pi}],\;\;\Rightarrow x=\sqrt{\arccos(y)},\;y\in[-1,1]\]
\[\;\;\;\;\;\frac{dx}{dy}=\frac{1}{2\sqrt{\arccos(y)}}\frac{-1}{\sqrt{1-y^2}},\;\;y\in(-1,1)\]
\hfill\break
3) Hallar la velocidad de variaci\'on $d\phi/dt$ del \'angulo 
determinado por la posici\'on de un globo que asciende 
con velocidad $v(t)=dh/dt$ a una altura $h(t)$ y a distancia horizontal fija $l$. 
\hfill\break
Tenemos $\tan[\phi(t)]=h(t)/l$, y por lo tanto, 
\[\phi(t)=\arctan[h(t)/l]\]
\[\frac{d\phi}{dt}=\frac{1}{1+h^2(t)/l^2}\frac{1}{l}\frac{dh}{dt}
=\frac{lv(t)}{l^2+h^2(t)}\]

\vspace*{7cm}
\epsfxsize=6cm  
\centerline{\epsffile{triang.ps}}
\vspace*{-3.6cm}
%A pesar de que es imposible expresar $\arccos$ en t\'erminos de funciones elementales, su 
%derivada es en cambio una funci\'on sencilla. Notemos que $\lim_{y\rightarrow \pm 1}\frac{1}{\sqrt{1-y^2}}=\infty$.
\newpage
\section{Funciones exponenciales y logar\'{\i}tmicas}
\subsection{Introducci\'on}
Comenzaremos con el estudio de las muy importantes funciones exponencial y 
 logaritmo. Daremos en esta primera secci\'on una introducci\'on intuitiva,
 conveniente en un primer acercamiento al tema. 
Luego daremos el tratamiento m\'as riguroso convencional. 
\subsubsection{Funci\'on exponencial}
Recordemos donde estamos situados en relaci\'on con las funciones exponenciales. 
Tenemos la definici\'on b\'asica  
\[a^n=\overbrace{a\cdot a\cdot\ldots \cdot a}^{n\;{\rm veces}},\;\;\;n\;{\rm natural}
. \] 
Se satisface obviamente 
\[ a^{n+m}=a^na^m.\] 
Queremos que la propiedad anterior se cumpla $\forall n,m$ enteros. 
Por lo tanto, 
\[ a^n=a^{0+n}=a^0 a^n,\;\;\Rightarrow a^0=1\]
\[ 1=a^0=a^{-n+n}=a^{-n}a^{n},\;\;\Rightarrow a^{-n}=1/a^n\]
Para $a>0$, tenemos adem\'as 
\[ a=a^{n/n}=a^{\overbrace{\scriptstyle 1/n+\ldots+1/n}^{\scriptstyle n\;{\rm veces}}}=
\overbrace{a^{1/n}\ldots a^{1/n}}^{n\;{\rm veces}}=(a^{1/n})^n,\;\;\Rightarrow 
a^{1/n}=\sqrt[n]{a}\]
\[ a^{m/n}=a^{\overbrace{\scriptstyle 1/n+\ldots+1/n}^{\scriptstyle m\;{\rm veces}}}=
\overbrace{a^{1/n}\ldots a^{1/n}}^{m\;{\rm veces}}=(a^{1/n})^m=(\sqrt[n]{a})^m 
\]
lo que nos permite definir $a^x$ para cualquier $x$ racional si $a>0$. 

Para definir $a^x$ para cualquier $x$ real, nos hace falta una 
definici\'on consistente para $x$ irracional. Como ``consistente'' lo que pretendemos 
es que las propiedades anteriores sigan siendo v\'alidas para cualquier $x$ real,  
y por su puesto, que $a^x$ sea una funci\'on continua (y derivable) de $x$. 
Nos contentaremos en una primera 
etapa con una  extensi\'on intuitivamente obvia: Si  $\{x_1,x_2,\ldots,x_n,\dots\}$ 
es una sucesi\'on de n\'umeros racionales que converge al n\'umero irracional $x$, 
 definimos $a^x$ como el l\'{\i}mite de la sucesi\'on 
$\{a^{x_1},a^{x_2},\ldots,a^{x_n},\ldots\}$, asumiendo por su puesto que tal 
sucesi\'on converge $\forall$ $x$ si $a>0$. Por ej., 
para calcular $2^\pi$, podemos formar la sucesi\'on $\{2^{3,14},2^{3,141},2^{3,1415},2^{3,14159},\ldots\}$. 

Se cumplen las propiedades  
\be a^{x_1+x_2}=a^{x_1}a^{x_2}\label{ax1}\ee
\be(a^{x_1})^{x_2}=a^{x_1 x_2}\label{ax2}\ee
Para $a>0$ podemos entonces definir la funci\'on exponencial 
\[ f(x)=a^x,\;\;x\in \Re \]
Por ej., $f(0)=a^0=1$, $f(1)=a^1=a$, $f(2)=a^2$, $f(-1/2)=1/\sqrt{a}$. En general,  
 $f(x)>0$ $\forall x$.  Si $a>1$, $f(x)$ es estrictamente 
creciente, con $\lim_{x\rightarrow+\infty}f(x)=\infty$,  $\lim_{x\rightarrow-\infty}f(x)=0$ 
(pues $a^{-x}=1/a^{x}$). Si $0<a<1$, $f(x)$ es estrictamente decreciente y 
$\lim_{x\rightarrow+\infty}f(x)=0$,  $\lim_{x\rightarrow-\infty}f(x)=\infty$. 
Si $a=1$, $f(x)=1$ $\forall x$. 
\newpage
\subsubsection{Funci\'on logaritmo}
Para $a>0$, $a\neq 1$, la funci\'on $a^x$ es estrictamente creciente o decreciente, y posee 
por lo tanto inversa. la funci\'on inversa  se denomina  {\it logaritmo} en base $a$, 
\[ y=a^x,\;\;a>0,\;a\neq 1,\;\;x\in\Re,\;\;\;\Rightarrow\;\; x=\log_a(y),\;\; y>0\] 
El dominio de $\log_a(y)$ son los reales positivos. 
Ej., $\log_a(1)=0$, $\log_a(a)=1$, $\log_a(a^2)=2$, $\log_a(1/a)=-1$. 
Para $a>1$, como $\lim_{x\rightarrow+\infty}a^x=\infty$, $\lim_{x\rightarrow-\infty}a^x=0$, 
$\Rightarrow$ $\lim_{x\rightarrow+\infty}\log_a(x)=\infty$, $\lim_{x\rightarrow 0+}\log_a(x)=-\infty$. 

Repasemos ahora las propiedades b\'asicas del logaritmo. Sean  
\[ y_1=a^{x_1},\;\;\;y_2=a^{x_2},\;\;y_1y_2=a^{x1}a^{x2}=a^{x_1+x_2}\]
para $y_1>0$, $y_2>0$. Entonces, 
\[ x_1=\log_a(y_1),\;\;x_2=\log_a(y_2)\]
y por lo tanto, 
\[ \log_a(y_1y_2)=x_1+x_2=\log_a(y_1)+\log_a(y_2)\] 
Asimismo, como $y_1/y_2=a^{x_1-x_2}$, 
\[\log_a(y_1/y_2)=x_1-x_2=\log_a(y_1)-\log_a(y_2)\] 
Tambi\'en, como  
$y_1^x=(a^{x_1})^x=a^{xx_1}$, se tiene 
\[ \log_a(y_1^x)=xx_1=x\log_a(y_1)\]
Finalmente, para $b>0$, $b\neq 1$, si 
\[ y=b^x\;\;\Rightarrow\;\; x=\log_b(y)\]
Tomando logaritmos en base $a$ en ambos lados, 
\[ \log_a(y)=\log_a(b^x)=x\log_a(b)=\log_b(y)\log_a(y)\]
de donde 
\[ \log_b(y)=\log_a(y)/\log_a(b)\]
lo que muestra que los logaritmos en bases diferentes son directamente 
proporcionales. Basta conocer los logaritmos en una base para conocerlos en una 
base arbitraria. En particular, dado que $\log_a(a)=1$, 
\[\log_b(a)=1/\log_a(b)\]
Notemos tambi\'en que 
\[ \log_a(b^x)=x\log_a(b)\;\;\Rightarrow\;\; b^x=a^{x\log_a(b)}\]
de modo que podemos expresar una potencia cualquiera de base $b$ en t\'erminos de una potencia de 
base  $a$. 

\vspace*{9cm}
\epsfxsize=6cm  
%%%\epsffile{bcsepar.ps}
\epsffile{ax.ps}
\vspace*{-3.6cm}

\epsfxsize=6cm 
\hspace*{6cm}\epsffile{loga.ps}
\vspace*{-5cm}
\newpage
\subsubsection{Derivada de la funci\'on exponencial}
Tratemos ahora de derivar la funci\'on exponencial con nuestros 
conocimientos. Se trata obviamente de derivar $a^x$ {\it respecto de $x$}, 
donde $x$ est\'a en el exponente 
(lo que sabemos hasta ahora es derivar $x^a$ respecto de $x$, 
es decir, $dx^a/dx=ax^{a-1}$, que hemos demostrado s\'olo para $a$ racional).   
Tenemos 
\[\frac{d a^x}{dx}=\lim_{h\rightarrow 0}\frac{a^{x+h}-a^x}{h}=
\lim_{h\rightarrow 0}\frac{a^{x}a^{h}-a^x}{h}=
a^x\lim_{h\rightarrow 0}\frac{a^{h}-1}{h}\]
El problema se reduce al conocimiento del \'ultimo l{\'{\i}mite}, 
que depende de $a$ pero no de $x$, y que es de la forma  $0/0$ 
(pues $a^h\rightarrow 1$ cuando $h\rightarrow 0$). Llamando 
$u=a^h-1$,  tenemos $a^h=1+u$ y $h=\log_a(1+u)$, 
con  $u\rightarrow 0$ para $h\rightarrow 0$. 
Por tanto, 
\[\lim_{h\rightarrow 0}\frac{a^{h}-1}{h}=
\lim_{u\rightarrow 0}\frac{u}{\log_a(1+u)}=
\lim_{u\rightarrow 0}\frac{1}{\frac{1}{u}\log_a(1+u)}=
\lim_{u\rightarrow 0}\frac{1}{\log_a[(1+u)^{1/u}]}=
\frac{1}{\log_a(e)},\]
%\lim_{u\rightarrow 0}{\log_a(u+1)}=\log_a(u+1)}=
donde hemos definido 
\be e\equiv \lim_{u\rightarrow 0}
(1+u)^{1/u}\label{e}\ee 
Hemos pues reducido el problema al c\'alculo del l\'{\i}mite anterior, 
que no depende de $a$ ni de $x$. 

No resulta en realidad muy dif\'{\i}cil calcular $e$, asumiendo que el l\'{\i}mite existe. 
Llamando $s=1/u$, de modo que $u=1/s$, con $s\rightarrow\infty$ para $u\rightarrow 0$, 
podemos reescribir (\ref{e}) en la forma  
\[e=\lim_{s\rightarrow\infty}(1+1/s)^s.\]
Para $s=n$, con $n$ natural, podemos aplicar la f\'ormula del binomio de Newton, 
\[(a+b)^n=\sum_{k=0}^n(^n_k)a^{n-k}b^k,\;\;(^n_k)={\textstyle\frac{n!}{k!(n-k)!}}.\]
Tenemos 
\ben {(1+\frac{1}{n})^n}&=&{\sum_{k=0}^n(^n_k)1^{n-k}\frac{1}{n^k}
=1+\frac{n}{1!}\frac{1}{n}+\frac{n(n-1)}{2!}\frac{1}{n^2}+\frac{n(n-1)(n-2)}{3!}\frac{1}{n^3}+\ldots}
\nonumber\\
&=&{ 1+1+\frac{1}{2!}(1-\frac{1}{n})+\frac{1}{3!}(1-\frac{1}{n})(1-\frac{2}{n})+\ldots}\een
En el l\'{\i}mite $n\rightarrow \infty$,  podemos despreciar los t\'erminos proporcionales a 
$\frac{1}{n}$, y obtenemos 
\ben e&=&\lim_{n\rightarrow\infty}(1+{\frac{1}{n}})^n={1+1+
\frac{1}{2!}+\frac{1}{3!}+\ldots}
=\sum_{k=0}^{\infty}{\frac{1}{k!}}\nonumber\\&=&2,718281828459\ldots\label{en}\een
 donde la suma infinita  denota una serie que puede demostrarse convergente 
(como veremos mas adelante). El n\'umero $e$ resulta ser irracional, y fue introducido 
por el genial matem\'atico suizo Leonhard Euler (1707-1783).  
Con s\'olo considerar los primeros 15 t\'erminos 
de la suma,  obtenemos ya la estimaci\'on indicada, correcta en los primeros 12 decimales.  
%(con 8 t\'erminos ya tenemos 5 decimales correctos). 
Para dar una idea de la facilidad con que es posible manejar hoy con este tipo 
de n\'umeros,  mencionamos que la evaluaci\'on de $e$ con 10.000 decimales correctos demora 
apenas una fracci\'on de segundo en una PC est\'andar. 

En resumen, hemos mostrado que si la derivada de la exponencial existe, debe ser 
\be\frac{da^x}{dx}=a^x\frac{1}{\log_a(e)}=a^x \log_e(a)\label{axd}\ee
donde $\log_e$, el logaritmo en base $e$, es denominado {\it logaritmo natural} y denotado 
como $\ln$: $\log_e(a)=\ln(a)$. 

En particular, si $a=e$, tenemos 
\[ \frac{de^x}{dx}=e^x \log_e(e)=e^x\]
es decir, que la derivada de la funci\'on $e^x$ {\it es igual a si misma}. 
Por esta raz\'on, se utiliza en general la exponencial de base $e$, 
pudiendo expresarse la exponencial de base arbitraria $a$ como 
\[ a^x=e^{\log_e(a^x)}=e^{x\log_e(a)}\]
%El logaritmo en base $e$ se denomina logaritmo {\it natural}, 
%y se denota como $\ln$: $\log_e(x)=\ln(x)$.  
\newpage
\subsubsection{Derivada del logaritmo}\hfill\break
Calculemos ahora la derivada del logaritmo. 
Por el teorema de la derivada de la funci\'on inversa, tenemos, 
siendo $y=a^x$ y $x=\log_a(y)$, 
\be \frac{d\log_a(y)}{dy}=\frac{dx}{dy}=\frac{1}{\frac{dy}{dx}}=\frac{1}{a^x\ln(a)}=
\frac{1}{y}\frac{1}{\ln(a)}\label{logd}\ee
donde $\ln(a)=\log_e(a)$. 
La derivada de $\log_a(y)$ es pues proporcional a $1/y$. En particular, para $a=e$, 
\[\frac{d\ln(y)}{dy}=\frac{1}{y}\frac{1}{\ln(e)}=\frac{1}{y}\]
La derivada del logaritmo natural $\ln (y)$ es pues directamente 
igual a $1/y$. 

Podemos tambi\'en obtener el resultado anterior directamente: 
\ben\frac{d\log_a(y)}{dy}&=&\lim_{h\rightarrow 0}\frac{\log_a(y+h)-\log_a(y)}{h}=
\lim_{h\rightarrow 0}\frac{\log_a\frac{y+h}{y}}{h}
=\lim_{h\rightarrow 0}\log_a(1+\frac{h}{y})^{1/h}=
\log_a\lim_{h\rightarrow 0}(1+\frac{h}{y})^{1/h}\nonumber\een
donde hemos supuesto que el \'ultimo l\'{\i}mite existe. 
Llamando $u=h/y$, obtenemos (asumiendo siempre $y>0$) $h=uy$, y 
\[ \lim_{h\rightarrow 0}(1+\frac{h}{y})^{1/h}=\lim_{u\rightarrow 0}(1+u)^{1/(uy)}=
[\lim_{u\rightarrow 0}(1+u)^{1/u}]^{1/y}=e^{1/y}\] 
y por lo tanto, 
\[{\log_a\lim_{h\rightarrow 0}(1+\frac{h}{y})^{1/h}=\log_a e^{1/y}=\frac{1}{y}\log_a(e)=
\frac{1}{y}\frac{1}{\ln(a)}}\] 
lo que conduce al mismo resultado (\ref{logd}). 

\newpage
\subsection{Definici\'on geom\'etrica del logaritmo}
Consideraremos ahora el tratamiento geom\'etrico convencional, 
m\'as riguroso, del logaritmo y la exponencial, en el que se comienza 
por el final. 
 Como veremos mas adelante, el \'area comprendida entre una curva $y=f(x)$ y 
el eje horizontal, entre dos puntos $x_0$ y $x$, 
es una funci\' on $A(x)$ cuya derivada es $f(x)$.
As\'{\i}, para evitar ciertos puntos oscuros en el desarrollo de la secci\'on anterior 
(tales como la definici\'on de exponencial para exponentes irracionales y 
la existencia del l\'{\i}mite (\ref{e})), \hfill\break 
{\it se define el logaritmo natural $\ln(x)$ como el \'area bajo la curva $1/x$ entre $1$ y $x$, 
si $x>1$, y como el negativo del \'area si $0<x<1$}. Adem\'as, $\ln(1)=0$. 
Es f\'acil probar que el logaritmo as\'{\i} definido posee todas las 
propiedades esperadas de un logaritmo, y que la derivada de $\ln(x)$ existe y es $1/x$: 
\be\frac{d\ln(x)}{dx}=\lim_{h\rightarrow 0}\frac{\ln(x+h)-\ln(x)}{h}=\frac{1}{x}\label{lnd}\ee
Para $h>0$, el \'area bajo la curva $y=1/x$ entre $x$ y $x+h$ 
est\'a comprendida entre dos rect\'angulos de base 
$h$ y altura $1/x$ y $1/(x+h)$, 
\[\frac{h}{x+h}<\ln(x+h)-\ln(x)<\frac{h}{x},\;\;
\Rightarrow \frac{1}{x+h}<\frac{\ln(x+h)-\ln(x)}{h}<\frac{1}{x}\]
de donde, tomando el l\'{\i}mite, 
\[\frac{1}{x}=\lim_{h\rightarrow 0^+}\frac{1}{x+h}\leq\lim_{h\rightarrow 0^+}\frac{\ln(x+h)-\ln(x)}{h}\leq
\lim_{h\rightarrow 0^+}\frac{1}{x}=\frac{1}{x}\]
lo que conduce al resultado (\ref{lnd}). 
%\[\lim_{h\rightarrow 0^+}\frac{\ln(x+h)-\ln(x)}{h}=\frac{1}{x}\]
Para $h<0$,  la longitud de la base es $-h>0$ y 
\[\frac{-h}{x}<\ln(x)-\ln(x+h)<\frac{-h}{x+h},\;\;
\Rightarrow \frac{1}{x}<\frac{\ln(x+h)-\ln(x)}{h}<\frac{1}{x+h}\]
de donde, tomando el l\'{\i}mite $h\rightarrow 0^-$, llegamos nuevamente al resultado 
(\ref{lnd}).  
Como $x>0$, esto implica que $\ln(x)$ es {\it estrictamente creciente}, dado que 
su derivada $1/x$ es siempre positiva. 

\vspace*{9cm}
\epsfxsize=6cm  
\epsffile{lnxa.ps}
\vspace*{-3.6cm}

\epsfxsize=6cm 
\hspace*{6cm}\epsffile{lnxhh.ps}
\vspace*{-5cm}

\newpage
\subsubsection{Propiedades del logaritmo}
Demostremos ahora las propiedades b\'asicas de $\ln(x)$ 
a partir de esta definici\'on. Comencemos con 
\be\ln(x_1x_2)=\ln(x_1)+\ln(x_2),\;\;\forall x_1,x_2>0\label{lxy}\ee
Considerando a $\ln(x_1x_2)$ como funci\'on de $x_1$,  tenemos 
\[\frac{d\ln(x_1x_2)}{dx_1}=\frac{1}{x_1x_2} x_2=\frac{1}{x_1}\] 
es decir,
\[\frac{d(\ln(x_1x_2)-\ln(x_1))}{dx_1}=\frac{1}{x_1}-\frac{1}{x_1}=0\]
de donde $\ln(x_1x_2)-\ln(x_1)=C$, con $C$ constante, es decir, 
\[\ln(x_1x_2)=\ln(x_1)+C\]
Para determinar $C$, evaluamos lo anterior para $x_1=1$, lo que da 
$\ln(x_2)=C$, de donde se obtiene el resultado (\ref{lxy}). 
%(Puede observarse lo indirecto hemos demostrado 
%una de las propiedades fundamentales del logaritmo.  

Demostremos ahora que 
\[\ln(x^n)=n\ln(x),\;\;\;n\in Z\]
Utilizando (\ref{lxy}), tenemos $\ln(x^2)=\ln(x+x)=\ln(x)+\ln(x)=2\ln(x)$. 
Por lo tanto, para $n>2$, entero, obtenemos por inducci\'on
\[\ln(x^n)=\ln(x x^{n-1})=\ln(x)+\ln(x^{n-1})=\ln(x)+(n-1)\ln(x)=(1+n-1)\ln(x)=n\ln(x)\]
Para $n<0$, 
\[0=\ln(1)=\ln(x^{n-n})=\ln(x^n x^{-n})=\ln(x^n)+\ln(x^{-n})\;\;\Rightarrow\;\;\ln(x^{-n})=
-\ln(x^n)=-n\ln(x)\]

Esto conduce a dos importantes resultados. Dado que, por ejemplo,  $\ln(2^n)=n\ln(2)$, con 
$\ln(2)>0$ (pues $2>1$), vemos que el logaritmo toma valores arbitrariamente grandes 
si el argumento es grande. Como adem\'as  $\ln(x)$ es estrictamente creciente, 
esto implica que 
 \[ \lim_{x\rightarrow+\infty}\ln(x)=\infty,\]
Asimismo, como   $\ln(2^{-n})=-n\ln(2)$, con 
$\ln(2)>0$, vemos que el logaritmo toma valores arbitrariamente grandes pero negativos 
cuando el argumento es peque\~no (pues $2^{-n}=1/2^{n}\rightarrow 0$ para $n\rightarrow\infty$). 
Por lo tanto, 
\[\lim_{x\rightarrow 0^+}\ln(x)=-\infty.\]
Como es una funci\'on continua (pues es derivable), 
vemos que el $\ln(x)$ toma todos los valores reales. 

\subsubsection{La funci\'on exponencial}
Acabamos de ver que  $\ln(x)$ es una funci\'on continua estrictamente creciente 
que toma todos los valores reales. Posee entonces una inversa definida para todos los 
reales. Esta inversa es por definici\'on la funci\'on exponencial $\exp$: 
\[y=\ln(x),\;\;x>0,\;\;\;\Rightarrow x=\exp(y),\;\;\; y\in\Re\]
Demostremos ahora las propiedades b\'asicas de esta funci\'on. 
% derivadas todas ellas de las del logaritmo. Comencemos con 
\[\exp(y_1+y_2)=\exp(y_1)\exp(y_2)\]
Llamando $x_1=\exp(y_1)$, $x_2=\exp(y_2)$ $\Rightarrow$ $y_1=\ln(x_1)$, 
$y_2=\ln(x_2)$,  $y_1+y_2=\ln(x_1)+\ln(x_2)=\ln(x_1 x_2)$, 
de donde $\exp(y_1+y_2)=x_1x_2=\exp(y_1)\exp(y_2)$. 
Analogamente, 
\[\exp[ny]=[\exp(y)]^n,\;\;n\;{\rm entero}\]
pues  si $y=\ln(x)$, con $x=\exp(y)$,  $ny=n\ln(x)=\ln(x^n)=\ln\{[\exp(y)]^n\}$, 
de donde $\exp(ny)=[\exp(y)]^n$. 

{\it Se define} el n\'umero $e$ como 
\[e=\exp(1)\]
De esta forma, $\ln(e)=1$, de modo que el area bajo la curva $1/x$ entre $1$ y $e$ es 1. 
Entonces,   
\[\exp(n)=\exp(n 1)=[\exp(1)]^n=e^n\]
En general, {\it se define $e^x$ como }
\[e^x=\exp[x]\]
Calculemos ahora su derivada. Llamando nuevamente $y=\ln(x)$ y $x=\exp(y)$, tenemos 
\ben\frac{d\exp(y)}{dy}&=&\frac{dx}{dy}=\frac{1}{\frac{dy}{dx}}=\frac{1}{\frac{1}{x}}=x
=\exp(y)\een
es decir, $d\,\exp(y)/dy=\exp(y)$. La derivada de la funci\'on exponencial es igual a si misma. 

La exponencial de base $a>0$ {\it se define} como 
\be a^x=\exp[x\ln(a)]\label{axx}\ee
De esta forma 
\[\ln (a^x)=x\ln(a)\]
En particular, para $a=e$, $\ln(e)=1$ y $e^x=\exp(x)$.  
Demostremos que se cumplen con esta definici\'on las propiedades fundamentales 
(\ref{ax1}) y (\ref{ax2}):
\[a^{x_1+x_2}=\exp[(x_1+x_2)\ln(a)]=\exp[x_1\ln(a)+x_2\ln(a)]=
\exp[x_1\ln(a)]+\exp[x_2\ln(a)]=a^{x_1}a^{x_2}\]
\[(a^{x_1})^{x_2}=\exp[x_2\ln(a^{x_1})]=\exp[x_2x_1\ln(a)]=a^{x_1x_2}\] 
La derivada es 
\[\frac{da^x}{dx}=\frac{d\,\exp[x\ln(a)]}{dx}=\exp[x\ln(a)]\ln(a)=a^x\ln(a)\]
o sea que volvemos al resultado (\ref{axd}). 

Para $a>0$, $a\neq 1$, el logaritmo en base $a$ {\it se define} 
como la inversa de la funci\'on (\ref{axx}),  
\[ y=a^x\;\;a>0,\;a\neq 1,\;\; \Rightarrow\;\; x=\log_a(y)\]
Como 
%Tomando logaritmos en ambos miembros de $y=a^x$,  
\[\ln(y)=\ln(a^x)=x\ln(a)=\log_a(y)\ln(a)\]
se obtiene 
\[\log_a(y)=\ln(y)/\ln(a)\]
La derivada es entonces 
\[\frac{d\log_a(y)}{dy}=\frac{1}{y}\frac{1}{\ln(a)}\]

Notemos que si realizamos ahora la derivada de $a^x$  
por definici\'on, siguiendo el camino de la secci\'on 2.1.3, 
obtenemos, utilizando el resultado anterior, 
\[ \lim_{h\rightarrow 0}\frac{a^h-1}{h}=\ln(a),\;\;\;\;\;\;
\lim_{u\rightarrow 0}(1+u)^{1/u}=\exp(1)=e.\]
La existencia de los  l\'{\i}mites anteriores queda entonces demostrada 
a partir de la presente definici\'on del logaritmo natural. Utilizando 
el procedimiento de la secci\'on 2.1.3, llegamos al valor (\ref{en}) para $e$. 

\subsubsection{Ejemplo: Derivada de $x^r$}
Estamos reci\'en ahora en posici\'on seria de poder derivar nuestra conocida 
funci\'on $x^r$, con $r$ real {\it arbitrario}, pues reci\'en ahora hemos definido 
$x^r$ $\forall r$ en forma rigurosa a traves de 
\[x^r=\exp[r\ln(x)],\;\;\;x>0\] 
Tenemos 
\[\frac{dx^r}{dx}=\frac{d\exp[r\ln(x)]}{dx}=\exp[r\ln(x)]\frac{d(r\ln(x))}{dx}=
\exp[r\ln(x)]r\frac{1}{x}=x^r\frac{r}{x}=rx^{r-1}
\]
con lo que llegamos a la generalizaci\'on de la conocida f\'ormula $(x^n)'=nx^{n-1}$ 
derivada originalmente para $n$ natural. 
\subsubsection{Ejemplo: Derivada de $x^x$}
Consideremos la funci\'on 
\[f(x)=x^x,\;\;\;x>0\]
Esta funci\'on no es ni del tipo $a^x$ ni tampoco $x^r$, pues tanto el exponente como 
la base var\'{\i}an. Pero por definici\'on, tenemos que 
\[ x^x=\exp[x\ln(x)]\]
de modo que su derivada es 
\[\frac{dx^x}{dx}=\frac{d\exp[x\ln(x)]}{dx}=\exp[x\ln(x)]\frac{d(x\ln(x))}{dx}=
\exp[x\ln(x)](\ln(x)+1)=x^x(\ln(x)+1)\]
lo que representa la suma de $xx^{x-1}=x^x$ y $x^x\ln(x)$. 

En general, si 
\[f(x)=a(x)^{r(x)},\;\;\;a(x)>0\]
tenemos,  
\[ f(x)=\exp\{r(x)\ln[a(x)]\}\]
\ben f'(x)&=&\exp\{r(x)\ln[a(x)]\}(r(x)\ln[a(x)])'=a(x)^{r(x)}[r'(x)\ln[a(x)]+\frac{r(x)}{a(x)}a'(x)]
\nonumber\\&=&r(x)a(x)^{r(x)-1}a'(x)+a(x)^{r(x)}\ln[a(x)]r'(x)\nonumber\\\een
\subsection{Resumen}
\[\frac{d\,\ln(x)}{dx}=\frac{1}{x},\;\;\;x>0,\;\;\;\;
\frac{d\,\exp(x)}{dx}=\exp(x),\;\;\;x\in\Re \]
Para $a>0$, 
\[a^x=\exp[x\ln(a)],\;\;x\in\Re,\;\;\;\frac{d a^x}{dx}=a^x\ln(a),\]
Para $a>0$, $a\neq 1$, \[\log_a(x)=\frac{\ln(x)}{\ln(a)},\;\;x>0,\;\;\;
\frac{d\,\log_a(x)}{dx}=\frac{1}{x\ln(a)}\]

\vspace*{9cm}
\epsfxsize=6cm  
%%%\epsffile{bcsepar.ps}
\epsffile{ln.ps}
\vspace*{-3.6cm}

\epsfxsize=6cm 
\hspace*{6cm}\epsffile{ex.ps}
\vspace*{-5cm}

\newpage
\subsection{El seno y coseno hiperb\'olicos}
Las siguientes funciones son muy utilizadas en la pr\'actica, y sus propiedades 
poseen cierta similitud con las del seno y coseno usuales.  Definici\'on:
\[\senh(x)=\half[\exp(x)-\exp(-x)],\;\;\;\cosh(x)=\half[\exp(x)+\exp(-x)],\;\;\;\;\;x\in\Re\]
($\senh$ se lee seno hiperb\'olico, $\cosh$ coseno hiperb\'olico). 
A partir de la definici\'on pueden probarse f\'acilmente las siguientes 
propiedades (quedan a cargo del lector)
\[\frac{d\,\senh(x)}{dx}=\cosh(x),\;\;\;\;\frac{d\,\cosh(x)}{dx}=\senh(x),\]
\[\cosh^2(x)-\senh^2(x)=1,\;\;\;\senh(-x)=-\senh(x),\;\;\;\cosh(x)=\cosh(-x),\]
\[\senh(0)=0,\;\;\;\cosh(0)=1,\;\;\;\lim_{x\rightarrow\pm\infty}\senh(x)
=\pm\infty,\;\;\;\lim_{x\rightarrow\pm\infty}\cosh(x)=\infty\]
(La similitud no es casual, y puede mostrarse que $\sen(x)=-i\senh(ix)$, $\cos(x)=\cosh(ix)$, 
con $i$ la unidad imaginaria. El nombre proviene de la definici\'on geom\'etrica de estas funciones, 
en la que $(\cosh(\alpha),\senh(\alpha))$ representa las coordenadas de un punto sobre 
la hip\'erbola $x^2-y^2=1$ (ver figura),  
del mismo modo que $(\cos(\alpha),\sen(\alpha))$ representa 
las coordenadas de un punto sobre la circunferencia $x^2+y^2=1$). 
En forma an\'aloga, la tangente hiperb\'olica 
se define como $\tanh(x)=\senh(x)/\cosh(x)$.


\vspace*{8cm}
\epsfxsize=6cm  
%%%\epsffile{bcsepar.ps}
\epsffile{senh.ps}
\vspace*{-3.6cm}

\epsfxsize=6cm 
\hspace*{6cm}\epsffile{cosh.ps}
\vspace*{-4cm}

\vspace*{9cm}
\epsfxsize=5cm  
\epsffile{coseh.ps}
\vspace*{-5cm}

\epsfxsize=5cm 
\hspace*{7cm}\epsffile{cosehh.ps}
\vspace*{-6cm}

\newpage
\subsection{Orden de magnitud en el infinito}
Intentaremos ahora establecer que tan r\'apido divergen las funciones exponencial 
y logaritmo para $x\rightarrow \infty$ en comparaci\'on con  $x$ o una potencia de $x$. 
Mostremos primero que 
\[\lim_{n\rightarrow+\infty}\frac{(1+a)^n}{n}=\infty,\;\;\;a>0\]
En efecto, utilizando la f\'ormula del binomio de Newton, 
\[\frac{(1+a)^n}{n}=\frac{\sum_{k=0}^n(^n_k)a^n}{n}=
\frac{1+na+{\textstyle\frac{n(n-1)}{2!}}a^2+\ldots}{n}=
\frac{1}{n}+a+\frac{n-1}{2}a^2+\ldots>\frac{1}{n}+a+\frac{n-1}{2}a^2\]
dado que los t\'erminos que restan en el desarrollo son positivos.  
%(creciendo algunos de ellos a\'un m\'as r\'apido con $n$ que 
%los indicados). 
Por lo tanto, como la \'ultima expresi\'on tiende a infinito 
cuando $n\rightarrow\infty$, lo mismo sucede con $(1+a)^n/n$.  

Es f\'acil demostrar ahora el siguiente teorema, de importantes consecuencias:\hfill\break
{\it Teorema}:  La funci\'on $\exp(x)/x$ (es decir $e^x/x$) 
es estrictamente creciente para $x>1$ y 
\[\lim_{x\rightarrow+\infty}\frac{\exp(x)}{x}=\infty\] 
(es decir, la exponencial crece m\'as r\'apido que $x$ para $x\rightarrow\infty$). 
\hfill\break
{\it Demostraci\'on}: Tenemos, si $f(x)=\exp(x)/x$,  
\[f'(x)=\exp(x)(\frac{1}{x}-\frac{1}{x^2})=\frac{\exp(x)}{x}(1-\frac{1}{x})\]
de modo que $f'(x)>0$ si $x>1$. Como $e>1$, podemos escribir $e=1+a$, con $a>0$. 
Entonces, por lo visto anteriormente, para $x$ entero, $e^n/n=(1+a)^n/n$ tiende a 
infinito para $n$ grande. Por lo tanto, como es estrictamente creciente para $x>1$, 
$f(x)$ tambi\'en tomar\'a valores arbitrariamente grandes para valores de $x$ 
no enteros, cuando $x$ crece.  

Veamos ahora las importantes consecuencias: \hfill\break
1)\[ \lim_{x\rightarrow+\infty}\frac{x}{\ln(x)}=\infty\]
(es decir, el logaritmo crece menos r\'apido que $x$ para $x\rightarrow+\infty$). 
\hfill\break
Llamando $x=\exp(u)$, con $u=\ln(x)$, tenemos que $u\rightarrow+\infty$ para $x\rightarrow+\infty$. 
Por lo tanto, 
\[\lim_{x\rightarrow+\infty}\frac{x}{\ln(x)}=\lim_{u\rightarrow+\infty}\frac{\exp(u)}{u}=+\infty\]
2)\[\lim_{x\rightarrow+\infty}\frac{\exp(\lambda x)}{x^r}=\infty,\;\;\;\lambda>0,\]
(es decir, la exponencial crece m\'as r\'apido que {\it cualquier potencia de x} 
cuando $x\rightarrow+\infty$). 
Tenemos, utilizando la definici\'on $x^r=\exp[r\ln(x)]$,   
\[\lim_{x\rightarrow+\infty}\frac{\exp(\lambda x)}{x^r}=
\lim_{x\rightarrow+\infty}\exp[\lambda x-r\ln(x)]=\lim_{x\rightarrow+\infty}
\exp[\lambda x(1-\frac{r}{\lambda}\frac{\ln(x)}{x})]\]
Como, por 1),  $\lim_{x\rightarrow+\infty}\ln(x)/x=0$, vemos que el 
argumento de la exponencial tiende a $\lambda x$ cuando $x\rightarrow+\infty$, 
de modo que el l\'{\i}mite es $\infty$ (Para $r<0$ el resultado es obvio, ya que 
en este caso $1/x^r=1/x^{-|r|}=x^{|r|}$ y el l\'{\i}mite es de la forma $\infty\cdot\infty$) \hfill\break
Ejemplos: 
\[\lim_{x\rightarrow+\infty}\exp(2x)/x^4=\infty\] 
\[\lim_{x\rightarrow+\infty}\exp(2\sqrt{x})/x^2=\lim_{u\rightarrow+\infty}\exp(2u)/u^4
=\infty\;\;\;\;\;\;(u=\sqrt{x})\]  
\[\lim_{x\rightarrow+\infty}\frac{\exp(x^2-x)}{x^3-8 x^2}=
\lim_{x\rightarrow+\infty}\frac{\exp[x^2(1-1/x)]}{x^3(1-8/x)}=
\lim_{x\rightarrow+\infty}\frac{\exp[x^2]}{x^3}=\lim_{u\rightarrow+\infty}
\frac{\exp(u)}{u^{3/2}}=\infty,\;\;\;\;\;(u=x^2)\]
3) \[\lim_{x\rightarrow+\infty}\frac{x^\lambda}{[\ln(x)]^r}=\infty,\;\;\;\lambda>0\]
(es decir, que el logaritmo, as\'{\i} como cualquier potencia $r$ de \'el, 
 crece m\'as lentamente que {\it cualquier potencia positiva} de $x$ 
cuando $x\rightarrow+\infty$). Tenemos, llamando $x=\exp(u)$, con $u=\ln(x)$,  
\[\lim_{x\rightarrow+\infty}\frac{x^\lambda}{[\ln(x)]^r}=
\lim_{x\rightarrow+\infty}\frac{\exp(\lambda u)}{u^r}=\infty\]
Ejemplos:
\[\lim_{x\rightarrow+\infty}\frac{\sqrt{x}}{\ln(x)}=\lim_{x\rightarrow+\infty}\frac{x^{1/2}}
{\ln(x)}=\infty\]
\[\lim_{x\rightarrow+\infty}\frac{x^2}{[\ln(x)]^4}=\infty\]
\[\lim_{x\rightarrow+\infty}\frac{x^2-x\ln(x)}{\ln(x)-\sen(x)}
=\lim_{x\rightarrow+\infty}\frac{x^2(1-\ln(x)/x)}{\ln(x)(1-\sen(x)/\ln(x))}
=\lim_{x\rightarrow+\infty}\frac{x^2}{\ln(x)}=\infty\]
Como otra consecuencia inmediata, notemos que 
los l\'{\i}mites de las rec\'{\i}procas de las anteriores 
funciones son nulos:
\[\lim_{x\rightarrow+\infty}\exp(-x)x=
\lim_{x\rightarrow+\infty}\frac{x}{\exp(x)}=
\lim_{x\rightarrow+\infty}\frac{1}{\frac{\exp(x)}{x}}=\frac{1}{\infty}=0\]
es decir, la funci\'on $\exp(-x)$ decrece {\it m\'as rapidamente} que lo que crece $x$ para 
$x\rightarrow+\infty$. Tambi\'en,  
\[\lim_{x\rightarrow+\infty}\frac{\ln(x)}{x}=
\lim_{x\rightarrow+\infty}\frac{1}{\frac{x}{\ln(x)}}=\frac{1}{\infty}=0\]
Analogamente,
\[\lim_{x\rightarrow+\infty}\exp[-\lambda x] x^r=
\lim_{x\rightarrow+\infty}=\frac{x^r}{\exp(\lambda x)}=0,\;\;\;\lambda>0\]
\[\lim_{x\rightarrow+\infty}\frac{[\ln(x)]^r}{x^\lambda}=0,\;\;\;\lambda>0\]
Ejemplos:
\[\lim_{x\rightarrow+\infty} x^8\exp(-\sqrt{x})=
\lim_{u\rightarrow+\infty}u^{16}\exp(-u)=0\]
\[\lim_{x\rightarrow-\infty}\exp(x)x^3=0\]
\[\lim_{x\rightarrow+\infty}\ln(x)/\sqrt{x}=0\]
%\[\lim_{x\rightarrow-\infty}\ln(|x|)/x^3=0\]
Notemos tambi\'en que  
\[ \lim_{x\rightarrow 0^+}x\ln(x)=\lim_{u\rightarrow+\infty}\frac{1}{u}\ln(\frac{1}{u})=
\lim_{u\rightarrow+\infty}-\frac{\ln(u)}{u}=0\]
donde $u=1/x$. En general, podemos ver que 
\[\lim_{x\rightarrow 0^+}x^\lambda[-\ln(x)]^r=
\lim_{u\rightarrow+\infty}\frac{1}{u^\lambda}[-\ln(\frac{1}{u})]^r=
\lim_{u\rightarrow+\infty}\frac{[\ln(u)]^r}{u^\lambda}=0,\;\;\;\lambda>0\]
Estos l\'{\i}mites implican en particular que 
\[\lim_{x\rightarrow 0^+}x^x=\lim_{x\rightarrow 0^+}\exp[x\ln(x)]=
\exp[\lim_{x\rightarrow 0^+}x\ln(x)]=\exp(0)=1\]
\[\lim_{n\rightarrow \infty}=\sqrt[n]{n}=\lim_{n\rightarrow \infty}=n^{1/n}=
\lim_{n\rightarrow \infty}=\exp[\frac{1}{n}\ln(n)]=
\exp[\lim_{n\rightarrow \infty}\frac{\ln(n)}{n}]=\exp(0)=1\]
\newpage
Ejemplos:\hfill\break
1) Graficar $f(x)=x^x$. Dominio: $x>0$.  Como  $x^x=\exp[x\ln(x)]$,  
\[f'(x)=x^x(\ln(x)+1),\;\;\;f''(x)=x^x[{\textstyle\frac{1}{x}}+(\ln(x)+1)^2]>0\]
La ecuaci\'on $f'(x)=0$ conduce a $\ln(x)+1=0$, es decir $\ln(x)=-1$ y $x=\exp(-1)=e^{-1}\approx 0.368$. 
 Este punto es un m\'{\i}nimo absoluto de $f(x)$, con $f'(x)<0$ si $0<x<e^{-1}$, y $f'(x)>0$ 
si $x>e^{-1}$. \hfill\break
Adem\'as, $\lim_{x\rightarrow+\infty}f(x)=\infty$, y 
$\lim_{x\rightarrow 0^+}f(x)=1$ (pues $\lim_{x\rightarrow 0^+}x\ln(x)=0$). 
Notemos tambi\'en que $\lim_{x\rightarrow 0^+}f'(x)=-\infty$. Como $x^x/e^x=\exp[x\ln(x)-x]=
\exp[x(\ln(x)-1)]$, tenemos $\lim_{x\rightarrow\infty}x^x/e^x=\infty$, 
es decir, $x^x$ crece m\'as rapidamente que $e^x$ para $x\rightarrow\infty$. Esto resulta obvio 
escribiendo $x^x/e^x=(x/e)^x$. 

\vspace*{7.5cm}
\epsfxsize=6cm  
%%%\epsffile{bcsepar.ps}
\epsffile{xx.ps}
\vspace*{-7cm}

2) Graficar $f(x)=\exp[-\frac{x^2}{2a^2}]\;\;\;\;(a>0)$ (esta funci\'on, denominada a veces 
{\it gaussiana}, es de fundamental importancia en estad\'{\i}stica). \hfill\break
Dominio $x\in \Re$. Tenemos $\lim_{x\rightarrow\pm\infty}f(x)=0$, 
\[f'(x)={-\textstyle\exp[-\frac{x^2}{2a^2}]\frac{x}{a^2},\;\;
f''(x)=\exp[-\frac{x^2}{2a^2}]\frac{1}{a^2}(\frac{x^2}{a^2}-1)}\]
La ecuaci\'on $f'(x)=0$ conduce a $x=0$, que es un m\'aximo absoluto. 
La ecuaci\'on $f''(x)=0$ conduce a $x=\pm a$, que son puntos de 
inflexi\'on, con $f''(x)>0$ si $|x|>a$, $f''(x)<0$ si $|x|<a$. 
%Se grafica el caso $a=1/\sqrt{2}\approx 0.707$ ($2a^2=1$). 

\vspace*{7.5cm}
\epsfxsize=6cm  
\epsffile{exp22.ps}
\vspace*{-7cm}

3) Graficar $f(x)=\exp[-\frac{1}{x}]$. \hfill\break
Dominio $x\neq 0$. Tenemos $\lim_{x\rightarrow\pm\infty}f(x)=1$, 
$\lim_{x\rightarrow 0^+}f(x)=0$, $\lim_{x\rightarrow 0^-}f(x)=\infty$, 
\[f'(x)={\textstyle\exp[-\frac{1}{x}]\frac{1}{x^2},\;\;\;\;
f''(x)=\exp[-\frac{1}{x}](\frac{1}{x^4}-\frac{2}{x^3})=
\exp[-\frac{1}{x}]\frac{1}{x^4}(1-2x)}\]
La ecuaci\'on $f'(x)=0$ no tiene soluci\'on, de modo que 
$f(x)$ es estrictamente creciente para $x<0$, y para $x>0$. 
La ecuaci\'on $f''(x)=0$ conduce a $x=\frac{1}{2}$, que es un punto de 
inflexi\'on, con $f''(x)>0$ si $x<0$ o $0<x<\frac{1}{2}$, y $f''(x)<0$ si $x>\frac{1}{2}$. 
Notemos tambi\'en que $\lim_{x\rightarrow 0^+}f'(x)=\lim_{x\rightarrow 0^+}f''(x)=0$.

\vspace*{7.5cm}
\epsfxsize=6cm  
\epsffile{expusx.ps}
\vspace*{-5.6cm}
\newpage

4) Graficar \[f(x)=\{^{\exp[-1/x^2],\;\;x\neq 0}_{0,\hspace*{1.4cm}x=0}\]
y mostrar que es continua y derivable en $x=0$.   

Tenemos \[\lim_{x\rightarrow 0}f(x)=\lim_{u\rightarrow\infty}\exp(-u)=0,\;\;\;u=1/x\] 
 de modo que es continua en $x=0$. Adem\'as, 
\[f'(x)=\exp[-1/x^2]\frac{2}{x^3},\;\;x\neq 0,\;\;\;\;\lim_{x\rightarrow 0^{\pm}}
f'(x)=\lim_{w\rightarrow\pm\infty}\exp[-w^2]2 w^3=0,\;\;\;w=1/x\]
\[f'(0)=\lim_{h\rightarrow 0}\frac{\exp[-1/h^2]-0}{h}=\lim_{h\rightarrow 0}
\exp[-1/h^2]\frac{1}{h}=\lim_{w\rightarrow\pm\infty}\exp[-w^2]w=0,\;\;\;w=1/h\]
Es decir, $f'(0)=0$ y adem\'as $f'(x)$ es continua en $x=0$. 
Asimismo 
\[f''(x)=\exp[-1/x^2](\frac{4}{x^6}-\frac{6}{x^4})=\exp[-1/x^2]\frac{2}{x^6}(2-3x^2),\;\;x\neq 0,\]
\[\lim_{x\rightarrow 0^{\pm}}f''(x)=
\lim_{w\rightarrow \pm\infty}\exp[-w^2][4w^6-6w^4]=0\]
\[f''(0)=\lim_{h\rightarrow 0}\frac{\exp[-1/h^2]\frac{2}{h^3}-0}{h}=
\lim_{h\rightarrow 0}\exp[-1/h^2]\frac{2}{h^4}=\lim_{w\rightarrow\pm\infty}\exp[-w^2]2w^4=0\]
de modo que $f''(0)=0$  y $f''(x)$ es tambi\'en continua en $x=0$. 
 Puede mostrarse de igual forma que la derivada {\it en\'esima} $f^{(n)}(x)$ es continua en $x=0$, 
es decir $f^{(n)}(0)=0$ y  $\lim_{x\rightarrow 0^{\pm}}f^{(n)}(x)=0$ (la raz\'on esencial es que 
$\lim_{x\rightarrow 0^{\pm}}\exp[-1/x^2]\frac{1}{x^m}=0$ $\forall m>0$). 

Notemos que $x=0$ es un m\'{\i}nimo absoluto de $f(x)$, ya que $f'(x)>0$ si $x>0$ y $f'(x)<0$ 
si $x<0$. La ecuaci\'on $f''(x)=0$ conduce a $x=0$ o $2-3x^2=0$, es decir $x=\pm\sqrt{2/3}$. 
Como $f''(x)<0$ si $|x|>\sqrt{2/3}$ y $f''(x)>0$ si $|x|<\sqrt{2/3}$, 
tenemos que $x=\pm\sqrt{2/3}$ son puntos de inflexi\'on mientras que $x=0$ no es punto de inflexi\'on. 
 
Finalmente, tenemos $\lim_{x\rightarrow\pm\infty}f(x)=\exp(0)=1$, 
y $\lim_{x\rightarrow\pm\infty}f'(x)=0$. Puede verse en la figura que la funci\'on es muy plana 
en la vecindad de $x=0$ (es decir, m\'as plana que cualquier potencia $x^n$, $n>0$, en esa regi\'on). 

\vspace*{7.5cm}
\epsfxsize=6cm  
\epsffile{expusx2.ps}
\vspace*{-6.6cm}

De los resultados anteriores es f\'acil ver que la funci\'on 
\[f(x)=\{^{\exp[-1/x^2],\;\;\;\;x>0}_{0,\hspace*{1.6cm}x\leq 0}\]
resulta no s\'olo continua en $x=0$, sino tambi\'en derivable a cualquier orden 
en $x=0$, con $f^{(n)}(0)=0$. Lo mismo sucede si reemplazamos, por ejemplo,  $\exp[-1/x^2]$ por 
$\exp[-1/x]$ (se dejan los detalles para el lector). 
\newpage 
5) Para visualizar la competencia entre exponenciales y potencias,  
podemos considerar por ejemplo 
\[ f(x)=e^{-x} x^n,\;\;\;x\geq 0,\;\;\;n\geq 1\]
Tenemos, $f(0)=0$, $\lim_{x\rightarrow+\infty}f(x)=0$, y 
\[f'(x)=e^{-x}x^{n-1}(n-x),\;\;f''(x)=e^{-x}x^{n-2}((x-n)^2-n)\]
La ecuaci\'on $f'(x)=0$ conduce a $x=n$, que es un m\'aximo absoluto para $x>0$, 
donde $f(x)=e^{-n} n^n=e^{n(\ln(n)-1)}$ 
(este valor m\'aximo crece al aumentar $n$ para $n\geq 1$, 
aproxim\'andose a $n!$ para $n$ grande).  
La ecuaci\'on $f''(x)=0$ conduce a $x=n\pm\sqrt{n}$, 
que son ambos puntos de inflexi\'on si $n>1$ (pues 
$f''(x)<0$ si $-\sqrt{n}+n<x<n+\sqrt{n}$,  o sea $|x-n|<\sqrt{n}$, 
y $f''(x)>0$ si $|x-n|>\sqrt{n}$). Para $n=1$, s\'olo $x=n+\sqrt{n}=2$ es 
punto de inflexi\'on. Vemos as\'{\i} que mientras $x^n$ ``domina la situaci\'on'' 
para $x<n$  (donde $f(x)$ crece), es $e^{-x}$ la que domina para 
$x>n$ (donde $f(x)$ decrece). Notemos tambi\'en que el ``ancho'' del m\'aximo es 
aproximadamente $2\sqrt{n}$. 
%Este tipo de funciones es muy com\'un en estudios estad\'{\i}sticos. 

\vspace*{7.5cm}
\epsfxsize=6cm  
\epsffile{exxn.ps}
\vspace*{-6.6cm}

6) Graficar $f(x)=\{^{-x\ln(x),\;\;0<x\leq 1}_{0,\;\;\;\;\;\;\;\;\;\;\;\;x=0}$\hfill\break
%Mostrar que es continua (por derecha) en $x=0$ 
%(es decir, $\lim_{x\rightarrow 0^+}f(x)=f(0)=0)$. 
%\hfill\break
%Si bien el dominio natural de $-x \ln(x)$ es $x>0$, 
Esta funci\'on tiene particular importancia en cuestiones 
estad\'{\i}sticas ($S=\sum_if(p_i)$ es la {\it entrop\'{\i}a} de una distribuci\'on de probabilidad 
$(p_1,\ldots,p_n)$, con $0\leq p_i\leq 1$; es un medida del desorden asociado). 

Notemos que  $f(x)>0$ si $x\in(0,1)$, (pues $\ln(x)< 0$  en este intervalo), 
con $f(0)=f(1)=0$. \hfill\break
Si bien $\lim_{x\rightarrow 0^+}\ln(x)=-\infty$,  el l\'{\i}mite de $f(x)$ para $x\rightarrow 0^+$ 
es nulo:  
\[\lim_{x\rightarrow 0^+}f(x)=\lim_{x\rightarrow 0^+}-x\ln(x)=\lim_{w\rightarrow+\infty}
e^{-w}w=0,\;\;\;\;(w=-\ln(x),\;\;\;x=e^{-w})\]
Es decir, que $f(x)$ es continua por derecha en $x=0$ 
($\lim_{x\rightarrow 0^+}f(x)=f(0)$). Adem\'as,
\[f'(x)=-1-\ln(x),\;\;\;x>0,\;\;\;\;\;\;f''(x)=-1/x,\;\;x>0\]
lo que implica $\lim_{x\rightarrow 0^+}f'(x)=\lim_{x\rightarrow 0^+}f''(x)=\infty$.  
La derivada por derecha en $x=0$ no existe: 
$\lim_{h\rightarrow 0^+}\frac{f(h)-f(0)}{h}=\lim_{h\rightarrow 0^+}\frac{-h\ln h}{h}
=-\lim_{h\rightarrow 0^+}\ln h=+\infty$. 
%Para $x\rightarrow 0^+$, la situaci\'on similar al caso  $g(x)=\sqrt{x}$, 
%donde $g(0)=0$, pero $g'(x)=\frac{1}{2\sqrt{x}}\rightarrow+\infty$ para $x\rightarrow 0^+$. 

La ecuaci\'on $f'(x)=0$ conduce a  $\ln(x)=-1$, cuya soluci\'on es 
$x=e^{-1}$, con $f(e^{-1})=e^{-1}$. Este punto es un 
{\it m\'aximo absoluto} de $f(x)$ para $x\in[0,1]$, ya que $f'(x)>0$ si $0<x<e^{-1}$, y 
$f'(x)<0$ si $x>e^{-1}$. Adem\'as, $f''(x)<0$ si $x>0$.  
El m\'{\i}nimo absoluto de $f(x)$ en $[0,1]$ es tomado en los extremos 
($f(x)=0$ si $x=0$ o $x=1$). Dado que $-x\ln(x)=-\ln(x^x)$, el comportamiento 
de $f(x)$ puede tambi\'en inferirse  de la gr\'afica del ejemplo 1. 

\vspace*{7.5cm}
\epsfxsize=6cm  
\epsffile{xlog.ps}
\vspace*{-6.6cm}

\newpage 
7) La funci\'on exponencial aparece naturalmente en muchos procesos f\'{\i}sicos, qu\'{\i}micos y 
estad\'{\i}sticos. La raz\'on es la existencia de una cierta magnitud  $f(t)$ 
cuya velocidad de variaci\'on en el tiempo $df/dt$ es proporcional a ella misma: 
\be\frac{df(t)}{dt}=k f(t)\label{ev}\label{e1x}\ee
con $k$ constante. Por ejemplo, si $f(t)$ es la poblaci\'on (N$^{\rm o}$ de personas) 
de una cierta ciudad o zona, 
podemos esperar que su velocidad de variaci\'on 
(es decir, el n\'umero de nacimientos menos el de 
muertes por unidad de tiempo) sea proporcional  al N$^{\rm o}$ de personas en ese instante.  
La misma ecuaci\'on es satisfecha por el n\'umero de part\'{\i}culas  
radioactivas $f(t)$ que permanecen sin decaer despues de un tiempo $t$, dado que 
el n\'umero de part\'{\i}culas que decaen por unidad de tiempo (es decir,  $-df/dt$) 
es proporcional a $f(t)$. 

 Veamos como debe ser $f(t)$. Dividiendo (\ref{e1x}) por $f(t)$ 
(asumiendo  $f(t)>0$) tenemos 
\[ \frac{1}{f(t)}\frac{df(t)}{dt}=k,\;\;\;{\rm es\;decir,\;\;\;\;}\frac{d\ln f(t)}{dt}=k\;\]
dado que $\frac{d\ln f}{dt}=\frac{1}{f}\frac{df}{dt}$. Esto indica que la velocidad de variaci\'on del 
logaritmo de $f(t)$ es {\it constante}. Si la derivada de una funci\'on es constante, 
la pendiente de su gr\'afica en funci\'on del tiempo es constante. Por lo tanto, 
dicha funci\'on debe ser una funci\'on {\it lineal} de la variable $t$ y su 
gr\'afica debe ser una recta con pendiente $k$, es decir, 
\[\ln f(t)=kt+b,\;\;\;\;\Rightarrow f(t)=e^{kt+b}=ce^{kt},\;\;\;c=e^b\]
donde $b$ es una constante arbitraria. Vemos pues  que si $f(t)$ satisface (\ref{ev}), 
entonces es una funci\'on exponencial. Para determinar $b$ o $c$, 
 notemos que a $t=0$ (tiempo inicial), $f(0)=ce^0=c$, o sea que  
%podemos 
%escribir el resultado final como 


\[ f(t)=f(0)e^{kt}\] 
Este resultado es v\'alido tanto para $f(0)>0$ o $f(0)<0$ (en cuyo caso hubiesemos escrito 
$\frac{1}{f}\frac{df}{dt}=\frac{d\ln(-f)}{dt}$). 

Si $k>0$ (como en el caso de la poblaci\'on de una cierta zona), 
se dice que la funci\'on $f(t)$ tiene un {\it crecimiento exponencial}. La cantidad 
inicial $f(0)$ se duplica para $t=\tau=(\ln 2)/k$, ya que $f(\tau)=2f(0)$, 
volvi\'endose a duplicar para $t=2\tau$, con  $f(2\tau)=4f(0)$. 
En general,  $f(n\tau)=2^nf(0)$, lo que indica un crecimiento 
``explosivo'' de $f(t)$. Podemos reescribirla como 
\[ f(t)=f(0)e^{(\ln 2)t/\tau}=f(0)2^{t/\tau},\;\;\;\tau=(\ln 2)/k>0\]

Si $k<0$ (como en el caso de las part\'{\i}culas radiactivas), 
la funci\'on $f(t)$ posee un {\it decrecimiento exponencial}. La cantidad 
inicial $f(0)$ se reduce a la mitad para $t=\tau=-(\ln 2)/k$, ya que $f(\tau)=f(0)e^{-\ln 2}=\half f(0)$, 
volvi\'endose a reducir a la mitad para $t=2\tau$, con $f(2\tau)=\quart f(0)$. 
Podemos reescribirla como 
\[ f(t)=f(0)e^{-(\ln 2)t/\tau}=f(0)(\half)^{t/\tau},\;\;\;\tau=-(\ln2)/k>0\]
con $f(n\tau)=f(0)/2^n$. En este caso $\tau$ es denominado a veces  ``tiempo de vida'', 
ya que $f(t)/f(0)$ se vuelve muy peque\~no para $t\gg \tau$. 
\vspace*{7.5cm}

\vspace*{.5cm}
\epsfxsize=6cm  
\epsffile{expt.ps}
\vspace*{-3.6cm}

\epsfxsize=6cm  
\hspace*{7cm}\epsffile{exptm.ps}
\vspace*{-6.6cm}

\newpage
\section{Desarrollo en serie de Taylor}
En esta secci\'on veremos como expresar aproximadamente una funci\'on 
 en la vecindad de un punto, a partir de sus derivadas sucesivas en ese punto. 
Esto es de fundamental importancia 
por varias razones: \hfill\break
 1) Permite evaluar funciones ``elementales'' como el seno, 
el logaritmo y la exponencial, en las proximidades de un punto donde el valor de la 
 funci\'on y sus derivadas son conocidos, con un grado de precisi\'on arbitrario. \hfill\break
 2) En muchas situaciones pr\'acticas 
la funci\'on real con la que se debe tratar es muy compleja, 
siendo deseable tener  una aproximaci\'on sencilla (es decir, de tipo polinomial) de la funci\'on 
  que contenga las caracter\'{\i}sticas esenciales de la misma en la vecindad de 
un punto. \hfill\break
%Por ejemplo, en qu\'{\i}mica o f\'{\i}sica te\'orica surgen frecuentemente 
%funciones (tales como la ener\'{\i}a de un sistema en funci\'on de un cierto par\'ametro) 
% cuya evaluaci\'on en un \'unico punto puede demorar horas, 
%a\~nos o siglos,  a'un con los m\'as modernos sistemas de c\'omputo. 
%Resulta entonces necesario poseer una estimaci\'on sencilla de la funci\'on 	
%en la regi\'on de inter\'es.  \hfill\break
3) En muchos casos pr\'acticos resulta directamente imposible conocer la funci\'on de inter\'es 
en forma exacta, y s\'olo pueden calcularse o conocerse las derivadas de la funci\'on 
en un punto. Surge entonces la necesidad de reconstruir aproximadamente la funci\'on 
 en la vecindad del punto a partir de estas derivadas. 
Por ejemplo, puede ser muy dif\'{\i}cil conocer exactamente 
la funci\'on $x(t)$ que representa la posici\'on de un automovil que circula 
por una ruta (o la posici\'on de una part\'{\i}cula en una cierta trayectoria),  
en funci\'on del tiempo $t$. Sin embargo,  puede resultar f\'acil  medir su posici\'on 
$x(t_0)$,  velocidad $x'(t_0)$ y aceleraci\'on $x''(t_0)$ en un instante 
dado $t_0$. Es de esperar que a partir de estas mediciones podamos predecir aproximadamente 
la posici\'on $x(t)$ al menos para tiempos cercanos al instante de medici\'on.  
%
%Resulta tambi\'en  obvio que tales estimaciones no ser\'an v\'alidas despu\'es de 
%un hecho ``imprevisto'' (frenada o choque con otro automovil o part\'{\i}cula), 
%es decir, sin relaci\'on con las derivadas en el instante de medici\'on. 
%Podemos  entonces esperar en general un cierto {\it rango de validez} de las 
%estimaci\'on de una funci\'on a partir de sus derivadas en un punto. 
%y se anula para $x=0$ a no ser 

\subsection{Expresi\'on de un polinomio en t\'erminos de sus derivadas en un punto}
Comenzaremos por expresar un polinomio arbitrario de grado $n$ en 
t\'erminos de sus derivadas en un punto. Sea 
\[P(x)=a_0+a_1x+a_2x^2+a_3x^3+\ldots+a_nx^n,\;\;\;a_n\neq 0,\]
Tenemos 
\[P(0)=a_0\]
Derivando,  
\[P'(x)=a_1+2a_2x+3a_3x^2+\ldots+na_nx^{n-1}\]
obtenemos 
\[P'(0)=a_1\]
Derivando nuevamente, 
\[P''(x)=2a_2+3.2 a_3 x+\ldots+n(n-1)x^{n-2}\]
obtenemos 
\[P''(0)=2a_2\]
En general, podemos ver que la derivada $k$-\'esima en $x=0$ es 
\[ P^{(k)}(0)=k!a_k,\;\;\;\;\;0\leq k\leq n\]
dado que $(x^{k})^{(k)}=k!$ (para $k=0$, $P^{(0)}(x)$ denota $P(x)$; 
recordemos  que $0!=1$, $1!=1$). Adem\'as,  
 $P^{(k)}(x)=0$ si $k>n$.  Por lo tanto  
\[a_k=\frac{P^{(k)}(0)}{k!}\]
Podemos entonces reescribir el polinomio  como 
\be P(x)=P(0)+P'(0)x+\frac{P''(0)}{2!}x^2+\frac{P'''(0)}{3!}x^3+\ldots+
\frac{P^{(n)}(0)}{n!}x^n \label{P0}\ee
quedando pues completamente determinado por $P(0)$ y las primeras $n$ derivadas en el origen. 

Si reescribimos $P(x)$ como un polinomio en $(x-x_0)$, es decir 
\[P(x)=b_0+b_1(x-x_0)+b_2(x-x_0)^2+\ldots+b_n(x-x_0)^n\]
(donde $b_n=a_n$, $b_{n-1}=nx_0a_n+a_{n-1}$, $b_{n-2}=(^n_2)a_nx_0^2+(^n_1)a_{n-1}x_0+a_{n-2}$, 
etc.), entonces 
\[ P(x_0)=b_0,\;\;P'(x_0)=b_1,\;\;P''(x_0)=2b_2\]
y en general 
\[P^{(k)}(x_0)=k!b_k,\;\;\;\;0\leq k\leq n\]
Por lo tanto, $b_k=\frac{P^{(k)}(x_0)}{k!}$ y podemos re-escribir $P(x)$ 
tambi\'en como 
\[P(x)=P(x_0)+P'(x_0)(x-x_0)+\frac{P''(x_0)}{2!}(x-x_0)^2+\frac{P'''(x_0)}{3!}(x-x_0)^3+
\ldots+\frac{P^{(n)}(x_0)}{n!}(x-x_0)^n\]
Un polinomio de grado $n$ queda entonces completamente 
determinado por su valor y sus primeras $n$ derivadas en un punto $x_0$ arbitrario.  
\hfill\break\break
Ejemplo: Obtener la f\'ormula del binomio 
\be(c+x)^n=\sum_{k=0}^n(^n_k)c^{n-k}x^k=c^n+nc^{n-1}x+\frac{n(n-1)}{2!}c^{n-2}x^2+
\ldots+x^n\label{N}\ee
donde $(^n_k)=\frac{n!}{k!(n-k)!}$. Dado que 
$P(x)=(c+x)^n$ es un polinomio de grado $n$, con 
$P(0)=c^n$,  $P'(0)=nc^{n-1}$, $P''(0)=n(n-1)c^{n-2}$ y en general 
\[P^{(k)}(0)=\frac{n!}{(n-k)!}c^{n-k},\;\;\;0\leq k\leq n\]  
(y $P^{(k)}(x)=0$ si $k>n$), la ecuaci\'on (\ref{P0}) 
conduce entonces al resultado (\ref{N}).  

\subsection{Polinomio de Taylor}
Consideremos ahora una funci\'on $f(x)$ arbitraria,  pero 
tal que sus derivadas $f^{(k)}(x_0)$ existen para $0\leq k\leq n$ en un determinado punto 
$x_0$. El polinomio 
%\[f'(x_0),\;\;f''(x_0),\;\;\ldots\;\;f^{(n)}(x_0)\] 
%existen todas para un determinado  $x_0$. Entonces el polinomio 
\[P_{n,x_0}(x)=f(x_0)+f'(x_0)(x-x_0)+\frac{f''(x_0)}{2!}(x-x_0)^2+\ldots+\frac{f^{(n)}(x_0)}{n!}
(x-x_0)^n\]
se denomina {\it Polinomio de Taylor} de grado $n$ en $x_0$  para $f(x)$. 
% Para $x_0=0$, 
%\[P_{n,0}(x)=f(0)+f'(0)x+\frac{f''(0)}{2!}x^2+\ldots+\frac{f^{(n)}(0)}{n!}
%x^n\]
La propiedad {\it fundamental} del polinomio de Taylor es que su valor en $x=x_0$ y el de 
sus primeras $n$ derivadas en $x=x_0$ {\it coinciden todas con las de} $f(x)$ en ese punto:  
$P_{n,x_0}(x_0)=f(x_0)$, $P'_{n,x_0}(x_0)=f'(x_0)$ y en general, 
%Sus derivadas en $x_0$ satisfacen obviamente 
\[ P^{(k)}_{n,x_0}(x_0)=k!\frac{f^{(k)}(x_0)}{k!}=f^{(k)}(x_0),\;\;\;0\leq k\leq n \]
Existe obviamente un {\it \'unico} polinomio de grado $n$ con estas caracter\'{\i}sticas, 
ya que este queda completamente determinado por su valor y 
sus primeras $n$ derivadas en $x_0$. 
 
Si $f(x)$ es un polinomio de grado $n$, entonces $P_{n,x_0}(x)=f(x)$ $\forall$ $x,x_0$. 
En el caso general,  $P_{n,x_0}(x)$ constiuir\'a s\'olo una {\it aproximaci\'on} a $f(x)$, v\'alida 
en las cercan\'{\i}as de $x_0$, que mejora al aumentar $n$ 
(como luego veremos con m\'as detalle). Notemos que para $n=1$, 
\[P_{1,x_0}(x)=f(x_0)+f'(x_0)(x-x_0) \] %\]
es {\it la recta tangente} a $f(x)$ 
en $x=x_0$. Analogamente,  $y=P_{2,x_0}(x)$ es la ecuaci\'on de una par\'abola cuya ordenada,  
pendiente y derivada segunda en  $x=x_0$ coinciden con las de $f(x)$ 
en $x=x_0$. 

%Notemos que $\lim_{x\rightarrow x_0}f(x)-P_{n,x_0}(x)=f(x_0)-P_{n,x_0}(x_0)=0$. 
 Es f\'acil probar que  
\[\lim_{x\rightarrow x_0}\frac{f(x)-P_{n,x_0}(x)}{(x-x_0)^n}=0\]
lo que implica que la diferencia $f(x)-P_{n,x_0}(x)$ no s\'olo es peque\~na cuando 
$x\rightarrow x_0$, sino que es peque\~na a\'un 
en comparaci\'on con $(x-x_0)^n$  (y por lo tanto con $(x-x_0)^k$ para  $0\leq k\leq n$). 
%$,pues $\lim_{x\rightarrow x_0}\frac{f(x)-P_{n,x_0}(x)}{(x-x_0)^n}$
Podemos aplicar la regla de L'Hopital $n-1$ veces sucesivas, ya que 
para $0\leq k\leq n-1$, $f^{(k)}(x)$ es necesariamente continua en $x=x_0$, con 
$\lim\limits_{x\rightarrow x_0}f^{(k)}(x)=f^{(k)}(x_0)= 
\lim\limits_{x\rightarrow x_0}P_{n,x_0}^{(k)}(x)$.  
Como  $P^{(n-1)}_{n,x_0}(x)=f^{(n-1)}(x_0)+f^{(n)}(x_0)(x-x_0)$, 
\[\lim_{x\rightarrow x_0}\frac{f(x)-P_{n,x_0}(x)}{(x-x_0)^n}=
\lim_{x\rightarrow x_0}\frac{f^{(n-1)}(x)-P^{(n-1)}_{n,x_0}(x)}{n!(x-x_0)}=
\frac{1}{n!}[\lim_{x\rightarrow x_0}\frac{f^{(n-1)}(x)-f^{(n-1)}(x_0)}{x-x_0}-
f^{(n)}(x_0)]=0\] 
donde  el l\'{\i}mite del \'ultimo cociente es, por definici\'on, $f^{(n)}(x_0)$. 

\newpage
Ejemplos:\hfill\break
1) 
\[f(x)=\sen(x)\]
Para $x_0=0$,  obtenemos $f(0)=\sen(0)=0$, $f'(0)=\cos(0)=1$, 
$f''(0)=-\sen(0)=0$, $f'''(0)=-\cos(0)=-1$ y en general, $f^{(2k)}(0)=0$, 
$f^{(2k+1)}(0)=(-1)^k$, con $k$ natural.  Entonces, 
\[P_{2n+1,0}(x)=x-\frac{x^3}{3!}+\frac{x^5}{5!}+\ldots+\frac{(-1)^{n}x^{2n+1}}{(2n+1)!}\]
Dado que $\sen(-x)=-\sen(x)$ (funci\'on impar)
 es obvio que no pueden aparecer potencias pares en el 
Polinomio de Taylor para $x_0=0$. \hfill\break\break
%Este debe respetar las 
%simetr\'{\i}as locales (es decir, en la vecindad de $x=x_0$)  de la funci\'on. \hfill\break
2) 
\[f(x)=\cos(x)\]
Para $x_0=0$, obtenemos  $f(0)=1$, $f'(0)=0$, $f''(0)=-1$, $f'''(0)=0$ y en general, 
$f^{(2k+1)}(0)=0$, $f^{(2k)}(0)=(-1)^{n}$. Por lo tanto,  
\[P_{2n,0}(x)=1-\frac{x^2}{2!}+\frac{x^4}{4!}+\ldots+\frac{(-1)^n x^{2n}}{(2n)!}\]
En este caso s\'olo pueden aparecer potencias pares pues $\cos(x)=\cos(-x)$. 
Se observa en la figura ($n$ indica el grado del Polinomio) 
que $P_{1,0}(x)$ y $P_{2,0}(x)$ constituyen ya una muy buena 
aproximaci\'on al seno y coseno respectivamente para $|x|<\pi/4$. 
%mientras que $P_{3,0}(x)$ y $P_{4,0}(x)$ lo es del coseno. 
%y $P_2(x)$ en el coseno los primeros Polinomios de Tauloy
\vspace*{7.8cm}

\epsfxsize=6cm  
\epsffile{taylsen.ps}
\vspace*{-3.6cm}

\epsfxsize=6cm  
\hspace*{7cm}\epsffile{taylcos.ps}
\vspace*{-7cm}

3)\[f(x)=\exp(x)\]
En este caso $f^{(k)}(x_0)=\exp(x_0)$ y por lo tanto, 
\[P_{n,x_0}(x)=\exp(x_0)[1+(x-x_0)+\frac{(x-x_0)^2}{2!}+\ldots+\frac{(x-x_0)^n}{n!}]\]
En particular, para $x_0=0$, $\exp(0)=0$ y 
\[ P_{n,0}(x)=1+x+\frac{x^2}{2!}+\frac{x^3}{3!}+\ldots+\frac{x^n}{n!}\]
Notemos que si $x=1$, volvemos a obtener el resultado (\ref{e}) para $e$ en el 
l\'{\i}mite $n\rightarrow\infty$.  \hfill\break\break
%Se muestran en la figura los polinomios de Taylor para $n=1,2 y 4$, 
%\newpage
4)
\[f(x)=\ln(x)\]
Tomemos en este caso $x_0=1$ (ya que $\ln(x)$ no est\'a definido en $x=0$). 
Tenemos  $f'(x)=1/x$, $f''(x)=-1/x^2$, $f'''(x)=2/x^3$, $f^{(4)}(x)=-3!/x^4$ 
y en general $f^{(k)}(x)=(-1)^{k-1}(k-1)!/x^k$. Por lo tanto, 
$f^{(k)}(1)=(-1)^{k-1}(k-1)!$. Teniendo en cuenta que $(k-1)!/k!=1/k$, 
\[P_{n,1}(x)=(x-1)-\frac{(x-1)^2}{2}+\frac{(x-1)^3}{3}+\ldots+\frac{(-1)^{n-1}(x-1)^n}{n}\]
\newpage
Si llamamos $u=x-1$, es decir $x=1+u$, 
la anterior expresi\'on corresponde al polinomio de Taylor de 
$\ln(u+1)$ alrededor de $u=0$. Es decir, si 
\[ f(x)=\ln(1+x)\] 
entonces 
\[P_{n,0}(x)=x-\frac{x^2}{2}+\frac{x^3}{3}+\ldots+\frac{(-1)^{n-1}x^n}{n}\]
Se muestran en la figura las gr\'aficas correspondientes a los Polinomios de Taylor 
para $n=1,2$ y 4. 
\vspace*{8.cm}

\epsfxsize=6cm  
\epsffile{taylexp.ps}
\vspace*{-3.6cm}

\epsfxsize=6cm  
\hspace*{7cm}\epsffile{taylln.ps}
\vspace*{-6.5cm}

5)\[f(x)=\senh(x)\]
Recordando que $(\senh(x))'=\cosh(x)$, y $(\cosh(x))'=\senh(x)$, 
para $x_0=0$, tenemos $f(0)=0$, $f'(0)=1$, y en general, $f^{(2k)}(0)=0$, 
$f^{(2k+1)}(0)=1$. Por lo tanto 
\[P_{2n+1,0}(x)=x+\frac{x^3}{3!}+\frac{x^5}{5!}+\ldots+\frac{x^{2n+1}}{(2n+1)!}\]
S\'olo aparecen potencias impares pues $\senh(-x)=-\senh(x)$. 
Notemos la similitud con el resultado para $\sen(x)$ 
(para $x\rightarrow ix$, con $i^2=-1$, 
el polinomio anterior coincide con $i$ veces el polinomio de Taylor 
para $\sen(x)$, lo que  sugiere que $\senh(ix)=i\sen(x)$,  
 como puede mostrarse en forma rigurosa). \hfill\break\break
6) \[f(x)=\cosh(x)\]
En este caso, $f(0)=1$, $f'(0)=0$, y en general, $f^{(2k)}(0)=1$, 
$f^{(2k+1)}(0)=0$. Por consiguiente, 
\[P_{2n,0}(x)=1+\frac{x^2}{2!}+\frac{x^4}{4!}+\ldots+\frac{x^{2n}}{(2n)!}\]
S\'olo aparecen potencias pares pues $\cosh(x)=\cosh(-x)$. Notemos nuevamente la 
similitud con el polinomio de Taylor para $\cos(x)$ 
(para $x\rightarrow ix$, el polinomio anterior coincide con el correspondiente a 
$\cos(x)$, sugiriendo que $\cosh(ix)=\cos(x)$).  \hfill\break\break
7) \[f(x)=\frac{1}{1-x}\] 
En este caso $f'(x)=\frac{1}{(1-x)^2}$, $f''(x)=\frac{2}{(1-x)^3}$, 
$f'''(x)=\frac{3!}{(1-x)^4}$, y en general, $f^{(k)}(x)=\frac{k!}{(1-x)^k}$. 
Tomando $x_0=0$, tenemos $f^{(k)}(0)=k!$ y el Polinomio de Taylor adopta una 
forma particularmente sencilla,  
\[P_{n,0}(x)=1+x+x^2+x^3+\ldots+x^n\]
\newpage
8)\[ f(x)=(c+x)^r,\]
Considerando $c>0$ y tomando  $x_0=0$, obtenemos 
$f(0)=c^r$, $f'(0)=rc^{r-1}$, $f''(0)=r(r-1)c^{r-2}$ y 
en general, $f^{(k)}(0)=r(r-1)\ldots (r-k+1)c^{r-k}$. Por lo tanto, 
\[ P_{n,0}(x)=c^r+rc^{r-1}x+\frac{r(r-1)}{2!}c^{r-2}+\ldots+\frac{r(r-1)\ldots(r-n+1)}{n!}
c^{r-n}x^n\]%\nonumber\\&=&\sum_{k=0}^n(^r_k)c^{r-k}x^k,\;\;\;(^r_k)\equiv
%\frac{\frac{r(r-1)\ldots(r-k+1)}{k!}\een 
%dando una expresi\'on {\it aproximada}. 
% Si $r=m$, con $m$ entero, 
%el coeficiente en\'esimo se anula para $n>m$ y reobtenemos la f\'ormula usual. 
%Si $r$ no es entero, el coeficiente en\'esimo permanece  no nulo para $n>r$. 
Definiendo 
\[(^r_k)=\frac{r(r-1)\ldots(r-k+1)}{k!}\]
podemos reescribir el polinomio de Taylor como 
\[P_{n,0}(x)=\sum_{k=0}^n(^r_k)c^{r-k}x^{k}\]
Esto parece  una generalizaci\'on inmediata de la f\'ormula del binomio para 
exponentes $r$ arbitrarios (no necesariamente enteros), pero debemos notar que 
cuando  $r$ no es entero, el coeficiente en\'esimo $(^r_n)$ {\it no se anula} para $n>r$, 
siendo $P_{n,0}(x)$ una expresi\'on {\it aproximada} de $f(x)$ 
en las cercan\'{\i}as de $x=0$,  que mejora al aumentar $n$. 

Para $f(x)=(1+x)^r$, ($c=1$),  obtenemos por ejemplo,
\[P_{n,0}(x)=1+rx+\frac{r(r-1)}{2!}x^2+\ldots+\frac{r(r-1)\ldots(r-n+1)}{n!}x^n=
\sum_{k=0}^n(^r_k)x^k\]
En particular,  el polinomio de Taylor para 
\[f(x)=\sqrt{1+x}\]
que corresponde a $c=1$, $r=1/2$,  es 
\[P_{n,0}(x)=1+\frac{1}{2}x-\frac{1}{8}x^2+\frac{1}{16}x^3+\ldots+\frac{(-1)^{n-1}(2n)!}
{2^{2n}(n!)^2(2n-1)}x^n\]
donde hemos utilizado la relaci\'on $\half(\half-1)(\half-2)\ldots(\half-(n-1))=
\frac{(-1)^{(n-1)}(2n)!}{2^{2n}n!(2n-1)}$ ($n\geq 1$) 
que el lector puede probar facilmente. 
\hfill\break\break
9) 
\[f(x)=\{^{\exp(-1/x^2),\;\;x\neq 0}_{0,\;\;\;\;\;\;\;\;\;\;\;\;\;\;\;\;\;\;x=0}\]
Hab\'{\i}amos visto en la secci\'on anterior que $f(x)$ es derivable a cualquier orden 
en $x=0$, siendo todas las derivadas nulas: $f(0)=0$, $f'(0)=0$ y en general, 
$f^{(k)}(0)=0$. Por lo tanto, para $x_0=0$, el polinomio de Taylor resultante es nulo
$\forall n$: 
\[ P_{n,0}(x)=0\]
Esto significa que la mejor aproximaci\'on polinomial a esta funci\'on 
%no podemos aproximar  esta funci\'on por un polinomio 
en las proximidades de $x=0$ es un polinomio nulo, 
 ya que $f(x)$ posee un m\'{\i}nimo ``muy plano'' en $x=0$. 
\newpage
\subsection{Estimaci\'on del error}
Podemos escribir en general 
\ben f(x)&=&P_{n,x_0}(x)+R_{n,x_0}(x)\nonumber\\
&=&f(x_0)+f'(x_0)(x-x_0)+\frac{f''(x_0)}{2!}(x-x_0)^2+\ldots+
\frac{f^{(n)}(x_0)}{n!}(x-x_0)^n+R_{n,x_0}(x)\een
donde $R_{n,x_0}(x)=f(x)-P_{n,x_0}(x)$ representa el {\it error} 
cometido al aproximar $f(x)$ por $P_{n,x_0}(x)$. 
Hemos visto ya que $\lim_{x\rightarrow 0}R_{n,x_0}(x)/(x-x_0)^n=0$. 
Daremos aqu\'{\i} una expresi\'on para $R_{n,x_0}(x)$ v\'alida 
en la mayor parte de los casos de inter\'es. 

{\it Teorema:} Supongamos que $f,f',\ldots f^{(n+1)}$ est\'an definidas en un intervalo 
$[x_0,x]$. Entonces 
\be R_{n,x_0}(x)=\frac{f^{(n+1)}(c)}{(n+1)!}(x-x_0)^{n+1},\;\;\;\;{\rm para\;alg\'un\;} 
c\in[x_0,x]\label{error}\ee
%El error es entonces como el t\'ermino $n+1$-\'esimo pero con la derivada evaluada en 
%algun punto intermedio $c$. 
Demostraci\'on: Para un n\'umero $t\in[x_0,x]$, podemos escribir 
\ben S(t)\equiv R_{n,t}(x)&=&f(x)-P_{n,t}(x)\nonumber\\
&=&f(x)-[f(t)+f'(t)(x-t)+\frac{f''(t)}{2!}(x-t)^2+\ldots+\frac{f^{(n)}(t)}{n!}(x-t)^n]\een
%donde hemos denotado por conveniencia 
Derivando ambos miembros de la anterior ecuaci\'on {\it respecto de} $t$,
se obtiene 
\ben S'(t)&=&-f'(t)+f'(t)\!-\!f''(t)(x-t)+f''(t)(x-t)\!-\!\frac{f'''(t)}{2!}(x-t)^2+\ldots
+\frac{f^{(n)}(t)}{(n-1)!}(x-t)^{n-1}\!-\!\frac{f^{(n+1)}(t)}{n!}(x-t)^{n}\nonumber\\
&=&-\frac{f^{(n+1)}(t)}{n!}(x-t)^{n}\een 
dado que todos los t\'erminos precedentes se cancelan. 
%Por consiguiente 
%\[S'(t)=-\frac{f^{(n+1)}(t)}{n!}(x-t)^{n}\]
Aplicando el teorema del valor medio de Cauchy 
%para funciones $g_1(t)$, $g_2(t)$ 
%derivables para  $t\in [t_0,t]$, 
\[\frac{g_1(t)-g_1(t_0)}{g_2(t)-g_2(t_0)}=\frac{g_1'(c)}{g_2'(c)},\;\;\;\;\;\;{\rm para\;alg\'un\;}
c\in[t_0,t]\]
a $g_1(t)=S(t)=R_{n,t}(x)$, $g_2(t)=(x-t)^{n+1}$, para $t=x$ y $t_0=x_0$, 
obtenemos $g_1(x)=R_{n,x}(x)=0$ (pues el error en $x$ para $x_0=x$ es $0$), 
 $g_1(x_0)=R_{n,x_0}(x)$, $g_2(x)=0$,  
$g_2(x_0)=(x-x_0)^{n+1}$. Por lo tanto, 
\[\frac{R_{n,x_0}(x)}{(x-x_0)^{n+1}}=\frac{-f^{(n+1)}(c)(x-c)^n/n!}
{-(n+1)(x-c)^n}=\frac{f^{(n+1)}(c)}{(n+1)!}\]
de donde se obtiene el resultado (\ref{error}). En el caso $x<x_0$,  $c\in[x,x_0]$. 

Si bi\'en en general no vamos a conocer el valor de $c$, la  f\'ormula anterior 
permite {\it acotar} el error  facilmente. 
Por ejemplo, si $|f^{(n+1)}(c)|\leq M_{n+1}$ $\forall c\in [x_0,x]$, entonces 
\[|R_{n,x_0}(x)|\leq M_{n+1}|x-x_0|^{n+1}/(n+1)!\]
 verific\'andose que $\lim\limits_{x\rightarrow x_0}|R_{n,x_0}(x)/(x-x_0)^n|\leq 
\lim\limits_{x\rightarrow x_0}M_{n+1}|x-x_0|/(n+1)!=0$. 

Adem\'as, si $f^{(n+1)}(c)$ permanece 
acotada para $n$ arbitrariamente grande $\forall c\in[x_0,x]$, 
es decir, si $M_{n+1}\leq M$ $\forall n$, entonces 
\[\lim_{n\rightarrow\infty}R_{n,x_0}(x)=0\]
para cualquier $x$. 
Esto implica que podemos hacer el  error arbitrariamente peque\~no 
aumentando el grado del polinomio, y escribir $f(x)=\lim_{n\rightarrow\infty}P_{n,x_0}(x)$, 
como veremos m\'as adelante. Para probar lo anterior, basta con demostrar que 
para cualquier n\'umero $a$, 
\[\lim_{n\rightarrow\infty} a^n/n!=0\]
Esto es obvio ya que para $n>n_0>2|a|$, podemos escribir 
\[\frac{|a|^n}{n!}=\frac{|a|^{n_0}}{n_0!}\frac{|a|^{n-n_0}}{(n_0+1)(n_0+2)\ldots n}
\leq\frac{|a|^{n_0}}{n_0!}\frac{|a|^{n-n_0}}{2|a|.2|a|\ldots 2|a|}
\leq \frac{|a|^{n_0}}{n_0!}\frac{|a|^{n-n_0}}{(2|a|)^{n-n_0}}=
\frac{|a|^{n_0}}{n_0!}\frac{1}{2^{n-n_0}}\]
y $\lim\limits_{n\rightarrow\infty}1/2^{n-n_0}=0$. Por lo tanto 
\[\lim_{n\rightarrow\infty}|R_{n,x_0}(x)|\leq
\lim_{n\rightarrow\infty}\frac{M_{n+1}|x-x_0|^{n+1}}{(n+1)!}
\leq M \lim_{n\rightarrow\infty}\frac{|x-x_0|^{n+1}}{(n+1)!}=0\]
\newpage
Ejemplos: \hfill\break\break
1) $f(x)=\sen(x)$. Como $|\sen(x)|\leq 1$, $|\cos(x)|\leq 1$, 
$\Rightarrow |f^{(n)}(c)|\leq 1$ $\forall c,n$ y 
\[|R_{n,0}(x)|=|\frac{f^{(n+2)}(c)}{(n+2)!}x^{n+2}|\leq\frac{|x|^{n+2}}{(n+2)!},\;\;\;\; 
n\;{\rm impar}\]
dado que s\'olo aparecen potencias impares en el polinomio de Taylor. 
 En este caso $\lim_{n\rightarrow\infty}R_{n,0}(x)=0$ $\forall x$, de modo 
que podemos aproximar  $\sen(x)$ por  $P_{n,0}(x)$ 
con un grado de precisi\'on arbitrario haciendo 
$n$ suficientemente grande. 

Por ejemplo,  si queremos evaluar $\sen(x)$ con un 
error menor que $\varepsilon$,   debemos emplear un $n$ (impar) tal que 
\[|R_{n,0}(x)|\leq\frac{|x|^{n+2}}{(n+2)!}<\varepsilon\]
Por ejemplo, para calcular $\sen(0.1)$ con un error menor que $10^{-3}$, 
basta con tomar $n\geq 1$, pues $|R_{1,0}(0.1)|\leq 0.1^3/3!\leq 10^{-3}$. 
Es decir, $\sen(0.1)\approx P_{1,0}(0.1)=0.1$, con un error menor que $10^{-3}$. 
 Si deseamos un \break error menor que $10^{-7}$, entonces basta con tomar $n\geq 3$, 
ya que $|R_{3,0}(0.1)|\leq 0.1^5/5!\leq 10^{-7}$. Es decir, 
$\sen(0.1)\approx P_{3,0}(0.1)\approx 0.1-0.1^3/3!$, con un error menor que $10^{-7}$. 
\hfill\break\break
2)  $f(x)=\cos(x)$. Nuevamente, como $|\cos(x)|\leq 1$, $|\sen(x)|\leq 1$, 
$\Rightarrow |f^{(n)}(c)|\leq 1$ $\forall c,n$ y  
\[|R_{n,0}(x)|=|\frac{f^{(n+2)}(c)}{(n+2)!}x^{n+2}|\leq\frac{|x|^{n+2}}{(n+2)!},\;\;\;\;
n\;{\rm par}\]
dado que s\'olo aparecen potencias pares en el desarrollo. Nuevamente, 
 $\lim\limits_{n\rightarrow\infty}R_{n,0}(x)=0$ $\forall x$. 
Si queremos evaluar $\cos(x)$ con un 
error menor que $\varepsilon$,   debemos emplear un $n$ (par) tal que 
\[|R_{n,0}(x)|\leq\frac{|x|^{n+2}}{(n+2)!}<\varepsilon\]
Por ejemplo, para calcular $\cos(0.1)$ con un error menor que $10^{-5}$, 
basta con tomar $n\geq 2$, ya que $|R_{2,0}(0.1)|\leq 0.1^4/4!\leq 10^{-5}$. 
Es decir, $\cos(0.1)\approx P_{2,0}(0.1)=1-0.1^2/2$, con un error menor que $10^{-5}$.
\hfill\break\break
3) $f(x)=\exp(x)$. 
 En este caso, $f^{(n)}(x)=\exp(x)$. 
Por lo tanto, para $x_0=0$ y $x>0$, $\exp(c)\leq \exp(x)$ para $c\in[0,x]$, pues $\exp(x)$ es 
una funci\'on creciente, y 
\[|R_{n,0}(x)|\leq\frac{\exp(x)\;|x|^{n+1}}{(n+1)!}\]
Por ejemplo, si queremos calcular $e=\exp(1)$ con un error menor que $\varepsilon$, 
debemos utilizar un $n$ tal que 
\[|R_{n,0}(1)|\leq \frac{e}{(n+1)!}\leq \frac{3}{(n+1)!}< \varepsilon\]
dado que  $|e|<3$. Por ejemplo, para $\varepsilon=10^{-6}$,  basta con tomar 
$n\geq 9$. Es decir, $e=\sum_{k=0}^9\frac{1}{k!}$, con un error menor que $10^{-6}$. 
Para $\varepsilon=10^{-12}$, basta con tomar $n\geq 15$ y para $\varepsilon=10^{-10000}$, 
$n\geq 3249$. \hfill\break
Si $x<0$, $\Rightarrow \exp(c)\leq\exp(0)=1$ para $c\in[x,0]$ y por lo tanto, 
\[|R_{n,0}(x)|\leq \frac{|x|^{n+1}}{(n+1)!}\]
En ambos casos, $\lim\limits_{x\rightarrow \infty}R_{n,0}(x)=0$ $\forall x$, 
 por lo que podemos escribir $\exp(x)=\lim\limits_{n\rightarrow\infty}P_{n,0}(x)$. 
\newpage
4) $f(x)=\ln(1+x)$. En este caso, $f^{(n+1)}(x)=(-)^{n}n!/(1+x)^{n+1}$.  
Por lo tanto, para $x_0=0$ y $x>0$,  $|f^{(n+1)}(c)|\leq n!$ para $c\in[0,x]$ y 
\[|R_{n,0}(x)|\leq \frac{n!|x|^{n+1}}{(n+1)!}=\frac{|x|^{n+1}}{n+1}\]
Pero si $-1<x<0$, puede probarse que $|f^{(n+1)}(c)|\leq n!/(1+x)$ para $c\in[x,0]$ y 
\[|R_{n,0}(x)|\leq \frac{n!|x|^{n+1}}{(n+1)!(1+x)}=\frac{|x|^{n+1}}{(n+1)(1+x)}\]
Notemos que $\lim\limits_{n\rightarrow \infty}R_{n,0}(x)=0$ {\it s\'olo si} $-1<x\leq 1$, 
de modo que podemos utilizar $P_{n,0}(x)$ para aproximar $\ln(1+x)$ 
con un grado de precisi\'on arbitrario {\it s\'olo en este intervalo}. 
La raz\'on es que cuando $n$ crece, 
$f^{(n+1)}(c)$ toma en este caso valores arbitrariamente grandes,  y el error 
tiende a $0$ s\'olo en alg\'un intervalo alrededor de $x_0=0$. 

Por ejemplo, para evaluar $\ln(1+x)$, $x>0$,  con un error menor que $\varepsilon$, 
debemos utilizar un $n$ tal que 
\[ |R_{n,0}(x)|\leq \frac{|x|^{n+1}}{n+1}<\varepsilon\]
pero  $|x|^{n+1}/(n+1)$ decrece para $n$ grande s\'olo si $|x|\leq 1$. 
Para evaluar $\ln(1.1)$ con un error menor que $\varepsilon=10^{-3}$, 
basta con tomar $n\geq 2$, pues $R_{2,0}(0.1)=0.1^3/3<10^{-3}$. 
Pero para evaluar $\ln(2)=\ln(1+1)$ con el mismo error, se necesita 
$n\geq 1000$, pues $R_{n,0}(1)=1/(n+1)$. 
En este caso el error decrece muy lentamente con $n$. 
\hfill\break\break
5) 
\[f(x)=\{^{\exp(-1/x^2),\;\;x\neq 0}_{0,\;\;\;\;\;\;\;\;\;\;\;\;\;\;\;\;\;\;x=0}\]
Hemos visto que $f^{(k)}(0)=0$ $\forall k\geq 0$, por lo que $P_{n,0}(x)=0$ $\forall x,n$. 
Por lo tanto, para $x_0=0$, %el error $R_{n,0}(x)$ es pues 
%{\it igual a la funci\'on misma} $\forall x,n$, 
\[R_{n,0}(x)=f(x)-P_{n,0}(x)=f(x)-0=f(x),\;\;\;\forall x,n\]
Es decir, el error es igual a la funci\'on misma. 
Esto tambi\'en implica que $\lim\limits_{n\rightarrow \infty}R_{n,0}(x)=f(x)\neq 0$ $\forall x\neq 0$.  
 No podemos entonces utilizar $P_{n,0}(x)$ para aproximar $f(x)$ con un grado de 
precisi\'on arbitrario {\it para ning\'un} $x\neq 0$, 
 a pesar de que $f(x)$ es derivable hasta cualquier orden en el origen. 

Que ha sucedido? 
La raz\'on  es que $\lim\limits_{x\rightarrow 0}f(x)/x^n=0$ $\forall n$, 
de modo que $|f(x)|$ es ``menor'' que cualquier potencia $|x^n|$ en las proximidades de $x=0$. 
Puede comprobarse que para $c\in[0,x]$, 
$f^{(n+1)}(c)$ y tambi\'en $f^{(n+1)}(c)/(n+1)!$ toman 
en este caso valores arbitrariamente grandes cuando 
$n$ crece, a\'un para $x$ arbitrariamente pr\'oximo al origen (con $x\neq 0$), 
tales que  $R_{n,0}(x)=f^{(n+1)}(c)x^{n+1}/(n+1)!=f(x)$  $\forall n,x$.  
 
\newpage
\subsection{Orden de Magnitud}
Consideremos una funci\'on $f(x)$ definida en un intervalo que contiene a $x=0$. 
Diremos que $f(x)$ es $O(x^n)$ ($f(x)$ es de orden $x^n$ en $x=0$) si 
 para $x$ suficientemente cercano a $0$,  existe una constante $C>0$ tal que 
\[|f(x)|\leq C |x|^n\]
Con esta definici\'on, si $f(x)$ es $O(x^n)$  y $k<n$,  entonces 
%existe eel l\'{\i}mite $entonces  existe $C$ tal que  
\[\lim_{x\rightarrow 0}|\frac{f(x)}{x^k}|\leq\lim_{x\rightarrow 0}C\frac{|x|^n}{|x|^k}
=\lim_{x\rightarrow 0}|x|^{n-k}=0,\;\;\;k<n\]
y por lo tanto $\lim_{x\rightarrow 0}f(x)/x^k=0$. 
%Obviamente, si $f(x)$ es 
%$O(x^n)$, $af(x)$, con $a$ una consante arbitraria,  es tambi\'en $O(x^n)$. 
Por ejemplo, en $x=0$, $f(x)=x^2$ es $O(x^2)$,  pues $|f(x)|\leq |x|^2$,  
 $f(x)=-2x^2$ tambi\'en es $O(x^2)$, pues 
$|f(x)|\leq 2|x|^2$,  y  $f(x)=|x|$ es $O(x)$.  
 Si $f(x)$ es $O(x^n)$ en $x=0$, es tambi\'en $O(x^m)$ con $m<n$, pues 
$|x|^n=|x|^m|x|^{n-m}\leq |x|^m$ si $m<n$ y $|x|\leq 1$. 

Si $f(x)$ es $O(x^n)$ y $g(x)$ es $O(x^m)$, con $m<n$, entonces 
$f(x)+g(x)$ es $O(x^m)$, pues 
\[|f(x)+g(x)|\leq |f(x)|+|g(x)|\leq C_f|x|^n+C_g|x|^m\leq (C_f+C_g)|x|^m\;\;{\rm si\;} 
|x|\leq 1\]
Obviamente, $f(x)-g(x)$ es tambi\'en $O(x^m)$. Por ejemplo, 
$f(x)=2x^2+5x^3$ es $O(x^2)$ y  $f(x)=1-x$ es $O(x^0)=O(1)$. 
Adem\'as, $f(x)g(x)$ es $O(x^{n+m})$, pues $|f(x)g(x)|\leq |f(x)||g(x)||\leq C_f C_g|x|^{n+m}$. 

  En general, $f(x)$ es $O((x-x_0)^n)$ en $x=x_0$ si existe $C>0$ tal que 
\[|f(x)|\leq C|x-x_0|^n\]
para $x$ suficientemente cercano a $x_0$. As\'{\i},  
$f(x)=3(x-2)^2$ es $O((x-2)^2)$ en $x=2$, y\hfill\break 
$f(x)=-2(x+3)^2+5(x+3)$ es $O(x+3)$ en $x=-3$. 

De esta forma,
si  $f^{n+1}(x)$ est\'a definida en un intervalo en torno a $x_0$, 
el error $R_{n,x_0}(x)=f(x)-P_{n,x_0}(x)$ 
es $O((x-x_0)^{n+1})$ en $x=x_0$ (ecuaci\'on \ref{error}). 
 Se suele escribir entonces  
\[ f(x)=f(x_0)+f'(x_0)(x-x_0)+\frac{f''(x_0)}{2!}(x-x_0)^2+\ldots+
\frac{f^{(n)}(x_0)}{n!}(x-x_0)^n+O(x^{n+1})\]
Por ejemplo, para $x$ cercano a $0$, puede escribirse 
\[\sen(x)=x-x^3/3!+O(x^5)\] 
\[\cos(x)=1-x^2/2+O(x^4)\]
\[\exp(x)=1+x+O(x^2)\]
\[\ln(1+x)=x-x^2/2+O(x^3)\]
para los desarrollos con dos t\'erminos no nulos. 
Tambi\'en podemos escribir $\sen(x)=x+O(x^3)$  o $\sen(x)=0+O(x)=O(x)$ 
para los desarrollos con $n=1$ y $n=0$.  
\hfill\break\break
Ejemplo: Hallar \[\lim_{x\rightarrow 0}\frac{\sen(x)-x}{x^3}\]
Si bien puede emplearse el m\'etodo de L'Hopital, es normalmente 
m\'as c\'omodo emplear una aproximaci\'on basada en alg\'un polinomio de Taylor. 
Utilizando $\sen(x)=x-x^3/3!+O(x^5)$, obtenemos 
\[\lim_{x\rightarrow 0}\frac{\sen(x)-x}{x^3}=
\lim_{x\rightarrow 0}\frac{x-x^3/3!+O(x^5)-x}{x^3}=
\lim_{x\rightarrow 0}\frac{-x^3/3!+O(x^5)}{x^3}=-\frac{1}{3!}=-\frac{1}{6}\]
pues el t\'ermino restante $R_{3,0}(x)=O(x^5)$ satisface  
$\lim_{x\rightarrow 0}R_{3,0}(x)/x^3=0$. 
\newpage
{\it Teorema de Unicidad}: 
Consideremos una funci\'on $f(x)$ tal que $f,f',f'',\ldots,f^{(n+1)}$ estan definidas 
 en un intervalo que contenga al origen. 
Supongamos que existe un polinomio $Q_n(x)$ de grado $n$ tal que 
\be f(x)=Q_n(x)+O(x^{n+1})\label{Qn}\ee 
es decir, que la diferencia $R_Q(x)=f(x)-Q_n(x)$ es $O(x^{n+1})$ 
en $x=0$.  Entonces 
\[Q_n(x)=P_{n,0}(x)\]
es decir, $Q_n(x)$ es {\it es el polinomio de Taylor} alrededor del origen de grado $n$. 
\hfill\break  
Demostraci\'on: Por hip\'otesis,  
\[ f(x)=Q_n(x)+R_Q(x)\]
donde  $R_Q(x)$ es $O(x^{n+1})$. Por otro lado, sabemos que 
\[ f(x)=P_{n,0}(x)+R_{n,0}(x)\]
donde  $R_{n,0}(x)$ es tambi\'en $O(x^{n+1})$. Restando, 
\[Q_n(x)-P_{n,0}(x)=R_{n,0}(x)-R_Q(x)\]
donde $R_{n,0}(x)-R_Q(x)$ es tambi\'en $O(x^{n+1})$. Por lo tanto, 
\be \lim_{x\rightarrow 0}\frac{Q_n(x)-P_{n,0}(x)}{x^k}=
\lim_{x\rightarrow 0}\frac{R_{n,0}(x)-R_Q(x)}{x^k}=0,\;\;\;\;0\leq k\leq n\label{ka}\ee
Como $Q_n(x)-P_{n,0}(x)$ es  un polinomio de grado $n$, podemos escribir  
\[Q_n(x)-P_{n,0}(x)=b_0+b_1x+b_2x^2+\ldots b_nx^n\] 
Por lo tanto, para $k=0$, la ec.\ (\ref{ka})  implica 
\[ \lim_{x\rightarrow 0}Q_n(x)-P_{n,0}(x)=b_0=0\] 
Analogamente, sabiendo ahora que $b_0=0$, para $k=1$ la ec.\ (\ref{ka}) implica 
 \[ \lim_{x\rightarrow 0}\frac{Q_n(x)-P_{n,0}(x)}{x}=
\lim_{x\rightarrow 0}b_1+b_2x+\ldots b_nx^{n-1}=b_1=0\] 
En general, sabiendo que $b_0=\ldots b_{k-1}=0$, la ec.\ (\ref{ka}) implica
 \[ \lim_{x\rightarrow 0}\frac{Q_n(x)-P_{n,0}(x)}{x^k}=
\lim_{x\rightarrow 0}b_k+b_{k+1}x+\ldots b_nx^{n-k}=b_k=0\] 
Por lo tanto $Q_n(x)-P_{n,0}(x)=0$. Este resultado es por dem\'as obvio, 
ya que el orden de un polinomio en el origen es igual al t\'ermino de menor 
grado que aparece  en su desarrollo en potencias de $x$, de modo que 
si un polinomio de grado $n$ debe ser $O(x^{n+1})$, s\'olo puede ser el polinomio nulo. 

Resumiendo,  si $Q_n(x)=b_0+b_1x+\ldots+b_nx^n$, 
 $\Rightarrow b_k=f^{(k)}(0)/k!$. 
Obviamente, lo mismo rige  alrededor de un punto cualquiera $x_0$: Si 
$f(x)=Q_n(x)+O((x-x_0)^{n+1})$ en $x=x_0$, $\Rightarrow Q_n(x)=P_{n,x_0}(x)$.  
Este teorema es muy \'util para hallar polinomios de Taylor de funciones compuestas 
o de producto de funciones. 
\hfill\break\break
Ejemplos: \hfill\break
1) Hallar el polinomio de Taylor de orden $4$ para $f(x)=\exp(x^2)$. 
\hfill\break
Llamando $g(u)=\exp(u)$, tenemos $f(x)=g(x^2)$. Como 
$g(u)=1+u+u^2/2!+O(u^3)$, entonces 
\[f(x)=g(x^2)=1+x^2+x^4/2!+O(x^6)\]
Puede verificar el lector que $1+x^2+x^4/2!=P_{4,0}(x)=\sum_{k=0}^4f^{k}(0)x^k/k!$, 
aunque el  c\'alculo directo es mucho m\'as engorroso.   
 \hfill\break\break
2) Hallar el polinomio de Taylor de orden $2$ para $f(x)=\sen^2(x)$ ($\sen^2(x)=[\sen(x)]^2$). 
\hfill\break
Como $\sen(x)=x+O(x^3)$, es decir, $\sen(x)=x+R_1(x)$, donde  $R_{1}(x)$ es $O(x^3)$, 
tenemos 
\[\sen^2(x)=(x+R_1(x))^2=x^2+(2xR_1(x)+(R_1(x))^2)=x^2+O(x^4)\]
dado que $2x R_1(x)+(R_1(x))^2$ es $O(x^4)$. Puede comprobar el lector que 
$x^2=P_{2,0}(x)=\sum_{k=0}^2f^{(k)}(0)x^k/k!$. 
Notemos que s\'olo puden aparecer potencias pares pues 
$\sen^2(-x)=\sen^2(x)$. 

\end{document}
7) Graficar $f(x)=\{^{-x Log[x]},\;\;0<x\leq 1}_{0,\;\;\;\;\;\;\;\;x=0}$, y mostrar que es 
continua (hacia la derecha) en $x=0$ (es decir, $\lim_{x\rightarrow 0^+}f(x)=0=f(0)$. 
\hfill\break
Si bien el dominio natural de $-x Log[x]$ es $x>0$, 
esta funci\'on tiene particular importancia en cuestiones 
estad\'{\i}sticas, en cuyo caso $x$ representa una probabilidad $p$, con $0\leq p\leq 1$ 
($S=\sum_if(p_i)$ es la {\it entrop\'{\i}a} de una distribuci\'on de probabilidad).  

Notemos en primer lugar que $f(x)\geq 0$ si $x\in[0,1]$,  pues $\ln(x)\leq 0$ en este intervalo. 
Adem\'as $f(1)=-1\ln(1)=0$. \hfill\break
Para $x\rightarrow 0^+$, $x\rightarrow 0$ y $\ln(x)\rightarrow -\infty$. El l\'mite del 
producto es sin embargo nulo:  
\[\lim_{x\rightarrow 0^+}f(x)=\lim_{x\rightarrow 0^+}-x Log[x]=\lim_{w\rightarrow+\infty}
e^{-w}w=0,\;\;\;\;w=-Log[x]\;\;\;(x=e^{-w})\]
Adem\'as,
\[f'(x)=-1-\ln(x),\;\;\;x>0,\;\;\;\;\;\;f''(x)=-1/x,\;\;x>0\]
lo que implica $\lim_{x\rightarrow 0^+}f'(x)=\infty$, 
$\lim_{x\rightarrow 0^+}f''(x)=+\infty$. Notemos que la derivada (por derecha) 
en $x=0$ no existe, 
\[\lim_{h\rightarrow 0^+}\frac{f(x+h)-f(0)}{h}=\lim_{h\rightarrow 0^+}\frac{-h\ln h}{h}
=\lim_{h\rightarrow 0^+}-\ln h=+\infty\]
Para $x\rightarrow 0^+$, tenemos una situaci\'on similar al caso  $g(x)=\sqrt{x}$, 
donde $g(0)=0$, pero $g'(x)=\frac{1}{2\sqrt{x}}\rightarrow+\infty$ para $x\rightarrow 0^+$. 

La ecuaci\'on $f'(x)=0$ conduce a $-1-\ln(x)=0$, o sea, $\ln(x)=-1$, cuyo resultado es 
$x=\exp(-1)=e^{-1}$, con $f(e^{-1})=-e^{-1}\ln(e^{-1})=e^{-1}$. Este punto es un 
{\it m\'aximo absoluto} para $x\in[0,1]$, ya que $f'(x)>0$ para $0<x<e^{-1}$ y 
$f'(x)<0$ para $x>e^{-1}$. Por otro lado, $f''(x)<0$ 
para $x>0$, de modo que es c\'oncava hacia abajo en todo su dominio. 
El m\'{\i}nimo absoluto de $f(x)$ en $[0,1]$ es tomado en los extremos, 
donde $f(x)=0$  ($x=0$ o $x=1$).

\end{document}


	
\nopagebreak
\begin{thebibliography}{999}

\bibitem{THTT.97}  
               M.T.\ Tuominen, J.M.\ Hergenrother, T.S.\ Tighe and M.\ Tinkham, 
               Phys.\ Rev.\ Lett.\ {\bf 69} (1992) 1997; Phys.\ Rev.\ {\bf B47} (1993) 11599.
\bibitem{L.93} P.\ Lafarge et al., Phys.\ Rev.\ Lett.\ {\bf 70} (1993) 994.
\bibitem{RBT.95}
               D.C.\ Ralph, C.T.\ Black and M.\ Tinkham, 
               Phys.\ Rev.\ Lett.\  {\bf 74}  (1995) 3241.
\bibitem{BRT.96}
               C.T.\ Black, D.C.\ Ralph and M.\ Tinkham, 
               Phys.\ Rev.\ Lett.\ {\bf 76} (1996) 688.
\bibitem{GZ.94} 
                D.S.\ Golubev and A.D.\ Zaikin, Phys.\ Lett.\ {\bf A195}
                 (1994) 380.   
\bibitem{DZ.96}
                J.\ von Delft, A.\ Zaikin, D.\ Golubev and W.\ Tichy, 
                Phys.\ Rev.\ Lett.\ {\bf 77} (1996) 3189. 

\bibitem{Ch.62} B.S.\ Chandrasekhar, Appl.\ Phys.\ Lett.\ {\bf 1}, 7 (1962).
\bibitem{Glog.62} A.M.\ Glogston\ Phys.\ Rev.\ Lett.\  {\bf 9}, 266 (1962).
48274502

\bibitem{SA.96}
                R.\ Smith and  V.\ Ambegaokar, 
                Phys.\ Rev.\ Lett.\ {\bf 77}  (1996) 4962. 
\bibitem{ML.97}             
                K.A.\ Matveev and A.I.\ Larkin, 
                Phys.\ Rev.\ Lett.\ {\bf 78}  (1997) 3749.
\bibitem{BFV.97}
                R.\ Balian, H.\ Flocard and M.\ Veneroni, 
                (nucl-th/9706041). 
\bibitem{BFV.98}
                R.\ Balian, H.\ Flocard and M.\ Veneroni, 
                (cond-mat/9802006). 
\bibitem{MSD.72} 
                B.\ M\"uhlschlegel, D.J.\ Scalapino and R.\ Denton,
                Phys.\ Rev.\  {\bf B6} (1972) 1767. 
\bibitem{RS.80}              
                J.L.\ Egido, P.\ Ring, S.\ Iwasaki and  H.J.\ Mang, 
                Phys.\ Lett.\  {\bf B154} (1985) 1. 
                 P. Ring, P. Schuck, 
               ``The Nuclear Many-Body Problem'', Springer, NY, 1980. 
\bibitem{PBB.91}
                G.\ Puddu, P.F.\ Bortignon and R.A.\ Broglia, 
                Ann.\ of Phys.\ (N.Y.) {\bf 206}  (1991) 409;
                B. Lauritzen et al,  Phys.\ Lett. {\bf B246} (1990) 329;
                G.\ Puddu, Phys.\ Rev.\ {\bf B45} (1992) 9882. 
\bibitem{AA.97}
               H.\ Attias  and Y.\ Alhassid,
               Nucl.\ Phys.\ {\bf A625} (1997) 363.
\bibitem{RC.97}
              R.\ Rossignoli, N.\ Canosa, Phys.\ Lett.\ {\bf B394} 
              (1997) 242; Phys.\ Rev.\ {\bf C56} (1997) 791.
\bibitem{RCR.98}
              R.\  Rossignoli, N.\  Canosa and P.\  Ring, 
              Phys.\ Rev.\ Lett.\ {\bf 80} (1998) 1853. 
              R.\  Rossignoli, N.\  Canosa and P.\  Ring, 
              Ann.\ of Phys.\ (N.Y.) (in press).  
\end{thebibliography}

%%%%%%%%%%%%%%%%%%%%%%%%%%%%%%%%%%%%%%%%%%%%%%%%%%%%%%%%%%%%%%%%%%%%%%%%%%%%%%%%%

\begin{figure}
\caption{NP projected BCS gap vs.\ temperature for uniform  spacing $\varepsilon$ and 
different values of the size parameter $\delta\equiv\varepsilon/\Delta_0$, with $\Delta_0$ 
and $T_c$ the bulk BCS $T=0$ gap and critical temperature. Top left: Even (E, $\nu=1$) 
and grand canonical (GC, $\nu=0$) results for  $N_0=\Omega$ ($\mu=0$).  
Right: Even and GC results for $\mu=\varepsilon/2$ (see text). Same sizes as left panel, 
except for case $f$. Bottom left: Results in the odd case ($\nu=-1$) for 
$\mu=\varepsilon/2$ (O, $N_0=\Omega+1$) and $\mu=0$ (O'). 
Right: Comparison between $\Delta$ and the BCS value  of 
$\Delta_P$.  Curves $b$, $c$, $d$ correspond to the 
even case $\nu=1$, $\mu=0$, $b'$, $c'$ to the same sizes in the 
odd case $\nu=-1$, $\mu=\varepsilon/2$ ($\Delta=\Delta_P=0$ in odd case $d$).} 
\label{f1} 
\end{figure}

\begin{figure}
\caption{Top: CSPA and SPA values of $\Delta_P$, eq.\ (\protect\ref{dp}), for the 
even case $\nu=1$, $N_0=\Omega$ (left) and odd case  $\nu=-1$, $N_0=\Omega+1$ (right), and 
 sizes $\delta=0.03$ ($a$), 0.3 ($b$), 0.9 ($c$) and 2.8 ($d$). 
The bulk BCS gap (dotted line) and the {\it exact canonical result\,}  in case $d$ are also 
depicted. Bottom: Left: Comparison between the CSPA values of $\Delta_P$ in the even (E) 
and odd (O) cases. Curves for size $d$ depict the exact results. 
 Right: The free energy difference (\protect\ref{fd}) for $N=\Omega$.}  
\label{f2}
\end{figure}

\begin{figure}
\caption{Specific heat $C_{\rm\scriptscriptstyle V}=dE/dT$ scaled to the bulk BCS value at 
$T=T_c$ ($C_{\rm m}$),  according to NP projected BCS (top) and CSPA and SPA (bottom), 
for the even (left) and odd (right) systems,  and sizes $\delta=0.03$ ($a$), 0.3 ($b$) and 
0.9 ($c$). The dotted line in all panels depicts the bulk result (almost coincident 
with curves $a$ in the upper panels).} 
\label{f3}
\end{figure}

\end{document}

(* triangulo phi(t) *)

Show[Plot[Sqrt[0.4^2-x^2],{x,0.4,0.4/Sqrt[2.]},Axes->False,
    PlotRange->{{-0.5,1.5},{-0.5,1.5}}],Graphics[Line[{{0,0},{1,0}}]],
	Graphics[Line[{{0,0},{1,1}}]],
	Graphics[Line[{{1,0},{1,1}}]],
	Graphics[Text["l",{0.5,-0.15}]],
	Graphics[Text["h(t)",{1.15,0.5}]],
	Graphics[Text["\[Phi](t)",{0.25,0.1}]]]

tambien 
\!\(\(\nPlot[Exp[\(-1\)/x^2], {x, \(-5\), 5}, 
    PlotRange -> {{\(-3.9\), 3.9}, {\(-0.1\), 1.1}}, 
    AxesLabel -> {"\<x\>", \*"\"\<\!\(e\^\(\(-1\)/x\^2\)\)\>\""}]\)\)

\subsection{Polinomio de Taylor}
Estudiaremos primero como expresar un polinomio en t\'erminos de sus derivadas sucesivas 
en un punto. Consideremos un polinomio  
\[ f(x)=a_nx^n,\;\;\;a_n\neq 0, n\geq 0\]
Si $n=0$, entonces $f(x)=a_0=f(0)$. Si $n=1$, $f(x)=a_1 x=f'(0) x$, 
dado que $f'(x)=a_1=f'(0)$. 
Sea $P(x)$ un polinomio de grado $n$, 
\[ P(x)=\sum_{i=0}^na_ix^i=a_0+a_1x+a_2x^2+\ldots+a_nx^n,\;\;\;a_n\neq 0, n\geq 1\]
Podemos ver que 
\[ P(0)=a_0\]
Tambi\'en, dado que la derivada es $P'(x)=a_1+2 a_2x+3a_3x^2+\ldots na_nx^{n-1}$, 
\[ P'(0)=a_1=1!a_1\]
y dado que $P''(x)=2a_2+3.2 a_3 x+4.3 a_4 x^2+\ldots+n(n-1)a_nx^{n-2}$, 
\[ P''(0)=2a_2=2!a_2\]
En general, para $0\leq i\leq n$, 
la derivada i-\'esima $P^{(i)}(x)$ evaluada en $x=0$ ser\'a 
\[ P^{(i)}(0)=i! a_i\]
Si el lector necesita una demostraci\'on, podemos ver 
que la derivada $k-\'esima$ de $f_i(x)=x^i$ es 
\[f^{(k)}_i(x)=\frac{d^k x^i}{d x^k}=\left\{\begin{array}{ll}
i(i-1)\ldots(i-k+1)x^{i-k}={\textstyle\frac{i!}{(i-k)!}}x^{i-k},& i\leq k\\
0,& i>k\end{array}\.\]  
con $f^{(k)}_i(0)=0$ si $i<k$ o $i>k$ y $f^{(i)}_i(0)=i!x^0=i!$ para $k=i$. 
con lo que obtenemos el resultado anterior. 
Por lo tanto, 
\[a_0=P(0),\;\;\a_1=p'(0), \;\;a_2=P'(0)/2!, \;\;\;a_i=P^{(i)}(0)/i!\]
y podemos escribir
\[ P(x)=P(0)+P'(0)x+P''(0)\] 

(Lavado de alfombra: Alberto 427-0801) 

