\documentclass{scrartcl}
\usepackage[a4paper,margin=2cm,footskip=1cm]{geometry}
\setkomafont{disposition}{\normalfont\bfseries}

%\documentclass{article}

\usepackage[table,xcdraw]{xcolor}
\usepackage{tikz}
\usetikzlibrary{angles,quotes}
\usetikzlibrary{babel}

\usepackage{booktabs}
\usepackage[utf8]{inputenc}
\usepackage[spanish, es-nodecimaldot]{babel}
\usepackage[per-mode=symbol]{siunitx}
\usepackage{graphicx}
\usepackage{subcaption}
\usepackage{caption}
\usepackage{mathtools}
\usepackage{amsmath}
\usepackage{slashed}
\usepackage{dsfont}
\usepackage{float}
\usepackage{multicol}
\usepackage{wrapfig}
\usepackage{lipsum}
\usepackage{textcomp}
\usepackage{gensymb}
\usepackage{longtable}
\usepackage{supertabular}
\usepackage{hhline}
\usepackage{enumerate}
\usepackage{multirow}
\usepackage{amssymb}
\usepackage{tabularx}
\usepackage{ragged2e}
\usepackage{rotating}
\usepackage{cancel}
\usepackage{physics}
\usepackage[framemethod=default]{mdframed}
\usepackage{csquotes}
%\usepackage[backend=biber, style=numeric, sorting=none]{biblatex}
\usepackage{qcircuit}
\usepackage{bm}

\renewcommand{\figurename}{Figura}
\renewcommand{\spanishtablename}{Tabla}
\newcommand{\inv}[1]{\frac{1}{#1}}
\newcommand{\uv}[1]{\hat{\mathbf{#1}}}
\newcommand{\uvs}[1]{\, \uv{#1}}

\newcommand{\realSet}{\mathbb{R}}
\newcommand{\complexSet}{\mathbb{C}}
\newcommand{\oref}{$\mathcal{O}$ }
\newcommand{\opref}{$\mathcal{O}'$ }
\newcommand{\oppref}{$\mathcal{O}''$ }

\def\residue{\mathop{\text{Res}}}

\setlength{\tabcolsep}{19pt}

\DeclareSIUnit\clight{\text{\ensuremath{c}}}
\DeclareSIUnit\MeV{\mega\electronvolt}
\DeclareSIUnit\GeV{\giga\electronvolt}
\DeclareSIUnit\MeVpc{\MeV\per\clight\squared}
\DeclareSIUnit\GeVpc{\GeV\per\clight\squared}

\newcommand{\sinc}{\text{sinc}}
\newcommand{\E}{\vb{E}}
\newcommand{\B}{\vb{B}}
\newcommand{\x}{\vb{x}}
\newcommand{\y}{\vb{y}}
\newcommand{\z}{\vb{z}}
\newcommand{\p}{\vb{p}}
\renewcommand{\k}{\vb{k}}
\newcommand{\Lag}{\mathcal{L}}
\newcommand{\Ham}{\mathcal{H}}

\newcommand{\tx}{\tilde{x}}

\renewcommand{\vb}[1]{\bm{#1}}

\renewcommand{\a}{\hat{a}}
\renewcommand{\b}{\hat{b}}
\renewcommand{\c}{\hat{c}}

\DeclareRobustCommand{\[}{\begin{equation}}
\DeclareRobustCommand{\]}{\end{equation}}
\mathtoolsset{showonlyrefs}

\allowdisplaybreaks

%\bibliography{bibliography}

%----------------------------------------------------------------------------------------
%	DOCUMENT INFORMATION
%----------------------------------------------------------------------------------------

\title{Teoría de la Información Cuántica}
\subtitle{Práctica 6 - Año 2020}
\author{\textsc{Beaucamp}, Jean Yves}
\date{}

\begin{document}

\maketitle

\section{Estados Coherentes}
\begin{enumerate}
    
    %-------------------------------------------------------------------------------------------------------
    %   Problema I.1
    %-------------------------------------------------------------------------------------------------------
    \item Sea
    \[ \ket{\alpha} = e^{-\abs{\alpha}^2 / 2} \sum_{n = 0}^\infty \frac{\alpha^n}{\sqrt{n!}} \ket{n} \]
    un estado coherente, donde $\ket{n} = (\a^\dagger)^n \ket{0} / \sqrt{n!}$, y $\a^\dagger$, $\a$ son los operadores de creación y aniquilación bosónicos respectivamente, satisfaciendo
    \[ \a \ket{n} = \sqrt{n} \ket{n - 1}, \quad \quad \a^\dagger \ket{n} = \sqrt{n + 1} \ket{n + 1}. \]
    
    \begin{enumerate}
        \item Podremos expresar a los estados coherentes alternativamente como
        \[ \ket{\alpha} = e^{-\abs{\alpha}^2 / 2} \sum_{n = 0}^\infty \frac{\alpha^n}{\sqrt{n!}} \ket{n} = e^{-\abs{\alpha}^2 / 2} \sum_{n = 0}^\infty \frac{(\alpha \a^\dagger)^n}{n!} \ket{0} = e^{-\abs{\alpha}^2 / 2} e^{\alpha \a^\dagger} \ket{0}. \]
        
        
        \item Sea
        \[ \hat T(\alpha) = \exp{ \alpha \a^\dagger - \alpha^* \a } = \exp{ -i\sqrt{2} \ \qty(\Re(\alpha) \ \hat{p} - \Im(\alpha) \ \hat{q})} \]
        un operador de traslación, donde
        \[ \hat{p} = \frac{\a - \a^\dagger}{i \sqrt{2}}, \quad \quad \hat{q} = \frac{\a + \a^\dagger}{\sqrt{2}} \]
        son los operadores de impulso y coordenada asociados. Luego, utilizando la identidad B.C.H. en la forma
        \[ e^{\hat A + \hat B} = e^{\hat A} e^{\hat B} e^{- \frac{\comm{\hat A}{\hat B}}{2}}, \]
        donde en este caso particular $\hat A = \alpha \a^\dagger$, $\hat B = -\alpha^* \a$, con $\comm{\hat A}{\hat B} = -\abs{\alpha}^2 \comm{\a^\dagger}{\a} = \abs{a}^2$, entonces podremos expresar a $\hat T(\alpha)$ como
        \[ \hat T(\alpha) = e^{-\frac{\abs{\alpha}^2}{2}} e^{\alpha \a^\dagger} e^{\alpha^* \a}. \]
        Luego, considerando que $e^{-\alpha^* \a} \ket{0} = \ket{0}$ (ya que solo contribuye el término de orden 0 de la expansión en serie, al no contener operadores de aniquilación),
        \[ \hat T(\alpha) \ket{0} = e^{-\frac{\abs{\alpha}^2}{2}} \sum_{n = 0}^\infty \frac{(\alpha \a^\dagger)^n}{n!} \ket{0} = \ket{\alpha}. \]
        
        Es fácil ver que $\hat T(\alpha)$ es un operador unitario, ya que
        \[ \hat{T}^\dagger(\alpha) = \exp{(\alpha \a^\dagger - \alpha^* \a)^\dagger} = \exp{-(\alpha \a^\dagger - \alpha^* \a)} = \hat T^{-1}(\alpha) = \hat T(-\alpha). \]
        
        
        \item Veremos a continuación que los estados coherentes resultan autoestados del operador de aniquilación. Evaluando directamente a partir de la expansión en serie,
        \begin{align}
            \a \ket{\alpha} = e^{-\frac{\abs{\alpha}^2}{2}} \sum_{n = 0}^\infty \frac{\alpha^n}{\sqrt{n!}} \a \ket{n} = e^{-\frac{\abs{\alpha}^2}{2}} \sum_{n = 0}^\infty \frac{\alpha^n}{\sqrt{n!}} \sqrt{n} \ket{n - 1} &= e^{-\frac{\abs{\alpha}^2}{2}} \sum_{n = 1}^\infty \frac{\alpha^n}{\sqrt{(n - 1)!}} \ket{n - 1} \\
                &= \alpha e^{-\frac{\abs{\alpha}^2}{2}} \sum_{n = 1}^\infty \frac{\alpha^{n - 1}}{\sqrt{(n - 1)!}} \ket{n - 1} \\
                &= \alpha e^{-\frac{\abs{\alpha}^2}{2}} \sum_{n = 0}^\infty \frac{\alpha^n}{\sqrt{n!}} \ket{n} \\
                &= \alpha \ket{\alpha}.
        \end{align}
        
        
        \item Podremos considerar a la cantidad $P(n) = \abs{\ip{n}{\alpha}}^2$ como una densidad de probabilidad, ya que
        \[ \ip{n}{\alpha} = e^{-\frac{\abs{\alpha}^2}{2}} \sum_{m = 0}^\infty \frac{\alpha^m}{\sqrt{m!}} \underbrace{\ip{n}{m}}_{\delta_{nm}} = e^{-\frac{\abs{\alpha}^2}{2}} \frac{\alpha^n}{\sqrt{n!}}. \]
        \[ \implies P(n) = \abs{\ip{n}{\alpha}}^2 = \frac{\displaystyle e^{-\abs{\alpha}^2} \qty(\abs{\alpha}^2)^n}{n!}, \]
        correspondiéndose a una distribución de Poisson de parámetros 
        \[ \mathbb{E}[P] = \mathbb{V}[P] = \abs{\alpha}^2 = \mel{\alpha}{\a^\dagger \a}{\alpha}. \]
        
        
        \item Para estados coherentes, los operadores de posición e impulso tendrán valores de expectación
        \begin{align}
            \mel{\alpha}{\hat p}{\alpha} = \mel{\alpha}{\frac{\a - \a^\dagger}{i \sqrt{2}}}{\alpha} = \inv{i \sqrt{2}} \left( \mel{\alpha}{\a}{\alpha} - \mel{\alpha}{\a^\dagger}{\alpha} \right) &= \inv{i \sqrt{2}} \left( \alpha \ip{\alpha}{\alpha} - \alpha^* \ip{\alpha}{\alpha} \right) \\
                &= \inv{i \sqrt{2}} \left( \alpha - \alpha^* \right) \\
                &= \frac{2}{\sqrt{2}} \Im(\alpha) = \sqrt{2} \Im(\alpha)
        \end{align}
        y
        \[ \mel{\alpha}{\hat q}{\alpha} = \mel{\alpha}{\frac{\a + \a^\dagger}{\sqrt{2}}}{\alpha} = \inv{\sqrt{2}} (\alpha + \alpha^*) = \frac{2}{\sqrt{2}} \Re(\alpha) = \sqrt{2} \Re(\alpha). \]
        
        
        \item Si $\hat H = \hbar \omega \a^\dagger \a$ (despreciando la energía de vacío $E_0 = \hbar \omega / 2$), entonces el operador de evolución temporal estará dado por $\hat U(t) = \exp{-i \hat H t}$. Luego, el estado coherente $\ket{\alpha(0)} = \ket{\alpha}$ evolucionado a un tiempo $t$ será
        \begin{align}
            \ket{\alpha(t)} = \hat U(t) \ket{\alpha(0)} = e^{-i \hat H t} \ket{\alpha} &= e^{-\frac{\abs{\alpha}^2}{2}} \sum_{n = 0}^\infty \frac{\alpha^n}{\sqrt{n!}} e^{-i\hbar \omega t \a^\dagger \a} \ket{n} \\
                &= e^{-\frac{\abs{\alpha}^2}{2}} \sum_{n = 0}^\infty \frac{\alpha^n}{\sqrt{n!}} e^{-i\hbar \omega t n} \ket{n} \\
                &= e^{-\frac{\abs{\alpha}^2}{2}} \sum_{n = 0}^\infty \frac{\qty(\alpha e^{-i\hbar \omega t})^n}{\sqrt{n!}} \ket{n} \\
                &= \ket{\alpha(t)},
        \end{align}
        donde $\displaystyle \alpha(t) = \alpha e^{-i\hbar \omega t}$.
        
        La evolución temporal de $\ket{\alpha}$ es, por lo tanto, la rotación del parámetro complejo $\alpha$ con velocidad angular $\omega$.
        
        
        \item Podemos verificar rápidamente que los estados coherentes no resultan ortogonales, viendo que
        \[ \ip{\beta}{\alpha} = \qty( \bra{0} e^{-\frac{\abs{\beta}^2}{2}} e^{\beta^* \a}) \qty(e^{-\frac{\abs{\alpha}^2}{2}} e^{\alpha \a^\dagger} \ket{0}) = e^{-\frac{\abs{\alpha}^2 + \abs{\beta}^2}{2}} \bra{0} e^{\beta^* \a} e^{\alpha \a^\dagger} \ket{0}. \]
        Luego, como
        \[ \comm{\beta^* \a}{\alpha \a^\dagger} = \beta^* \alpha \comm{\a}{\a^\dagger} = \beta^* \alpha, \]
        entonces
        \begin{align}
            e^{\beta^* \a} e^{\alpha \a^\dagger} = e^{\beta^* \a + \alpha \a^\dagger} e^{\frac{\comm{\beta^* \a}{\alpha \a^\dagger}}{2}} &= e^{\alpha \a^\dagger + \beta^* \a} e^{\frac{\beta^* \alpha}{2}} \\
                &= e^{\alpha \a^\dagger} e^{\beta^* \a} e^{-\frac{\comm{\alpha \a^\dagger}{\beta^* \a}}{2}} e^{\frac{\beta^* \alpha}{2}} \\
                &= e^{\alpha \a^\dagger} e^{\beta^* \a} e^{-\frac{(-\beta^* \alpha)}{2}} e^{\frac{\beta^* \alpha}{2}} \\
                &= e^{\alpha \a^\dagger} e^{\beta^* \a} e^{\beta^* \alpha},
        \end{align}
        y
        \begin{align}
            \ip{\beta}{\alpha} = e^{-\frac{\abs{\alpha}^2 + \abs{\beta}^2}{2}} e^{\beta^* \alpha} \ \underbrace{\bra{0} e^{\alpha \a^\dagger}}_{= \bra{0}} \underbrace{e^{\beta^* \a} \ket{0}}_{= \ket{0}} = e^{-\frac{\abs{\alpha}^2 + \abs{\beta}^2}{2} + \beta^* \alpha}.
        \end{align}
        
        
        \item Los estados coherentes resultarán además sobre-completos. Podemos verificar esta propiedad evaluando explícitamente la relación canónica de completitud
        \[ \int_\complexSet \dd{\alpha} \dyad{\alpha}{\alpha}, \]
        donde la integral de $\alpha$ sobre $\complexSet$ resulta equivalente a una integral real en $(x, y)$ sobre $\realSet^2$, donde $\alpha = x + i y$.
        \begin{align}
            \int_\complexSet \dd{\alpha} \dyad{\alpha}{\alpha} &= \iint_{\realSet^2} \dd{x} \dd{y} \qty(e^{-\frac{(x^2 + y^2)}{2}} \sum_{n = 0}^\infty \frac{(x + i y)^n}{\sqrt{n!}} \ket{n} ) \qty(e^{-\frac{(x^2 + y^2)}{2}} \sum_{m = 0}^\infty \frac{(x - i y)^m}{\sqrt{m!}} \bra{m} ) \\
                &= \sum_{n, m = 0}^\infty \frac{\dyad{n}{m}}{\sqrt{n! \ m!}} \iint_{\realSet^2} \dd{x} \dd{y} e^{-x^2 - y^2} (x + i y)^n (x - i y)^m \\
                &= \sum_{n, m = 0}^\infty \frac{\dyad{n}{m}}{\sqrt{n! \ m!}} \int_0^{+\infty} \int_0^{2\pi} r \dd{r} \dd{\theta} e^{-r^2} r^{n + m} \underbrace{e^{i\theta (n - m)}}_{= 2\pi \delta_{nm}} \\
                &= 2\pi \sum_{n = 0}^\infty \frac{\dyad{n}{n}}{n!} \int_0^{+\infty} r \dd{r} e^{-r^2} r^{2n} \\
                &\stackrel{\mathclap{u = r^2}}{=} \ 2\pi \sum_{n = 0}^\infty \frac{\dyad{n}{n}}{n!} \int_0^{+\infty} \frac{\dd{u}}{2} e^{-u} u^{n} \\
                &= \pi \sum_{n = 0}^\infty \frac{\dyad{n}{n}}{n!} \underbrace{\Gamma(n + 1)}_{=n!} \\
                &= \pi \sum_{n = 0}^\infty \dyad{n}{n} \\
                &= \pi \mathds{1},
        \end{align}
        \[ \implies \inv{\pi} \int_\complexSet \dd{\alpha} \dyad{\alpha}{\alpha} = \mathds{1}. \]
        
        
        \item Definiendo la varianza de un operador $\hat A$ como $(\Delta \hat{A})^2 = \expval{\hat A^2} - \expval{\hat A}^2$, donde los valores medios son tomados respecto a un estado común $\ket{\psi}$ (en este caso consideraremos $\ket{\psi} = \ket{\alpha}$), calcularemos a continuación las desviaciones estándar ($\Delta \hat A$) de los operadores de posición e impulso. Ya se han calculado los valores medios de $\hat p$ y $\hat q$ en el inciso I.1.e, por lo que solo resta evaluar $\expval{\hat p^2}$ y $\expval{\hat q^2}$. En cada caso,
        \begin{align}
            \expval{\hat p^2} = \inv{(i \sqrt{2})^2} \mel{\alpha}{(\a - \a^\dagger)^2}{\alpha} &= -\inv{2} \mel{\alpha}{\a^2 + (\a^\dagger)^2 - \a \a^\dagger - \a^\dagger \a}{\alpha} \\
                &= -\inv{2} \mel{\alpha}{\a^2 + (\a^\dagger)^2 - 2 \a^\dagger \a - 1}{\alpha} \\
                &= -\inv{2} \mel{\alpha}{\alpha^2 + (\alpha^*)^2 - 2 \alpha^* \alpha - 1}{\alpha} \\
                &= -\inv{2} \qty((\alpha - \alpha^*)^2 - 1) \\
                &= -\inv{2} \qty((2 i \Im(\alpha))^2 - 1) \\
                &= -\inv{2} \qty(-4 \Im(\alpha)^2 - 1) \\
                &= 2 \Im(\alpha)^2 + \inv{2},
        \end{align}
        y
        \begin{align}
            \expval{\hat q^2} = \inv{(\sqrt{2})^2} \mel{\alpha}{(\a + \a^\dagger)^2}{\alpha} &= \inv{2} \mel{\alpha}{\a^2 + (\a^\dagger)^2 + \a \a^\dagger + \a^\dagger \a}{\alpha} \\
                &= \inv{2} \mel{\alpha}{\a^2 + (\a^\dagger)^2 + 2 \a^\dagger \a + 1}{\alpha} \\
                &= \inv{2} \mel{\alpha}{\alpha^2 + (\alpha^*)^2 + 2 \alpha^* \alpha + 1}{\alpha} \\
                &= \inv{2} \qty((\alpha + \alpha^*)^2 + 1) \\
                &= \inv{2} \qty((2 \Re(\alpha))^2 + 1) \\
                &= 2 \Re(\alpha)^2 + \inv{2}.
        \end{align}
        
        Luego,
        \[ (\Delta \hat p)^2 = \expval{\hat p^2} - \expval{\hat p}^2 = 2 \Im(\alpha)^2 + \inv{2} - 2 \Im(\alpha)^2 = \inv{2} \]
        y
        \[ (\Delta \hat q)^2 = \expval{\hat q^2} - \expval{\hat q}^2 = 2 \Re(\alpha)^2 + \inv{2} - 2 \Re(\alpha)^2 = \inv{2}, \]
        pudiendo concluir que el producto de incertezas resulta
        \[ \Delta \hat p \Delta \hat q = \inv{\sqrt{2}} \inv{\sqrt{2}} = \inv{2}, \]
        siendo este el valor mínimos permitido por la relación de incerteza de Heisenberg (en unidades naturales) $\Delta \hat p \Delta \hat q \geq 1/2$.
        
    \end{enumerate}
    
    
    
    %-------------------------------------------------------------------------------------------------------
    %   Problema I.2
    %-------------------------------------------------------------------------------------------------------
    \item \begin{enumerate}
        \item Representamos el efecto de un beamsplitter (divisor de haces) por medio del operador unitario
        \[ \hat U = e^{-i\theta (\a_1 \a_2^\dagger + \a_1^\dagger \a_2)} = e^{-i\theta \hat{\mathbb{X}}}, \]
        donde $\theta$ es el parámetro del dispositivo. Operando sobre un estado coherente en el espacio producto tensorial $\ket{\alpha} \otimes \ket{\beta}$, el accionar de $\hat U$ resulta
        \begin{align}
            \hat U \ket{\alpha} \otimes \ket{\beta} &= e^{-\frac{\abs{\alpha}^2}{2}} e^{-\frac{\abs{\beta}^2}{2}} \hat U e^{\beta \a_1^\dagger} e^{\alpha \a_2^\dagger} \ket{0} \otimes \ket{0} \\
                &= e^{-\frac{\abs{\alpha}^2 + \abs{\beta}^2}{2}} \hat U e^{\beta \a_1^\dagger} \hat U^\dagger \hat U e^{\alpha \a_2^\dagger} \hat U^\dagger \underbrace{\hat U \ket{0} \otimes \ket{0}}_{= \mathds{1} \ket{0} \otimes \ket{0}} \\
                &= e^{-\frac{\abs{\alpha}^2 + \abs{\beta}^2}{2}} e^{\beta \hat U \a_1^\dagger \hat U^\dagger} e^{\alpha \hat U \a_2^\dagger \hat U^\dagger} \ket{0} \otimes \ket{0}, \label{eq:I_2_a_1}
        \end{align}
        donde se ha hecho uso de la unitariedad de $\hat U$ para colocarlo dentro de las exponenciales.
        
        Procederemos ahora a utilizar la fórmula BCH en la forma
        \[ e^{\lambda \hat B} \hat A e^{-\lambda \hat B} = \sum_{n = 0}^\infty \frac{\lambda^n}{n!} C_n, \]
        donde los coeficientes $C_n$ son determinados por la sucesión recursiva
        \[ C_0 = \hat A, \ C_1 = \comm{\hat B}{C_0} = \comm{\hat B}{C_1}, \ \dots \ , \ C_n = \comm{\hat B}{C_{n - 1}}. \]
        Resulta fácil de esta manera evaluar los argumentos de las exponenciales de \eqref{eq:I_2_a_1}, siendo en el primer caso ahora
        \[ \hat U \a_1^\dagger \hat U^\dagger = \sum_{n = 0}^\infty \frac{(-i\theta)^n}{n!} C^{(1)}_n, \]
        donde
        \[ C^{(1)}_0 = \a_1^\dagger, \quad \quad C^{(1)}_1 = \comm{\a_1 \a_2^\dagger + \a_1^\dagger \a_2}{\a_1^\dagger} = \a_2^\dagger \comm{\a_1}{\a_1^\dagger} = \a_2^\dagger \]
        \[ C^{(1)}_2 = \comm{\a_1 \a_2^\dagger + \a_1^\dagger \a_2}{\a_2^\dagger} = \a_1^\dagger \comm{\a_2}{\a_2^\dagger} = \a_1^\dagger \]
        \[ \implies C^{(1)}_{2n} = \a_1^\dagger, \quad \quad C^{(1)}_{2n + 1} = \a_2^\dagger. \]
        Luego,
        \begin{align}
            \hat U \a_1^\dagger \hat U^\dagger &= \sum_{n = 0}^\infty \frac{(-i\theta)^{2n}}{(2n)!} C^{(1)}_{2n} + \sum_{n = 0}^\infty \frac{(-i\theta)^{2n + 1}}{(2n + 1)!} C^{(1)}_{2n + 1} \\
                &= \a_1^\dagger \sum_{n = 0}^\infty (-1)^n \frac{\theta^{2n}}{(2n)!} - i \a_2^\dagger \sum_{n = 0}^\infty (-1)^n \frac{\theta^{2n + 1}}{(2n + 1)!} \\
                &= \a_1^\dagger \cos\theta - i \a_2^\dagger \sin\theta.
        \end{align}
        
        De la misma manera podremos evaluar el argumento de la segunda exponencial de \eqref{eq:I_2_a_1}, obteniendo en este caso
        \[ \hat U \a_2^\dagger \hat U^\dagger = \a_2^\dagger \cos\theta - i \a_1^\dagger \sin\theta, \]
        ya que
        \[ C^{(2)}_0 = \a_2^\dagger = C^{(1)}_1 \implies C^{(2)}_{2n} = \a_2^\dagger, \quad \quad C^{(2)}_{2n + 1} = \a_1^\dagger. \]
        
        Reemplazando estos resultados parciales en \eqref{eq:I_2_a_1} y utilizando el enunciado tradicional de la fórmula BCH para separar y re-agrupar las exponenciales resultantes, finalmente tendremos
        \begin{align}
            \hat U \ket{\alpha} \otimes \ket{\beta} &= e^{-\frac{\abs{\alpha}^2 + \abs{\beta}^2}{2}} e^{\beta (\a_1^\dagger \cos\theta - i \a_2^\dagger \sin\theta)} e^{\alpha (\a_2^\dagger \cos\theta - i \a_1^\dagger \sin\theta)} \ket{0} \otimes \ket{0} \\
                &= e^{-\frac{\abs{\alpha}^2 + \abs{\beta}^2}{2}} e^{(\beta \cos\theta - i \alpha \sin\theta) \a_1^\dagger} e^{(\alpha \cos\theta - i \beta \sin\theta) \a_2^\dagger} \ket{0} \otimes \ket{0} \\
                &= \ket{\alpha \cos\theta - i \beta \sin\theta} \otimes \ket{\beta \cos\theta - i \alpha \sin\theta},
        \end{align}
        ya que
        \begin{align}
            \abs{\alpha \cos\theta - i \beta \sin\theta}^2 &= \abs{(x + i y) \cos\theta - i (z + i w) \sin\theta}^2 \\
                &= \abs{x \cos\theta + w \sin\theta + i (y \cos\theta - z \sin\theta)}^2 \\
                &= (x \cos\theta + w \sin\theta)^2 + (y \cos\theta - z \sin\theta)^2,
        \end{align}
        \begin{align}
            \abs{\beta \cos\theta - i \alpha \sin\theta}^2 &= \abs{(z + i w) \cos\theta - i (x + i y) \sin\theta}^2 \\
                &= \abs{z \cos\theta + y \sin\theta + i (w \cos\theta - x \sin\theta)}^2 \\
                &= (z \cos\theta + y \sin\theta)^2 + (w \cos\theta - x \sin\theta)^2,
        \end{align}
        \begin{align}
            \implies &\abs{\beta \cos\theta - i \alpha \sin\theta}^2 + \abs{\alpha \cos\theta - i \beta \sin\theta}^2 \\
                &\quad = (x \cos\theta + w \sin\theta)^2 + (y \cos\theta - z \sin\theta)^2 + (z \cos\theta + y \sin\theta)^2 \\
                    &\quad\quad\quad + (w \cos\theta - x \sin\theta)^2 \\
                &\quad = x^2 + y^2 + z^2 + w^2 \\
                &\quad = \abs{\alpha}^2 + \abs{\beta}^2.
        \end{align}
    
    
        \item Definiendo los estados
        \[ \ket{10} = \a_1^\dagger \ket{00}, \quad \quad \ket{01} = \a_2^\dagger \ket{00}, \]
        el accionar de $\hat U$ puede ser rápidamente calculado repitiendo los procedimientos del inciso anterior, aunque en este caso no hará falta considerar una exponencial generadora de estados coherentes, sino un operador de creación individual.
        \[ \hat U \ket{01} = \hat U \a_1^\dagger \ket{00} = \hat U \a_1^\dagger \hat U^\dagger \hat U \ket{00} = (\a_1^\dagger \cos\theta - i \a_2^\dagger \sin\theta) \ket{00} = \ket{\cos\theta, \ -i \sin\theta}, \]
        \[ \hat U \ket{10} = \hat U \a_2^\dagger \ket{00} = \hat U \a_2^\dagger \hat U^\dagger \hat U \ket{00} = (\a_2^\dagger \cos\theta - i \a_1^\dagger \sin\theta) \ket{00} = \ket{-i \sin\theta, \ \cos\theta}. \]
    
    \end{enumerate}
    
\end{enumerate}

\section{Transformaciones Unitarias y de Bogoliubov}
\begin{enumerate}
    
    %-------------------------------------------------------------------------------------------------------
    %   Problema II.1
    %-------------------------------------------------------------------------------------------------------
    \item \begin{enumerate}
        \item Sea
        \[ \hat H = \varepsilon (\c_1^\dagger \c_1 + \c_2^\dagger \c_2) - v (\c_1^\dagger \c_2 + \c_2^\dagger \c_1) \]
        el hamiltoniano de un sistema. Procederemos a diagonalizarlo utilizando la transformada de Fourier de los operadores $\c_k$, definida como
        \[ b_j = \inv{\sqrt{n}} \sum_{k = 0}^n e^{-i \frac{2\pi}{n} jk} c_k \iff c_k = \inv{\sqrt{n}} \sum_{j = 0}^n e^{i \frac{2\pi}{n} jk} b_j. \]
        
        En general, para un sistema de $n$ partículas, los términos cinéticos del hamiltoniano resultarán
        \begin{align}
            \sum_{j = 1}^n c_j^\dagger c_j = \inv{n} \sum_{j, k, k' = 1}^n e^{-i \frac{2\pi}{n} jk} e^{i \frac{2\pi}{n} jk'} \b_k^\dagger \b_{k'} = \inv{n} \sum_{j, k, k' = 1}^n e^{i \frac{2\pi}{n} j (k' - k)} \b_k^\dagger \b_{k'} &= \inv{n} \sum_{k, k' = 1}^n n \delta_{k k'} \b_k^\dagger \b_{k'} \\
                &= \sum_{k = 1}^n \b_k^\dagger \b_k.
        \end{align}
        Así mismo, los términos de intercambio resultarán
        \begin{align}
            \sum_{j = 1}^n ( c_j^\dagger c_{j + 1} + c_{j + 1}^\dagger c_j) &= \inv{n} \sum_{j, k, k' = 1}^n \left( e^{-i \frac{2\pi}{n} jk} e^{i \frac{2\pi}{n} (j + 1) k'} b_k^\dagger b_{k'} + e^{-i \frac{2\pi}{n} (j + 1) k} e^{i \frac{2\pi}{n} jk'} b_{k}^\dagger b_{k'} \right) \\
                &= \inv{n} \sum_{j, k, k' = 1}^n \left( e^{i \frac{2\pi}{n} k'} e^{i \frac{2\pi}{n} j (k' - k)} b_k^\dagger b_{k'} + e^{-i \frac{2\pi}{n} k} e^{i \frac{2\pi}{n} j(k' - k)} b_{k}^\dagger b_{k'} \right) \\
                &= \inv{n} \sum_{k, k' = 1}^n \left( e^{i \frac{2\pi}{n} k'} n \delta_{k k'} b_k^\dagger b_{k'} + e^{-i \frac{2\pi}{n} k} n \delta_{k k'} b_{k}^\dagger b_{k'} \right) \\
                &= \sum_{k = 1}^n \left( e^{i \frac{2\pi}{n} k} + e^{-i \frac{2\pi}{n} k} \right) b_k^\dagger b_k \\
                &= \sum_{k = 1}^n 2 \cos(\frac{2\pi}{n} k) b_k^\dagger b_k.
        \end{align}
        Luego, el hamiltoniano de $n$ partículas será
        \[ \hat H = \sum_{j = 1}^n \left[ \varepsilon \c_j^\dagger \c_j - v (\c_j^\dagger \c_{j + 1} + \c_{j + 1}^\dagger \c_j) \right] = \sum_{k = 1}^n \left( \varepsilon - 2 v \cos(\frac{2\pi}{n} k) \right) b_k^\dagger b_k = \sum_{k = 1}^n \lambda_k b_k^\dagger b_k, \]
        donde $\lambda_k = \varepsilon - 2v \cos(2\pi k / n)$ son las autoenergías del sistema, con autoestados
        \[ \b_k^\dagger = \inv{\sqrt{n}} \sum_{j = 1}^n e^{-i \frac{2\pi}{n} jk} \c_j^\dagger \ket{0}. \]
        
        El nuevo vacío asociado a la base de operadores $\b_k$ deberá satisfacer $\b_k \ket{0'} = 0, \ \forall k$. Proponiendo entonces una combinación lineal de estados $\ket{0'} = \sum_{m = 0}^n \xi_m \ket{m}$,
        \[ 0 = \b_k \ket{0'} = \inv{\sqrt{n}} \sum_{j = 0}^{n - 1} \sum_{m = 0}^n e^{-i \frac{2\pi}{n} jk} \xi_m \c_j \ket{m} \implies \xi_m = \delta_{m0}, \]
        ya que todos los términos de la sumatoria en $m$ deben anularse independientemente, por la ortogonalidad de los estados $\ket{m}$. Luego, $\ket{0'} = \ket{0}$.
        
        En el caso particular de un sistema de 2 partículas, como el que se propone en la consigna, las autoenergías asociadas a $\hat H$ serán
        \[ \lambda_1 = \varepsilon - 2v \cos(\frac{2\pi}{2} 1) = \varepsilon + 2 v, \quad \quad \lambda_1 = \varepsilon - 2v \cos(\frac{2\pi}{2} 2) = \varepsilon - 2 v, \]
        con autoestados
        \[ \b_1^\dagger \ket{0} = \inv{\sqrt{2}} \left( e^{-i \pi} \c_1^\dagger \ket{0} + e^{-i 2\pi} \c_2^\dagger \ket{0} \right) = \inv{\sqrt{2}} (\ket{10} - \ket{01}) \]
        y
        \[ \b_2^\dagger \ket{0} = \inv{\sqrt{2}} \left( e^{-i 2\pi} \c_1^\dagger \ket{0} + e^{-i 4\pi} \c_2^\dagger \ket{0} \right) = \inv{\sqrt{2}} (\ket{10} + \ket{01}). \]
        
        
        \item Sea ahora el hamiltoniano
        \[ \hat H = \varepsilon (\c_1^\dagger \c_1 + \c_2^\dagger \c_2) - v (\c_1^\dagger \c_2^\dagger + \c_2 \c_1). \]
        \begin{enumerate}[(i)]
            \item En el caso fermiónico, donde los operadores de creación y aniquilación cumplen las relaciones de anticonmutación
            \[ \acomm{c_j}{c_k^\dagger} = \delta_{jk}, \quad\quad \acomm{c_j}{c_k} = \acomm{c_j^\dagger}{c_k^\dagger} = 0, \]
            podremos expresar el hamiltoniano del sistema como
            \begin{align}
                \hat H &= \epsilon \left( \frac{\c_1^\dagger \c_1 + \c_1^\dagger \c_1}{2} + \frac{\c_2^\dagger \c_2 + \c_2^\dagger \c_2}{2} \right) - v \left( \frac{\c_1^\dagger \c_2^\dagger + \c_2^\dagger \c_1^\dagger}{2} + \frac{\c_1 \c_2 + \c_2 \c_1}{2} \right) \\
                    &= \epsilon \left( \frac{\c_1^\dagger \c_1 - \c_1 \c_1^\dagger + 1}{2} + \frac{\c_2^\dagger \c_2 - \c_2 \c_2^\dagger + 1}{2} \right) - v \left( \frac{\c_1^\dagger \c_2^\dagger + \c_2^\dagger \c_1^\dagger}{2} + \frac{\c_1 \c_2 + \c_2 \c_1}{2} \right) \\
                    &= \epsilon \left( \frac{\c_1^\dagger \c_1 - \c_1 \c_1^\dagger}{2} + \frac{\c_2^\dagger \c_2 - \c_2 \c_2^\dagger}{2} \right) - v \left( \frac{\c_1^\dagger \c_2^\dagger + \c_2^\dagger \c_1^\dagger}{2} + \frac{\c_1 \c_2 + \c_2 \c_1}{2} \right) + \varepsilon \\
                    &= \inv{2} \begin{pmatrix} \c_1^\dagger & \c_2^\dagger & \c_1 & \c_2 \end{pmatrix} \underbrace{\begin{pmatrix} \varepsilon & 0 & 0 & -v \\ 0 & \varepsilon & v & 0 \\ 0 & v & -\varepsilon & 0 \\ -v & 0 & 0 & -\varepsilon \end{pmatrix}}_{\mathcal{H}} \begin{pmatrix} \c_1 \\ \c_2 \\ \c_1^\dagger \\ \c_2^\dagger \end{pmatrix} + \varepsilon.
            \end{align}
            
            Introduciremos ahora los operadores de creación y aniquilación transformados $\a_k^\dagger, \ \a_k$, definidos en términos de los operadores $\c_k^\dagger, \ \c_k$ por medio de una transformación de Bogoliubov como
            \[ \begin{pmatrix} \a \\ \a^\dagger \end{pmatrix} = \underbrace{\begin{pmatrix} U & V \\ V^* & U^* \end{pmatrix}}_{W} \begin{pmatrix} \c \\ \c^\dagger \end{pmatrix}, \]
            donde la matriz de transformación $W$ resulta ortogonal. Luego, podremos diagonalizar al hamiltoniano utilizando la transformación exigiendo
            \[ \hat H = \inv{2} \begin{pmatrix} \a^\dagger & \a \end{pmatrix} W \mathcal{H} W^t \begin{pmatrix} \a \\ \a^\dagger \end{pmatrix} - \varepsilon. \]
            
            Podremos diagonalizar $\mathcal{H}$ estudiando el bloque central y el bloque de las esquinas por separado
            \[ A = \begin{pmatrix} \varepsilon & -v \\ -v & -\varepsilon \end{pmatrix}, \quad\quad B = \begin{pmatrix} \varepsilon & v \\ v & -\varepsilon \end{pmatrix}. \]
            En ambos casos, los autovalores asociados resultarán $\lambda_\pm = \pm \lambda = \pm \sqrt{\varepsilon^2 + v^2}$, con matrices de transformación ortogonales\footnote{
                En el primer caso,
                \[ A - \lambda_{\pm} \mathds{1} = \begin{pmatrix} \varepsilon - \lambda_\pm & -v \\ -v & -(\varepsilon + \lambda_\pm) \end{pmatrix} = \begin{pmatrix} \varepsilon \mp \lambda & -v \\ -v & -(\varepsilon \pm \lambda) \end{pmatrix}, \]
                por lo que
                \[ \vb{v}_\pm^A = \begin{pmatrix} 1 \\ \frac{\varepsilon \mp \lambda}{v} \end{pmatrix} \longrightarrow \begin{pmatrix} \frac{v}{\sqrt{2\lambda}} \\ \mp \frac{\lambda \mp \varepsilon}{\sqrt{2\lambda}} \end{pmatrix} \mathrel{\stackrel{(\star)}{=}} \begin{pmatrix} \frac{\sqrt{\lambda - \varepsilon} \sqrt{\lambda + \varepsilon}}{\sqrt{2\lambda}} \\ \mp \frac{\lambda \mp \varepsilon}{\sqrt{2\lambda}} \end{pmatrix} \]
                \[ \implies \vb{v}_+^A = \begin{pmatrix} \frac{\sqrt{\lambda - \varepsilon} \sqrt{\lambda + \varepsilon}}{\sqrt{2\lambda}} \\ - \frac{\lambda - \varepsilon}{\sqrt{2\lambda}} \end{pmatrix} \longrightarrow \begin{pmatrix} \sqrt{\frac{\lambda + \varepsilon}{2\lambda}} \\ -\sqrt{\frac{\lambda - \varepsilon}{2\lambda}} \end{pmatrix} = \begin{pmatrix} u \\ -\mathrm{v} \end{pmatrix}, \quad\quad \vb{v}_-^A = \begin{pmatrix} \sqrt{\frac{\lambda - \varepsilon}{2\lambda}} \\ \sqrt{\frac{\lambda + \varepsilon}{2\lambda}} \end{pmatrix} = \begin{pmatrix} \mathrm{v} \\ u \end{pmatrix}, \]
                habiendo utilizado en $(\star)$ la expresión $\lambda_\pm = \pm \sqrt{\varepsilon^2 + v^2}$ para los autovalores, por lo que $v = \sqrt{\lambda^2 - \varepsilon^2} = \sqrt{\lambda + \varepsilon} \sqrt{\lambda - \varepsilon}$.
                
                Así mismo, para
                \[ B - \lambda_{\pm} \mathds{1} = \begin{pmatrix} \varepsilon - \lambda_\pm & v \\ v & -(\varepsilon + \lambda_\pm) \end{pmatrix} = \begin{pmatrix} \varepsilon \mp \lambda & v \\ v & -(\varepsilon \pm \lambda) \end{pmatrix}, \]
                tendremos ahora
                \[ \vb{v}_\pm^B = \begin{pmatrix} 1 \\ -\frac{\varepsilon \mp \lambda}{v} \end{pmatrix} \implies \vb{v}_+^B = \begin{pmatrix} \sqrt{\frac{\lambda + \varepsilon}{2\lambda}} \\ \sqrt{\frac{\lambda - \varepsilon}{2\lambda}} \end{pmatrix} = \begin{pmatrix} u \\ \mathrm{v} \end{pmatrix}, \quad \quad \vb{v}_-^B = \begin{pmatrix} \sqrt{\frac{\lambda - \varepsilon}{2\lambda}} \\ -\sqrt{\frac{\lambda + \varepsilon}{2\lambda}} \end{pmatrix} \longrightarrow \begin{pmatrix} -\sqrt{\frac{\lambda - \varepsilon}{2\lambda}} \\ \sqrt{\frac{\lambda + \varepsilon}{2\lambda}} \end{pmatrix} = \begin{pmatrix} -\mathrm{v} \\ u \end{pmatrix}. \]
                
                En ambos casos, es fácil verificar que las matrices de transformación $P_A$ y $P_B$ construidas con los autovectores obtenidos resultarán ortogonales, notando que
                \[ u^2 + \mathrm{v}^2 = \frac{\lambda + \varepsilon}{2\lambda} + \frac{\lambda - \varepsilon}{2\lambda} = 1, \]
                por lo que entonces $P_A P_A^t = P_A^t P_A = \mathds{1}$ (y análogamente $P_B P_B^t = P_B^t P_B = \mathds{1}$).
            }
            \[ P_A = \begin{pmatrix} u & \mathrm{v} \\ -\mathrm{v} & u \end{pmatrix}, \quad \text{y} \quad P_B = \begin{pmatrix} u & -\mathrm{v} \\ \mathrm{v} & u \end{pmatrix}, \]
            donde
            \[ u = \sqrt{\frac{\lambda + \varepsilon}{2\lambda}}, \quad \quad \mathrm{v} = \sqrt{\frac{\lambda - \varepsilon}{2\lambda}}, \]
            tal que $A = P_A D_A P_A^{-1}$ (y $B = P_B D_B P_B^{-1}$), con
            \[ D_A = D_B = \begin{pmatrix} \lambda_+ & 0 \\ 0 & \lambda_- \end{pmatrix} = \begin{pmatrix} \lambda & 0 \\ 0 & -\lambda \end{pmatrix}. \]
            Luego, 
            \[ \mathcal{H} = \underbrace{\begin{pmatrix} u & 0 & 0 & \mathrm{v} \\ 0 & u & -\mathrm{v} & 0 \\ 0 & \mathrm{v} & u & 0 \\ -\mathrm{v} & 0 & 0 & u \end{pmatrix}}_{P = W^t} \underbrace{\begin{pmatrix} \lambda & & & \\ & \lambda & & \\ & & -\lambda & \\ & & & -\lambda \end{pmatrix}}_{D} \underbrace{\begin{pmatrix} u & 0 & 0 & -\mathrm{v} \\ 0 & u & \mathrm{v} & 0 \\ 0 & -\mathrm{v} & u & 0 \\ \mathrm{v} & 0 & 0 & u \end{pmatrix}}_{P^{-1} = P^t = W}, \]
            pudiendo entonces escribir al hamiltoniano del sistema en la nueva base como
            \begin{align}
                \hat H &= \inv{2} \begin{pmatrix} \a_1^\dagger & \a_2^\dagger & \a_1 & \a_2 \end{pmatrix} \begin{pmatrix} \lambda & & & \\ & \lambda & & \\ & & -\lambda & \\ & & & -\lambda \end{pmatrix} \begin{pmatrix} \a_1 \\ \a_2 \\ \a_1^\dagger \\ \a_2^\dagger \end{pmatrix} + \varepsilon \\
                    &= \frac{\lambda}{2} \left[ \a_1^\dagger \a_1 + \a_2^\dagger \a_2 - \a_1 \a_1^\dagger - \a_2 \a_2^\dagger \right] + \varepsilon \\
                    &= \frac{\lambda}{2} \left[ 2 \left(\a_1^\dagger \a_1 + \a_2^\dagger \a_2 \right) - 2 \right] + \varepsilon \\
                    &= \lambda \hat{N} - (\lambda - \varepsilon) = \lambda \hat{N} + E_0,
            \end{align}
            donde $E_0 = -(\lambda - \varepsilon)$ es la energía de vacío del sistema.
            
            Podremos encontrar el nuevo vacío de la teoría a partir de la expresión general
            \[ \ket{0'} = C \exp{\inv{2} \sum_{ij} T_{ij} \c_i^\dagger \c_j^\dagger} \ket{0}, \]
            donde $T = U^{-1} V$, y $C$ es una constante de normalización a determinar. En este caso, tenemos
            \[ U = \begin{pmatrix} u & 0 \\ 0 & u \end{pmatrix}, \quad V = \begin{pmatrix} 0 & -\mathrm{v} \\ \mathrm{v} & 0 \end{pmatrix} \implies T = U^{-1} V = \begin{pmatrix} 0 & -\frac{\mathrm{v}}{u} \\ \frac{\mathrm{v}}{u} & 0 \end{pmatrix}, \]
            por lo que
            \begin{align}
                \ket{0'} = C \exp{\inv{2} \left( -\frac{\mathrm{v}}{u} \c_1^\dagger \c_2^\dagger + \frac{\mathrm{v}}{u} \c_2^\dagger \c_1^\dagger \right)} \ket{0} &= C \exp{\inv{2} \left( -\frac{\mathrm{v}}{u} \c_1^\dagger \c_2^\dagger - \frac{\mathrm{v}}{u} \c_1^\dagger \c_2^\dagger \right)} \ket{0} \\
                    &= C \exp{-\frac{\mathrm{v}}{u} \c_1^\dagger \c_2^\dagger} \ket{0} \\
                    &= C \left( \mathds{1} - \frac{\mathrm{v}}{u} \c_1^\dagger \c_2^\dagger + \cancel{\dots} \right) \ket{0} \\
                    &= C \left( \ket{00} - \frac{\mathrm{v}}{u} \ket{11} \right),
            \end{align}
            habiéndonos valido de las relaciones de anticonmutación para simplificar la exponencial, y de la estadística fermiónica de las partículas para despreciar términos superiores de la expansión en serie (no podremos tener más de una partícula en el sistema con el mismo número cuántico).
            
            Para determinar la constante de normalización deberemos simplemente evaluar
            \[ 1 = \ip{0'}{0'} = C^2 \left( 1 + \frac{\mathrm{v}^2}{u^2} \right) = \frac{C^2}{u^2} (u^2 + \mathrm{v}^2) = \frac{C^2}{u^2} \implies C = u. \]
            \[ \therefore \ket{0'} = u \ket{00} - \mathrm{v} \ket{11}. \]
            
            
            \item En el caso bosónico, donde ahora los operadores de creación y aniquilación cumplen las relaciones de conmutación
            \[ \comm{c_j}{c_k^\dagger} = \delta_{jk}, \quad\quad \comm{c_j}{c_k} = \comm{c_j^\dagger}{c_k^\dagger} = 0, \]
            podremos re-escribir el hamiltoniano del sistema en un procedimiento análogo al inciso anterior como
            \begin{align}
                \hat H &= \epsilon \left( \frac{\c_1^\dagger \c_1 + \c_1^\dagger \c_1}{2} + \frac{\c_2^\dagger \c_2 + \c_2^\dagger \c_2}{2} \right) - v \left( \frac{\c_1^\dagger \c_2^\dagger + \c_2^\dagger \c_1^\dagger}{2} + \frac{\c_1 \c_2 + \c_2 \c_1}{2} \right) \\
                    &= \epsilon \left( \frac{\c_1^\dagger \c_1 + \c_1 \c_1^\dagger - 1}{2} + \frac{\c_2^\dagger \c_2 + \c_2 \c_2^\dagger - 1}{2} \right) - v \left( \frac{\c_1^\dagger \c_2^\dagger + \c_2^\dagger \c_1^\dagger}{2} + \frac{\c_1 \c_2 + \c_2 \c_1}{2} \right) \\
                    &= \epsilon \left( \frac{\c_1^\dagger \c_1 + \c_1 \c_1^\dagger}{2} + \frac{\c_2^\dagger \c_2 + \c_2 \c_2^\dagger}{2} \right) - v \left( \frac{\c_1^\dagger \c_2^\dagger + \c_2^\dagger \c_1^\dagger}{2} + \frac{\c_1 \c_2 + \c_2 \c_1}{2} \right) - \varepsilon \\
                    &= \inv{2} \begin{pmatrix} \c_1^\dagger & \c_2^\dagger & \c_1 & \c_2 \end{pmatrix} \underbrace{\begin{pmatrix} \varepsilon & 0 & 0 & -v \\ 0 & \varepsilon & -v & 0 \\ 0 & -v & \varepsilon & 0 \\ -v & 0 & 0 & \varepsilon \end{pmatrix}}_{\mathcal{H}} \begin{pmatrix} \c_1 \\ \c_2 \\ \c_1^\dagger \\ \c_2^\dagger \end{pmatrix} - \varepsilon.
            \end{align}
            
            Nuevamente, introduciremos nuevos operadores de creación y aniquilación transformados $\a_k^\dagger, \ \a_k$, definidos en términos de los operadores $\c_k^\dagger, \ \c_k$ por medio de una transformación de Bogoliubov como
            \[ \begin{pmatrix} \a \\ \a^\dagger \end{pmatrix} = \underbrace{\begin{pmatrix} U & V \\ V^* & U^* \end{pmatrix}}_{W} \begin{pmatrix} \c \\ \c^\dagger \end{pmatrix}. \]
            A diferencia del caso fermiónico, la matriz de transformación $W$ no resultará ortogonal, sino que satisfará la identidad $W \Pi W^\dagger = \Pi$, donde
            \[ \Pi = \begin{pmatrix} \mathds{1} & 0 \\ 0 & -\mathds{1} \end{pmatrix} = \begin{pmatrix} \comm{\c}{\c^\dagger} & \comm{\c}{\c} \\ \comm{\c^\dagger}{\c^\dagger} & \comm{\c^\dagger}{\c} \end{pmatrix} \]
            es la matriz simpléctica.
            
            Luego, definiendo $M = W^{-1}$, podremos diagonalizar el hamiltoniano siguiendo
            \begin{align}
                \hat H = \inv{2} \begin{pmatrix} \a^\dagger & \a \end{pmatrix} M^\dagger \mathcal{H} M \begin{pmatrix} \a \\ \a^\dagger \end{pmatrix} - \varepsilon &= \inv{2} \begin{pmatrix} \a^\dagger & \a \end{pmatrix} \Pi M^{-1} \Pi \mathcal{H} M \begin{pmatrix} \a \\ \a^\dagger \end{pmatrix} - \varepsilon \\
                    &= \inv{2} \begin{pmatrix} \a^\dagger & \a \end{pmatrix} \Pi M^{-1} \mathcal{H}_\Pi M \begin{pmatrix} \a \\ \a^\dagger \end{pmatrix} - \varepsilon.
            \end{align}
            
            La matriz
            \[ \mathcal{H}_\Pi = \begin{pmatrix} \varepsilon & 0 & 0 & -v \\ 0 & \varepsilon & -v & 0 \\ 0 & v & -\varepsilon & 0 \\ v & 0 & 0 & -\varepsilon \end{pmatrix} \]
            tiene esta vez dos bloques iguales
            \[ A = \begin{pmatrix} \varepsilon & -v \\ v & -\varepsilon \end{pmatrix} \]
            por lo que solo requeriremos diagonalizar uno de ellos, de donde obtendremos autovalores $\lambda_\pm = \pm \lambda = \pm \sqrt{\varepsilon^2 - v^2}$ (teniendo que introducir la restricción $\abs{v} < \abs{\varepsilon}$ para que la energía sea una magnitud física) y una matriz de transformación asociada\footnote{
                Para cada uno de los bloques,
                \[ A - \lambda_{pm} \mathds{1} = \begin{pmatrix} \varepsilon - \lambda_\pm & v \\ -v & -(\varepsilon + \lambda_\pm) \end{pmatrix} = \begin{pmatrix} \varepsilon \mp \lambda & -v \\ v & -(\varepsilon \pm \lambda) \end{pmatrix}, \]
                por lo que
                \[ \vb{v}_\pm = \begin{pmatrix} 1 \\ \frac{\varepsilon \mp \lambda}{v} \end{pmatrix} \longrightarrow \begin{pmatrix} \frac{v}{\sqrt{2\lambda}} \\ \frac{\varepsilon \mp \lambda}{\sqrt{2\lambda}} \end{pmatrix} \mathrel{\stackrel{(\star)}{=}} \begin{pmatrix} \frac{\sqrt{\varepsilon - \lambda} \sqrt{\varepsilon + \lambda}}{\sqrt{2\lambda}} \\ \frac{\varepsilon \mp \lambda}{\sqrt{2\lambda}} \end{pmatrix} \]
                \[ \implies \vb{v}_+ = \begin{pmatrix} \frac{\sqrt{\varepsilon - \lambda} \sqrt{\varepsilon + \lambda}}{\sqrt{2\lambda}} \\ \frac{\varepsilon - \lambda}{\sqrt{2\lambda}} \end{pmatrix} \longrightarrow \begin{pmatrix} \sqrt{\frac{\varepsilon + \lambda}{2\lambda}} \\ \sqrt{\frac{\varepsilon - \lambda}{2\lambda}} \end{pmatrix} = \begin{pmatrix} u \\ \mathrm{v} \end{pmatrix}, \quad\quad \vb{v}_- = \begin{pmatrix} \sqrt{\frac{\varepsilon - \lambda}{2\lambda}} \\ \sqrt{\frac{\varepsilon + \lambda}{2\lambda}} \end{pmatrix} = \begin{pmatrix} \mathrm{v} \\ u \end{pmatrix}, \]
                habiendo utilizado en $(\star)$ la expresión $\lambda_\pm = \pm \sqrt{\varepsilon^2 - v^2}$ para los autovalores, por lo que $v = \sqrt{\varepsilon^2 - \lambda^2} = \sqrt{\varepsilon + \lambda} \sqrt{\varepsilon - \lambda}$.
                
                La matriz diagonal asociada será
                \[ D_A = \begin{pmatrix} \lambda_+ & 0 \\ 0 & \lambda_- \end{pmatrix} = \begin{pmatrix} \lambda & 0 \\ 0 & -\lambda \end{pmatrix}. \]
                
                A diferencia del caso fermiónico, ahora $u$ y $\mathrm{v}$ satisfarán
                \[ u^2 - \mathrm{v}^2 = \frac{\varepsilon + \lambda}{2\lambda} - \frac{\varepsilon - \lambda}{2\lambda} = 1. \]
            }
            \[ P_A = \begin{pmatrix} u & v \\ v & u \end{pmatrix}, \quad\quad \text{con} \quad u = \sqrt{\frac{\varepsilon + \lambda}{2\lambda}}, \quad v = \sqrt{\frac{\varepsilon - \lambda}{2\lambda}}. \]
            
            Luego,
            \[ \mathcal{H}_\Pi = \underbrace{\begin{pmatrix} u & 0 & 0 & \mathrm{v} \\ 0 & u & \mathrm{v} & 0 \\ 0 & \mathrm{v} & u & 0 \\ \mathrm{v} & 0 & 0 & u \end{pmatrix}}_{M = W^{-1}} \underbrace{\begin{pmatrix} \lambda & & & \\ & \lambda & & \\ & & -\lambda & \\ & & & -\lambda \end{pmatrix}}_{D} \underbrace{\begin{pmatrix} u & 0 & 0 & -\mathrm{v} \\ 0 & u & -\mathrm{v} & 0 \\ 0 & -\mathrm{v} & u & 0 \\ -\mathrm{v} & 0 & 0 & u \end{pmatrix}}_{M^{-1} = W}, \]
            por lo que el hamiltoniano del sistema en la nueva base resultará
            \begin{align}
                \hat H &= \inv{2} \begin{pmatrix} \a_1^\dagger & \a_2^\dagger & \a_1 & \a_2 \end{pmatrix} \Pi \begin{pmatrix} \lambda & & & \\ & \lambda & & \\ & & -\lambda & \\ & & & -\lambda \end{pmatrix} \begin{pmatrix} \a_1 \\ \a_2 \\ \a_1^\dagger \\ \a_2^\dagger \end{pmatrix} - \varepsilon \\
                    &= \frac{\lambda}{2} \left[ \a_1^\dagger \a_1 + \a_2^\dagger \a_2 + \a_1 \a_1^\dagger + \a_2 \a_2^\dagger \right] - \varepsilon \\
                    &= \frac{\lambda}{2} \left[ 2 \left(\a_1^\dagger \a_1 + \a_2^\dagger \a_2 \right) - 2 \right] - \varepsilon \\
                    &= \lambda \hat{N} - (\lambda - \varepsilon) = \lambda \hat{N} + E_0,
            \end{align}
            donde $E_0 = -(\lambda - \varepsilon)$ es la energía de vacío del sistema.
            
            El nuevo vacío correspondiente a la base de operadores $\a_k$, $\a_k^\dagger$ estará ahora dado por
            \[ \ket{0'} = C \exp{\inv{2} \sum_{ij} T_{ij} \c_i^\dagger \c_j^\dagger} \ket{0}. \]
            Esta vez,
            \[ U = \begin{pmatrix} u & 0 \\ 0 & u \end{pmatrix}, \quad V = \begin{pmatrix} 0 & -\mathrm{v} \\ -\mathrm{v} & 0 \end{pmatrix} \implies T = U^{-1} V = \begin{pmatrix} 0 & -\frac{\mathrm{v}}{u} \\ -\frac{\mathrm{v}}{u} & 0 \end{pmatrix}, \]
            por lo que
            \begin{align}
                \ket{0'} = C \exp{\inv{2} \left( -\frac{\mathrm{v}}{u} \c_1^\dagger \c_2^\dagger - \frac{\mathrm{v}}{u} \c_2^\dagger \c_1^\dagger \right)} \ket{0} &= C \exp{\inv{2} \left( -\frac{\mathrm{v}}{u} \c_1^\dagger \c_2^\dagger - \frac{\mathrm{v}}{u} \c_1^\dagger \c_2^\dagger \right)} \ket{0} \\
                    &= C \exp{-\frac{\mathrm{v}}{u} \c_1^\dagger \c_2^\dagger} \ket{0} \\
                    &= C \sum_{n = 0}^\infty \inv{n!} \qty(-\frac{\mathrm{v}}{u})^n \qty(\c_1^\dagger \c_2^\dagger)^n \ket{0} \\
                    &= C \sum_{n = 0}^\infty \qty(-\frac{\mathrm{v}}{u})^n \qty(\frac{\c_1^\dagger}{\sqrt{n!}})^n \qty(\frac{\c_2^\dagger}{\sqrt{n!}})^n \ket{0} \\
                    &= C \sum_{n = 0}^\infty \qty(-\frac{\mathrm{v}}{u})^n \ket{nn}.
            \end{align}
            
            Para determinar la constante de normalización evaluaremos
            \[ 1 = \ip{0'}{0'} = C^2 \sum_{n = 0}^\infty \qty(-\frac{\mathrm{v}}{u})^{2n} = \frac{1}{1-\qty(\frac{\mathrm{v}}{u})^2} \implies C^2 = 1 - \qty(\frac{\mathrm{v}}{u})^2 = \frac{u^2 - \mathrm{v}^2}{u^2} = \inv{u^2}, \]
            \[ \therefore \ket{0'} = \inv{u} \sum_{n = 0}^\infty \qty(-\frac{\mathrm{v}}{u})^n \ket{nn}. \]
            
        \end{enumerate}
    \end{enumerate}
    
    
    
    %-------------------------------------------------------------------------------------------------------
    %   Problema II.2
    %-------------------------------------------------------------------------------------------------------
    \item \begin{enumerate}
        \item Habiendo verificado que en este caso el vacío no presenta cambios al diagonalizar el hamiltoniano, la matriz densidad reducida para el estado fundamental resultará, tanto para el sistema bosónico como para el sistema fermiónico,
        \[ \rho_A = \Tr_B \dyad{00}{00} = \dyad{0}{0}. \]
        Luego, la entropía de entrelazamiento resultará naturalmente nula ($E_{AB} = S(\rho_A) = 0$).
        
        
        \item \begin{enumerate}[(i)]
            \item \textbf{Caso fermiónico:} Hemos demostrado en el punto II.1.b que el estado fundamental del sistema resultará
            \[ \ket{0'} = u \ket{00} - \mathrm{v} \ket{11}, \]
            pudiendo entonces construir las matrices densidad total y reducida
            \[ \rho = \dyad{0'}{0'} = u^2 \dyad{00}{00} + \mathrm{v}^2 \dyad{11}{11} - u\mathrm{v} (\dyad{00}{11} + \dyad{11}{00}) \]
            \[ \implies \rho_A = \Tr_B \rho = u^2 \dyad{0}{0} + \mathrm{v}^2 \dyad{1}{1}. \]
            Por lo tanto, el entrelazamiento del sistema fermiónico resultará
            \[ E_{AB} = S(\rho_A) = - u^2 \log_2 u^2 - \mathrm{v}^2 \log_2 \mathrm{v}^2 > 0. \]
            
            
            \item \textbf{Caso bosónico:} Esta vez, el nuevo vacío no tendrá una forma abreviada como en los casos anteriores, siendo la matriz densidad asociada
            \[ \rho = \dyad{0'}{0'} = \inv{u^2} \sum_{n, m = 0}^\infty \qty(-\frac{\mathrm{v}}{u})^{n + m} \dyad{nn}{mm}. \]
            Luego,
            \begin{align}
                \rho_A = \Tr_B \rho &= \sum_{k = 0}^\infty \inv{u^2} \sum_{n, m = 0}^\infty \qty(-\frac{\mathrm{v}}{u})^{n + m} (\mathds{1} \otimes \bra{k}_B) \dyad{nn}{mm} (\mathds{1} \otimes \ket{k}_B) \\
                    &= \sum_{k = 0}^\infty \inv{u^2} \sum_{n, m = 0}^\infty \qty(-\frac{\mathrm{v}}{u})^{n + m} \delta_{kn} \delta_{km} \dyad{n}{m} \\
                    &= \inv{u^2} \sum_{n = 0}^\infty \qty(\frac{\mathrm{v}}{u})^{2n} \dyad{n}{n}.
            \end{align}
            
            El entrelazamiento asociado al sistema entonces será
            \[ E_{AB} = S(\rho_A) = - \inv{u^2} \sum_{n = 0}^\infty \qty(\frac{\mathrm{v}}{u})^{2n}. \]
            
            
        \end{enumerate}
    \end{enumerate}
    
\end{enumerate}

%\nocite{*}
%\printbibliography
\end{document}