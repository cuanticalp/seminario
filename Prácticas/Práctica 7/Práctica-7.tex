\documentclass[12pt]{article}
\textheight=280mm
\evensidemargin=-0cm
\oddsidemargin=-0cm
\textwidth=160mm
\topmargin=-2cm
\usepackage{bbm}
\newcommand{\be}{\begin{equation}}
\newcommand{\ee}{\end{equation}}
\newcommand{\ben}{\begin{eqnarray}}
\newcommand{\een}{\end{eqnarray}}
\newcommand{\half}{\mbox{${\textstyle\frac{1}{2}}$}}
\newcommand{\quart}{\mbox{${\textstyle\frac{1}{4}}$}}
\newcommand{\sen}{\mbox{${\rm sen}$}}
\newcommand{\senh}{\mbox{${\rm senh}$}}
\newcommand{\arcsen}{\mbox{${\rm arcsen}$}}
\newcommand{\dxy}{\mbox{$\frac{dx}{dy}$}}
\newcommand{\dyx}{\mbox{$\frac{dy}{dx}$}}
\newcommand{\rdo}{\mbox{${\Rightarrow}$}}
%\newcommand{\arccos}{\mbox{${\rm arccos}$}}
%\newcommand{\arctan}{\mbox{${\rm arctan}$}}
%\newcommand{\cos}{\mbox{${\rm cos}$}}
%\newcommand{\tan}{\mbox{${\rm tan}$}}
\newcommand{\pid}{\mbox{${\textstyle\frac{\pi}{2}}$}}
\newcommand{\pic}{\mbox{${\textstyle\frac{\pi}{4}}$}}
\newcommand{\pitt}{\mbox{${\textstyle\frac{\pi}{3}}$}}
\newcommand{\pis}{\mbox{${\textstyle\frac{\pi}{6}}$}}
%\newcommand{\quart}{\mbox{$\case{1}{4}$}}
\renewcommand{\thepage}{-- \arabic{page} --}
\begin{document}
\begin{center}
\subsection*{Seminario de Mec\'anica Cu\'antica / \\Teor\'{\i}a de la Informaci\'on Cu\'antica}
\subsection*{Pr\'actica VII (Curso 2020)}
\end{center}
I. Hallar las energ\'{\i}as y autoestados exactos del Hamiltoniano 
\[H=\sum_{i=1}^n [bc^\dagger_ic_i-v(c^\dagger_ic_{i+l}+c^\dagger_{i+l}c_i)]\]
en los casos fermi\'onico y bos\'onico, para el caso c\'{\i}clico  $n+1\equiv 1$ y a) $l=1$ b) $l=2$.  Hallar tambi\'en la energ\'{\i}a fundamental si 
el n\'umero de part\'{\i}culas es $N=n/2$ (con $n$ par) e 
interpretar $H$  (Sugerencia: aplicar la transformada de Fourier discreta). 
\\ \\
II. Consideremos el Hamiltoniano  fermi\'onico
%Transformada de Fourier Cu\'antica. Algoritmo de Shor. \\
%Consideremos la T.F. discreta 
\[H=\frac{1}{2}\varepsilon\sum_{p=1}^\Omega (c^\dagger_{p+}c_{p+}-c^\dagger_{p-}c_{p-})-
\frac{1}{2}V\sum_{p\neq q}(c^\dagger_{p+}c^\dagger_{q+}c_{q-}c_{p-}+c^\dagger_{p-}c^\dagger_{q-}c_{q+}c_{p+})-
G\sum_{p\neq q}c^\dagger_{p+}c^\dagger_{q-}c_{q+}c_{p-}\]
donde $V>0$, $G>0$ y $p,q=1\ldots,\Omega$. Plantear las ecuaciones de Hartree-Fock para el caso de $N=\Omega$ fermiones, hallando la soluci\'on de energ\'{\i}a m\'{\i}nima para 
$G>0$, $V>0$.  Identificar el umbral para ruptura de simetr\'{\i}a (cual?) en la aproximaci\'on. Hallar tambi\'en la energ\'{\i}a m\'{\i}nima en esta aproximaci\'on, y las energ\'{\i}as de part\'{\i}cula independiente. Discutir tambi\'en la soluci\'on exacta. 
\\ \\
III. Consideremos el Hamiltoniano de un sistema de $n$ espines $1/2$ en una cadena con interacci\'on de primeros vecinos tipo $XY$ y en un campo magn\'etico transverso $\propto B$:
\[H=\sum_i [B \sigma_{iz}-J_{x}\sigma_{ix}\sigma_{i+1,x}-J_{y}\sigma_{iy}\sigma_{i+1,y}]\] 
donde $i=1,\ldots n$ es un \'{\i}ndice de sitio y $\sigma_{i\mu}$ matrices de Pauli, 
con $n+1=1$ (condici\'on c\'{\i}clica). 
Hallar el estado separable de m\'{\i}nima energ\'{\i}a en el caso $J_{x}>0$, $|J_y|<J_x$, e identificar 
el umbral de $J_x$ para ruptura de simetr\'{\i}a (cual?) de la aproximaci\'on. 
Comparar con el resultado exacto para el caso particular $n=2$. \\ \\
IV. Sea  
%Transformada de Fourier Cu\'antica. Algoritmo de Shor. \\
%Consideremos la T.F. discreta 
\[H=\varepsilon\sum_p (c^\dagger_{p+}c_{p+}+c^\dagger_{p-}c_{p-})-
G\sum_{p, q}c^\dagger_{p+}c^\dagger_{p-}c_{q-}c_{q+}\]
donde $G>0$ y $p,q=1\ldots,\Omega$. \\
a) Plantear las ecuaciones de BCS para el presente sistema 
y hallar la soluci\'on de energ\'{\i}a m\'{\i}nima para el caso 
$G>0$ y $N=\Omega$. Identificar el 
umbral de $G$ para ruptura de simetr\'{\i}a en la aproximaci\'on (umbral de soluci\'on superconductora), 
y determinar, en funci\'on de $G$, el gap $\Delta$, el potencial qu\'{\i}mico $\mu$,  la energ\'{\i}a del estado fundamental en la aproximaci\'on BCS y las energ\'{\i}as de cuasipart\'{\i}cula. Determinar tambi\'en 
el estado fundamental en esta aproximaci\'on y la fluctuaci\'on $\langle N^2\rangle-\langle N\rangle^2$ 
del n\'umero de part\'{\i}culas. \\
b) Hallar los niveles de energ\'{\i}a exactos del sistema y comparar con los resultados anterio- res.\\ \\
%fundamental exacta.  y la fluctuaci\'on $\langle N^2\rangle-\langle N\rangle^2$ 
%del n\'umero de part\'{\i}culas. 
%\\ \\
%Consideremos Hamiltoniano  fermi\'onico 
%Transformada de 	 Fourier Cu\'antica. Algoritmo de Shor. \\
%Consideremos la T.F. discreta 
%\[H=\varepsilon\sum_p (c^\dagger_{p+}c_{p+}+c^\dagger_{p-}c_{p-})-
%G\sum_{p\neq q}c^\dagger_{p+}c^\dagger_{p-}c_{q-}c_{q+}\]
%donde $G>0$ y $p=1\ldots,\Omega$. Plantear las ecuaciones de BCS para el presente sistema 
%y hallar la soluci\'on de energ\'{\i}a m\'{\i}nima para el caso 
%$G>0$ y $N=\Omega$. Identificar el 
%punto de ruptura de simetr\'{\i}a en la aproximaci\'on (umbral de soluci\'on superconductora) 
%y el valor del gap $\Delta$. Hallar tambi\'en la energ\'{\i}a fundamental exacta. 
\end{document}